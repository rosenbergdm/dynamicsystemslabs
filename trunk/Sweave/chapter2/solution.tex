%
%  Created by David Rosenberg on 2009-09-10.
%  Copyright (c) 2009 University of Chicago. All rights reserved.
%
\documentclass[10pt,letterpaper]{article}
\usepackage[Rpdflatex]{Rosenberg}
\usepackage{svn-multi}
\usepackage{tikz}
\pgfrealjobname{solution}

\svnidlong
{$LastChangedDate: 2009-11-05 03:39:38 -0600 (Thu, 05 Nov 2009) $}
{$LastChangedRevision: 198 $}
{$LastChangedBy: root $}
% \svnid{$Id: example_main.tex 146 2008-12-03 13:29:19Z martin $}
% Don't forget to set the svn property 'svn:keywords' to
% 'HeadURL LastChangedDate LastChangedRevision LastChangedBy' or
% 'Id' or both depending if you use \svnidlong and/or \svnid
%
\newcommand{\svnfooter}{Last Changed Rev: \svnkw{LastChangedRevision}}
\svnRegisterAuthor{davidrosenberg}{David M. Rosenberg}

%\usepackage{Sweave}




\title{Introduction to computational programming\\\smaller Chapter 2 Lab Exercise \\\smaller Cobweb plots and stability}
\author{David M. Rosenberg\\\small University of Chicago\\\small Committee on Neurobiology\medskip\\
{\footnotesize \parbox[t]{10cm} {
Version control information:
\begin{tabbing}
\footnotesize\sffamily
 Last changes revision: \= \kill
 Last changed date: \> \svndate\\
 Last changes revision: \> \svnrev\\
 Version: \> \svnFullRevision*{\svnrev}\\
 Last changed by: \> David M. Rosenberg\\
\end{tabbing}
}
}}
\begin{document}


%\tracingonline=1
%\tracingmacros=2


\maketitle

\part{Exercise} % (fold)
\label{prt:exercise}

\begin{enumerate}
  \item Find the equilibrium values of populations governed by the following equations, and determine their stability analytically. Is there is a stable nonzero equilibrium  (carrying capacity) that the population may approach in the long run?
  \begin{enumerate}
    \item $N_{t+1} = 41N_t -10N_t^2$
    \item $N_{t+1} = 41N_t +2N_t^2$
  \end{enumerate}
\end{enumerate}

% \newcommand{\RBenumi}{\theenumi}
% \renewcommand{\theenumi}{\theenumii}

\begin{Solution}
  \begin{enumerate}
    \begin{enumerate}
\item
    \begin{align*}
        N_{t+1} &= 41 N_t - 10 N_t^2 \\
        \text{Let } x &= N_{t+1} = N_t \\
        x &= 41 x - 10 x^2 \\
        0 &= 10 x (4 - x) \\
        x &\in \{ 0, 4 \} 
      \end{align*}
\item
    \begin{align*}
        N_{t+1} &= 41 N_t - 2 N_t^2\\
        \text{Let } x &= N_{t+1} = N_t \\
        x &= 41 x - 2 x^2 \\
        0 &= 2 x (20 - x) \\
        x  &\in \{ 0, 20 \}
    \end{align*}
\end{enumerate}
\end{enumerate}
\end{Solution}

% \renewcommand{\theenumi}{\RBenumi}
%
      
\begin{enumerate}
  \setcounter{enumi}{2}
  \item Generate cobweb plots for the following logistic models.  Graphically identify all fixed points, and state whether they are ``stable.''
  \begin{enumerate}
    \item $f(x) = 3x - \frac{3x^2}{4} + 1$
    \item $f(x) = 100x -2x^2$
    \item $f(x) = -100 x + \frac{x^2}{2}$
  \end{enumerate}
  \item Write a \emph{functor} which takes as an argument a function describing a logistic model and generates a cobweb plot.
\end{enumerate}


\begin{Solution}
\begin{Schunk}
\begin{Sinput}
 f1 <- function (x) 
   x * (5 - 4 * x)
 f2 <- function (x) 
   2 * x * (1 - 3 * x / 2)
 f3 <- function (x)
   x / 2 - x^2 / 5
 f4 <- function (x)
   x * (5/2 - 7 * x)
\end{Sinput}
\end{Schunk}
\end{Solution}
\begin{Solution}
\begin{Schunk}
\begin{Sinput}
 plotCobWeb <- function (f, max_iter=50) {
   ## initialize variables
   f_exp <- deparse(body(f))
   df <- mDeriv(f);
   df_exp <- deparse(body(df))
   fixed_pts <- sort(mSolve.RServe(paste(f_exp, 'x', sep='=')));
   zeros <- sort(mSolve.RServe(paste(f_exp, '0', sep='=')));
 
   rate <- df(0);
   carrying_cap <- max(zeros);
   
   init_value <- min(zeros) + diff(zeros) / 4;
   y <- x <- numeric(length=max_iter*2);
   x[1] <- init_value;
   y[1] <- 0;
 
   ## Loop over iterations
 
   for (ii in seq(along=1:max_iter)) {
     y[2 * ii + c(0,1)] <- f(x[2 * ii - 1]);
     x[2 * ii + c(0,1)] <- c(x[2 * ii - 1], y[2 * ii + 1]);
   }
 
   ## Determine plot limits
   if (all(is.finite(x) & is.finite(y))) {
     xlim <- c(min(floor(zeros)), max(ceiling(zeros)));
     ylim <- c(  max(c(0, min(f(c(zeros, fixed_pts, mean(zeros)))))),
                 max(f(c(zeros, fixed_pts, mean(zeros)))))
   } else {
     yranges <- xranges <- c(1);
     for (ii in 2:length(x)) {
       xranges <- c(xranges, diff(range(x[1:ii])));
       yranges <- c(yranges, diff(range(y[1:ii])));
     }
     xlim <- c(min(floor(zeros)), max(ceiling(zeros)));
     ymagnitudes <- na.omit(yranges[-1] /  yranges[-length(yranges)])
     first_runaway <- min( (1:length(ymagnitudes) )[ymagnitudes > 100])
     
     ylim <- c(0, max(c(y[1:(first_runaway - 1)], f(mean(zeros)), f(zeros), 
               f(fixed_pts))));
   }
 
   ## Draw plot elements
   curve(f, from=xlim[1], to=xlim[2], ylim=ylim, xlim=c(min(zeros), 
         max(zeros)), fg=gray(0.6), bty='n',col='red', xlab="$t$", 
         ylab="$f(t)$");
   abline(h=0, lwd=1.5);
   abline(v=0, lwd=1.5);
   abline(0, 1, col='blue', lwd=1);
   lines(x, y, col='darkgreen', lwd=1);
 
   ########################################################
   ##  Page break only: Function continues on next page  ##
   ########################################################
 }
\end{Sinput}
\end{Schunk}
\end{Solution}
\begin{Solution}
\begin{Schunk}
\begin{Sinput}
 {
   ##########################################################
   ##  Page break only: Function continues from last page  ##
   ##########################################################
 
   ## Calculate label positions
   typ <- text_y_positions <- mean(ylim) - diff(ylim) / 15 * c(4, 5, 6, 7, 8);
   txp <- mean(zeros)
 
   ## Add plot description
   title(main=paste("{\\larger\\bf Cobweb plot of $ f(x) = ",
         gsub("\\*", "", f_exp),"$ }"));
   text(   x=mean(zeros), y=typ[1], 
           paste(  "Fixed points: $ \\{ ", paste(as.character(fixed_pts), 
                 collapse=", "), " $ \\} \nCarrying capacity $ k=", 
                 carrying_cap, " $"));
   if(rate < 3 && rate > 1) {
     text(x=txp, y=typ[3], paste('Rate: $ r= ', rate,
                                 "\\quad \\rightarrow 1 < r < 3$"));
     text(x=txp, y=typ[4], "{\\bf $\\therefore $ the fixed point is stable }");
 
   } else {
     text(x=txp, y=typ[3], 
          paste('Rate: $r = ', rate, "\\quad \\rightarrow r > 3$") );
     text(x=txp, y=typ[4], 
          "{\\bf $\\therefore $ the fixed point is unstable }");
   }
 }
\end{Sinput}
\end{Schunk}
\end{Solution}
\begin{Solution}

\end{Solution}
\begin{center}
\begin{Schunk}
\begin{Sinput}
 f <- f1;
 plotCobWeb(f)
\end{Sinput}
\end{Schunk}
\beginpgfgraphicnamed{solution-functor_def11}
% Created by Eps2pgf 0.7.0 (build on 2008-08-24) on Thu Nov 05 06:11:32 CST 2009
\begin{pgfpicture}
\pgfpathmoveto{\pgfqpoint{0cm}{0cm}}
\pgfpathlineto{\pgfqpoint{12.7cm}{0cm}}
\pgfpathlineto{\pgfqpoint{12.7cm}{12.7cm}}
\pgfpathlineto{\pgfqpoint{0cm}{12.7cm}}
\pgfpathclose
\pgfusepath{clip}
\begin{pgfscope}
\begin{pgfscope}
\end{pgfscope}
\begin{pgfscope}
\pgfpathmoveto{\pgfqpoint{2.083cm}{2.591cm}}
\pgfpathlineto{\pgfqpoint{11.633cm}{2.591cm}}
\pgfpathlineto{\pgfqpoint{11.633cm}{10.617cm}}
\pgfpathlineto{\pgfqpoint{2.083cm}{10.617cm}}
\pgfpathclose
\pgfusepath{clip}
\pgfsetdash{}{0cm}
\pgfsetlinewidth{0.265mm}
\pgfsetroundcap
\pgfsetroundjoin
\definecolor{eps2pgf_color}{rgb}{1,0,0}\pgfsetstrokecolor{eps2pgf_color}\pgfsetfillcolor{eps2pgf_color}
\pgfpathmoveto{\pgfqpoint{2.437cm}{2.888cm}}
\pgfpathlineto{\pgfqpoint{2.578cm}{3.356cm}}
\pgfpathlineto{\pgfqpoint{2.72cm}{3.809cm}}
\pgfpathlineto{\pgfqpoint{2.861cm}{4.246cm}}
\pgfpathlineto{\pgfqpoint{3.002cm}{4.669cm}}
\pgfpathlineto{\pgfqpoint{3.144cm}{5.076cm}}
\pgfpathlineto{\pgfqpoint{3.285cm}{5.468cm}}
\pgfpathlineto{\pgfqpoint{3.427cm}{5.844cm}}
\pgfpathlineto{\pgfqpoint{3.568cm}{6.206cm}}
\pgfpathlineto{\pgfqpoint{3.71cm}{6.552cm}}
\pgfpathlineto{\pgfqpoint{3.851cm}{6.883cm}}
\pgfpathlineto{\pgfqpoint{3.993cm}{7.199cm}}
\pgfpathlineto{\pgfqpoint{4.134cm}{7.5cm}}
\pgfpathlineto{\pgfqpoint{4.276cm}{7.785cm}}
\pgfpathlineto{\pgfqpoint{4.417cm}{8.055cm}}
\pgfpathlineto{\pgfqpoint{4.559cm}{8.31cm}}
\pgfpathlineto{\pgfqpoint{4.7cm}{8.55cm}}
\pgfpathlineto{\pgfqpoint{4.842cm}{8.775cm}}
\pgfpathlineto{\pgfqpoint{4.983cm}{8.984cm}}
\pgfpathlineto{\pgfqpoint{5.125cm}{9.178cm}}
\pgfpathlineto{\pgfqpoint{5.266cm}{9.357cm}}
\pgfpathlineto{\pgfqpoint{5.408cm}{9.52cm}}
\pgfpathlineto{\pgfqpoint{5.549cm}{9.669cm}}
\pgfpathlineto{\pgfqpoint{5.691cm}{9.802cm}}
\pgfpathlineto{\pgfqpoint{5.832cm}{9.92cm}}
\pgfpathlineto{\pgfqpoint{5.974cm}{10.023cm}}
\pgfpathlineto{\pgfqpoint{6.115cm}{10.11cm}}
\pgfpathlineto{\pgfqpoint{6.257cm}{10.183cm}}
\pgfpathlineto{\pgfqpoint{6.398cm}{10.239cm}}
\pgfpathlineto{\pgfqpoint{6.54cm}{10.281cm}}
\pgfpathlineto{\pgfqpoint{6.681cm}{10.308cm}}
\pgfpathlineto{\pgfqpoint{6.823cm}{10.319cm}}
\pgfpathlineto{\pgfqpoint{6.964cm}{10.316cm}}
\pgfpathlineto{\pgfqpoint{7.106cm}{10.297cm}}
\pgfpathlineto{\pgfqpoint{7.247cm}{10.262cm}}
\pgfpathlineto{\pgfqpoint{7.389cm}{10.213cm}}
\pgfpathlineto{\pgfqpoint{7.53cm}{10.148cm}}
\pgfpathlineto{\pgfqpoint{7.672cm}{10.068cm}}
\pgfpathlineto{\pgfqpoint{7.813cm}{9.973cm}}
\pgfpathlineto{\pgfqpoint{7.954cm}{9.863cm}}
\pgfpathlineto{\pgfqpoint{8.096cm}{9.737cm}}
\pgfpathlineto{\pgfqpoint{8.237cm}{9.597cm}}
\pgfpathlineto{\pgfqpoint{8.379cm}{9.44cm}}
\pgfpathlineto{\pgfqpoint{8.521cm}{9.269cm}}
\pgfpathlineto{\pgfqpoint{8.662cm}{9.083cm}}
\pgfpathlineto{\pgfqpoint{8.804cm}{8.881cm}}
\pgfpathlineto{\pgfqpoint{8.945cm}{8.664cm}}
\pgfpathlineto{\pgfqpoint{9.086cm}{8.432cm}}
\pgfpathlineto{\pgfqpoint{9.228cm}{8.185cm}}
\pgfpathlineto{\pgfqpoint{9.369cm}{7.922cm}}
\pgfpathlineto{\pgfqpoint{9.511cm}{7.644cm}}
\pgfpathlineto{\pgfqpoint{9.652cm}{7.352cm}}
\pgfpathlineto{\pgfqpoint{9.794cm}{7.043cm}}
\pgfpathlineto{\pgfqpoint{9.935cm}{6.72cm}}
\pgfpathlineto{\pgfqpoint{10.077cm}{6.381cm}}
\pgfpathlineto{\pgfqpoint{10.218cm}{6.027cm}}
\pgfpathlineto{\pgfqpoint{10.36cm}{5.658cm}}
\pgfpathlineto{\pgfqpoint{10.501cm}{5.274cm}}
\pgfpathlineto{\pgfqpoint{10.643cm}{4.874cm}}
\pgfpathlineto{\pgfqpoint{10.784cm}{4.459cm}}
\pgfpathlineto{\pgfqpoint{10.926cm}{4.03cm}}
\pgfpathlineto{\pgfqpoint{11.067cm}{3.585cm}}
\pgfpathlineto{\pgfqpoint{11.209cm}{3.124cm}}
\pgfpathlineto{\pgfqpoint{11.35cm}{2.648cm}}
\pgfpathlineto{\pgfqpoint{11.492cm}{2.158cm}}
\pgfpathlineto{\pgfqpoint{11.633cm}{1.651cm}}
\pgfpathlineto{\pgfqpoint{11.775cm}{1.13cm}}
\pgfpathlineto{\pgfqpoint{11.916cm}{0.594cm}}
\pgfpathlineto{\pgfqpoint{12.058cm}{0.042cm}}
\pgfpathlineto{\pgfqpoint{12.068cm}{-0cm}}
\pgfusepath{stroke}
\end{pgfscope}
\begin{pgfscope}
\pgfpathmoveto{\pgfqpoint{0cm}{0cm}}
\pgfpathlineto{\pgfqpoint{12.7cm}{0cm}}
\pgfpathlineto{\pgfqpoint{12.7cm}{12.7cm}}
\pgfpathlineto{\pgfqpoint{0cm}{12.7cm}}
\pgfpathclose
\pgfusepath{clip}
\pgfsetdash{}{0cm}
\pgfsetlinewidth{0.265mm}
\pgfsetroundcap
\pgfsetroundjoin
\definecolor{eps2pgf_color}{gray}{0.6}\pgfsetstrokecolor{eps2pgf_color}\pgfsetfillcolor{eps2pgf_color}
\pgfpathmoveto{\pgfqpoint{2.437cm}{2.591cm}}
\pgfpathlineto{\pgfqpoint{10.926cm}{2.591cm}}
\pgfusepath{stroke}
\pgfsetdash{}{0cm}
\pgfpathmoveto{\pgfqpoint{2.437cm}{2.591cm}}
\pgfpathlineto{\pgfqpoint{2.437cm}{2.337cm}}
\pgfusepath{stroke}
\pgfsetdash{}{0cm}
\pgfpathmoveto{\pgfqpoint{3.851cm}{2.591cm}}
\pgfpathlineto{\pgfqpoint{3.851cm}{2.337cm}}
\pgfusepath{stroke}
\pgfsetdash{}{0cm}
\pgfpathmoveto{\pgfqpoint{5.266cm}{2.591cm}}
\pgfpathlineto{\pgfqpoint{5.266cm}{2.337cm}}
\pgfusepath{stroke}
\pgfsetdash{}{0cm}
\pgfpathmoveto{\pgfqpoint{6.681cm}{2.591cm}}
\pgfpathlineto{\pgfqpoint{6.681cm}{2.337cm}}
\pgfusepath{stroke}
\pgfsetdash{}{0cm}
\pgfpathmoveto{\pgfqpoint{8.096cm}{2.591cm}}
\pgfpathlineto{\pgfqpoint{8.096cm}{2.337cm}}
\pgfusepath{stroke}
\pgfsetdash{}{0cm}
\pgfpathmoveto{\pgfqpoint{9.511cm}{2.591cm}}
\pgfpathlineto{\pgfqpoint{9.511cm}{2.337cm}}
\pgfusepath{stroke}
\pgfsetdash{}{0cm}
\pgfpathmoveto{\pgfqpoint{10.926cm}{2.591cm}}
\pgfpathlineto{\pgfqpoint{10.926cm}{2.337cm}}
\pgfusepath{stroke}
\begin{pgfscope}
\definecolor{eps2pgf_color}{gray}{0}\pgfsetstrokecolor{eps2pgf_color}\pgfsetfillcolor{eps2pgf_color}
\pgftext[x=2.437cm,y=1.821cm,rotate=0]{0.0}
\end{pgfscope}
\begin{pgfscope}
\definecolor{eps2pgf_color}{gray}{0}\pgfsetstrokecolor{eps2pgf_color}\pgfsetfillcolor{eps2pgf_color}
\pgftext[x=3.849cm,y=1.821cm,rotate=0]{0.2}
\end{pgfscope}
\begin{pgfscope}
\definecolor{eps2pgf_color}{gray}{0}\pgfsetstrokecolor{eps2pgf_color}\pgfsetfillcolor{eps2pgf_color}
\pgftext[x=5.267cm,y=1.821cm,rotate=0]{0.4}
\end{pgfscope}
\begin{pgfscope}
\definecolor{eps2pgf_color}{gray}{0}\pgfsetstrokecolor{eps2pgf_color}\pgfsetfillcolor{eps2pgf_color}
\pgftext[x=6.681cm,y=1.821cm,rotate=0]{0.6}
\end{pgfscope}
\begin{pgfscope}
\definecolor{eps2pgf_color}{gray}{0}\pgfsetstrokecolor{eps2pgf_color}\pgfsetfillcolor{eps2pgf_color}
\pgftext[x=8.095cm,y=1.821cm,rotate=0]{0.8}
\end{pgfscope}
\begin{pgfscope}
\definecolor{eps2pgf_color}{gray}{0}\pgfsetstrokecolor{eps2pgf_color}\pgfsetfillcolor{eps2pgf_color}
\pgftext[x=9.524cm,y=1.821cm,rotate=0]{1.0}
\end{pgfscope}
\begin{pgfscope}
\definecolor{eps2pgf_color}{gray}{0}\pgfsetstrokecolor{eps2pgf_color}\pgfsetfillcolor{eps2pgf_color}
\pgftext[x=10.937cm,y=1.825cm,rotate=0]{1.2}
\end{pgfscope}
\pgfsetdash{}{0cm}
\pgfpathmoveto{\pgfqpoint{2.083cm}{2.888cm}}
\pgfpathlineto{\pgfqpoint{2.083cm}{10.023cm}}
\pgfusepath{stroke}
\pgfsetdash{}{0cm}
\pgfpathmoveto{\pgfqpoint{2.083cm}{2.888cm}}
\pgfpathlineto{\pgfqpoint{1.829cm}{2.888cm}}
\pgfusepath{stroke}
\pgfsetdash{}{0cm}
\pgfpathmoveto{\pgfqpoint{2.083cm}{5.266cm}}
\pgfpathlineto{\pgfqpoint{1.829cm}{5.266cm}}
\pgfusepath{stroke}
\pgfsetdash{}{0cm}
\pgfpathmoveto{\pgfqpoint{2.083cm}{7.644cm}}
\pgfpathlineto{\pgfqpoint{1.829cm}{7.644cm}}
\pgfusepath{stroke}
\pgfsetdash{}{0cm}
\pgfpathmoveto{\pgfqpoint{2.083cm}{10.023cm}}
\pgfpathlineto{\pgfqpoint{1.829cm}{10.023cm}}
\pgfusepath{stroke}
\begin{pgfscope}
\definecolor{eps2pgf_color}{gray}{0}\pgfsetstrokecolor{eps2pgf_color}\pgfsetfillcolor{eps2pgf_color}
\pgftext[x=1.328cm,y=2.888cm,rotate=90]{0.0}
\end{pgfscope}
\begin{pgfscope}
\definecolor{eps2pgf_color}{gray}{0}\pgfsetstrokecolor{eps2pgf_color}\pgfsetfillcolor{eps2pgf_color}
\pgftext[x=1.328cm,y=5.265cm,rotate=90]{0.5}
\end{pgfscope}
\begin{pgfscope}
\definecolor{eps2pgf_color}{gray}{0}\pgfsetstrokecolor{eps2pgf_color}\pgfsetfillcolor{eps2pgf_color}
\pgftext[x=1.328cm,y=7.658cm,rotate=90]{1.0}
\end{pgfscope}
\begin{pgfscope}
\definecolor{eps2pgf_color}{gray}{0}\pgfsetstrokecolor{eps2pgf_color}\pgfsetfillcolor{eps2pgf_color}
\pgftext[x=1.328cm,y=10.035cm,rotate=90]{1.5}
\end{pgfscope}
\end{pgfscope}
\begin{pgfscope}
\pgfpathmoveto{\pgfqpoint{0cm}{0cm}}
\pgfpathlineto{\pgfqpoint{12.7cm}{0cm}}
\pgfpathlineto{\pgfqpoint{12.7cm}{12.7cm}}
\pgfpathlineto{\pgfqpoint{0cm}{12.7cm}}
\pgfpathclose
\pgfusepath{clip}
\begin{pgfscope}
\definecolor{eps2pgf_color}{gray}{0}\pgfsetstrokecolor{eps2pgf_color}\pgfsetfillcolor{eps2pgf_color}
\pgftext[x=6.857cm,y=0.8cm,rotate=0]{$t$}
\end{pgfscope}
\begin{pgfscope}
\definecolor{eps2pgf_color}{gray}{0}\pgfsetstrokecolor{eps2pgf_color}\pgfsetfillcolor{eps2pgf_color}
\pgftext[x=0.337cm,y=6.603cm,rotate=90]{$f(t)$}
\end{pgfscope}
\end{pgfscope}
\begin{pgfscope}
\pgfpathmoveto{\pgfqpoint{2.083cm}{2.591cm}}
\pgfpathlineto{\pgfqpoint{11.633cm}{2.591cm}}
\pgfpathlineto{\pgfqpoint{11.633cm}{10.617cm}}
\pgfpathlineto{\pgfqpoint{2.083cm}{10.617cm}}
\pgfpathclose
\pgfusepath{clip}
\pgfsetdash{}{0cm}
\pgfsetlinewidth{0.395mm}
\pgfsetroundcap
\pgfsetroundjoin
\definecolor{eps2pgf_color}{gray}{0}\pgfsetstrokecolor{eps2pgf_color}\pgfsetfillcolor{eps2pgf_color}
\pgfpathmoveto{\pgfqpoint{2.083cm}{2.888cm}}
\pgfpathlineto{\pgfqpoint{11.633cm}{2.888cm}}
\pgfusepath{stroke}
\pgfsetdash{}{0cm}
\pgfpathmoveto{\pgfqpoint{2.437cm}{2.591cm}}
\pgfpathlineto{\pgfqpoint{2.437cm}{10.617cm}}
\pgfusepath{stroke}
\pgfsetdash{}{0cm}
\pgfsetlinewidth{0.265mm}
\definecolor{eps2pgf_color}{rgb}{0,0,1}\pgfsetstrokecolor{eps2pgf_color}\pgfsetfillcolor{eps2pgf_color}
\pgfpathmoveto{\pgfqpoint{2.083cm}{2.65cm}}
\pgfpathlineto{\pgfqpoint{11.633cm}{9.071cm}}
\pgfusepath{stroke}
\pgfsetdash{}{0cm}
\definecolor{eps2pgf_color}{rgb}{0,0.3922,0}\pgfsetstrokecolor{eps2pgf_color}\pgfsetfillcolor{eps2pgf_color}
\pgfpathmoveto{\pgfqpoint{4.647cm}{2.888cm}}
\pgfpathlineto{\pgfqpoint{4.647cm}{8.462cm}}
\pgfpathlineto{\pgfqpoint{10.727cm}{8.462cm}}
\pgfpathlineto{\pgfqpoint{10.727cm}{4.63cm}}
\pgfpathlineto{\pgfqpoint{5.027cm}{4.63cm}}
\pgfpathlineto{\pgfqpoint{5.027cm}{9.046cm}}
\pgfpathlineto{\pgfqpoint{11.595cm}{9.046cm}}
\pgfpathlineto{\pgfqpoint{11.595cm}{1.789cm}}
\pgfpathlineto{\pgfqpoint{0.802cm}{1.789cm}}
\pgfpathlineto{\pgfqpoint{0.802cm}{0cm}}
\pgfusepath{stroke}
\end{pgfscope}
\begin{pgfscope}
\pgfpathmoveto{\pgfqpoint{0cm}{0cm}}
\pgfpathlineto{\pgfqpoint{12.7cm}{0cm}}
\pgfpathlineto{\pgfqpoint{12.7cm}{12.7cm}}
\pgfpathlineto{\pgfqpoint{0cm}{12.7cm}}
\pgfpathclose
\pgfusepath{clip}
\begin{pgfscope}
\definecolor{eps2pgf_color}{gray}{0}\pgfsetstrokecolor{eps2pgf_color}\pgfsetfillcolor{eps2pgf_color}
\pgftext[x=6.858cm,y=11.619cm,rotate=0]{{\larger\bf Cobweb plot of $ f(x) =  x  (5 - 4  x) $ }}
\end{pgfscope}
\end{pgfscope}
\begin{pgfscope}
\pgfpathmoveto{\pgfqpoint{2.083cm}{2.591cm}}
\pgfpathlineto{\pgfqpoint{11.633cm}{2.591cm}}
\pgfpathlineto{\pgfqpoint{11.633cm}{10.617cm}}
\pgfpathlineto{\pgfqpoint{2.083cm}{10.617cm}}
\pgfpathclose
\pgfusepath{clip}
\begin{pgfscope}
\definecolor{eps2pgf_color}{gray}{0}\pgfsetstrokecolor{eps2pgf_color}\pgfsetfillcolor{eps2pgf_color}
\pgftext[x=6.817cm,y=4.844cm,rotate=0]{Fixed points: $ \{  0, 1  $ \} }
\end{pgfscope}
\begin{pgfscope}
\definecolor{eps2pgf_color}{gray}{0}\pgfsetstrokecolor{eps2pgf_color}\pgfsetfillcolor{eps2pgf_color}
\pgftext[x=6.86cm,y=4.334cm,rotate=0]{Carrying capacity $ k= 1.25  $}
\end{pgfscope}
\begin{pgfscope}
\definecolor{eps2pgf_color}{gray}{0}\pgfsetstrokecolor{eps2pgf_color}\pgfsetfillcolor{eps2pgf_color}
\pgftext[x=6.869cm,y=3.631cm,rotate=0]{Rate: $r =  5 \quad \rightarrow r > 3$}
\end{pgfscope}
\begin{pgfscope}
\definecolor{eps2pgf_color}{gray}{0}\pgfsetstrokecolor{eps2pgf_color}\pgfsetfillcolor{eps2pgf_color}
\pgftext[x=6.858cm,y=3.136cm,rotate=0]{{\bf $\therefore $ the fixed point is unstable }}
\end{pgfscope}
\end{pgfscope}
\end{pgfscope}
\end{pgfpicture}

\endpgfgraphicnamed
\begin{Schunk}
\begin{Sinput}
 f <- f2
 plotCobWeb(f2)
\end{Sinput}
\end{Schunk}
\beginpgfgraphicnamed{solution-functor_def12}
% Created by Eps2pgf 0.7.0 (build on 2008-08-24) on Thu Nov 05 06:11:36 CST 2009
\begin{pgfpicture}
\pgfpathmoveto{\pgfqpoint{0cm}{0cm}}
\pgfpathlineto{\pgfqpoint{12.7cm}{0cm}}
\pgfpathlineto{\pgfqpoint{12.7cm}{12.7cm}}
\pgfpathlineto{\pgfqpoint{0cm}{12.7cm}}
\pgfpathclose
\pgfusepath{clip}
\begin{pgfscope}
\begin{pgfscope}
\end{pgfscope}
\begin{pgfscope}
\pgfpathmoveto{\pgfqpoint{2.083cm}{2.591cm}}
\pgfpathlineto{\pgfqpoint{11.633cm}{2.591cm}}
\pgfpathlineto{\pgfqpoint{11.633cm}{10.617cm}}
\pgfpathlineto{\pgfqpoint{2.083cm}{10.617cm}}
\pgfpathclose
\pgfusepath{clip}
\pgfsetdash{}{0cm}
\pgfsetlinewidth{0.265mm}
\pgfsetroundcap
\pgfsetroundjoin
\definecolor{eps2pgf_color}{rgb}{1,0,0}\pgfsetstrokecolor{eps2pgf_color}\pgfsetfillcolor{eps2pgf_color}
\pgfpathmoveto{\pgfqpoint{2.437cm}{2.888cm}}
\pgfpathlineto{\pgfqpoint{2.569cm}{3.327cm}}
\pgfpathlineto{\pgfqpoint{2.702cm}{3.753cm}}
\pgfpathlineto{\pgfqpoint{2.835cm}{4.166cm}}
\pgfpathlineto{\pgfqpoint{2.967cm}{4.565cm}}
\pgfpathlineto{\pgfqpoint{3.1cm}{4.951cm}}
\pgfpathlineto{\pgfqpoint{3.233cm}{5.323cm}}
\pgfpathlineto{\pgfqpoint{3.365cm}{5.682cm}}
\pgfpathlineto{\pgfqpoint{3.498cm}{6.027cm}}
\pgfpathlineto{\pgfqpoint{3.63cm}{6.36cm}}
\pgfpathlineto{\pgfqpoint{3.763cm}{6.678cm}}
\pgfpathlineto{\pgfqpoint{3.896cm}{6.984cm}}
\pgfpathlineto{\pgfqpoint{4.028cm}{7.276cm}}
\pgfpathlineto{\pgfqpoint{4.161cm}{7.554cm}}
\pgfpathlineto{\pgfqpoint{4.294cm}{7.82cm}}
\pgfpathlineto{\pgfqpoint{4.426cm}{8.072cm}}
\pgfpathlineto{\pgfqpoint{4.559cm}{8.31cm}}
\pgfpathlineto{\pgfqpoint{4.692cm}{8.535cm}}
\pgfpathlineto{\pgfqpoint{4.824cm}{8.747cm}}
\pgfpathlineto{\pgfqpoint{4.957cm}{8.946cm}}
\pgfpathlineto{\pgfqpoint{5.09cm}{9.131cm}}
\pgfpathlineto{\pgfqpoint{5.222cm}{9.302cm}}
\pgfpathlineto{\pgfqpoint{5.355cm}{9.461cm}}
\pgfpathlineto{\pgfqpoint{5.487cm}{9.606cm}}
\pgfpathlineto{\pgfqpoint{5.62cm}{9.737cm}}
\pgfpathlineto{\pgfqpoint{5.753cm}{9.856cm}}
\pgfpathlineto{\pgfqpoint{5.885cm}{9.96cm}}
\pgfpathlineto{\pgfqpoint{6.018cm}{10.052cm}}
\pgfpathlineto{\pgfqpoint{6.151cm}{10.13cm}}
\pgfpathlineto{\pgfqpoint{6.283cm}{10.194cm}}
\pgfpathlineto{\pgfqpoint{6.416cm}{10.246cm}}
\pgfpathlineto{\pgfqpoint{6.549cm}{10.283cm}}
\pgfpathlineto{\pgfqpoint{6.681cm}{10.308cm}}
\pgfpathlineto{\pgfqpoint{6.814cm}{10.319cm}}
\pgfpathlineto{\pgfqpoint{6.947cm}{10.317cm}}
\pgfpathlineto{\pgfqpoint{7.079cm}{10.301cm}}
\pgfpathlineto{\pgfqpoint{7.212cm}{10.273cm}}
\pgfpathlineto{\pgfqpoint{7.344cm}{10.23cm}}
\pgfpathlineto{\pgfqpoint{7.477cm}{10.174cm}}
\pgfpathlineto{\pgfqpoint{7.61cm}{10.105cm}}
\pgfpathlineto{\pgfqpoint{7.742cm}{10.023cm}}
\pgfpathlineto{\pgfqpoint{7.875cm}{9.927cm}}
\pgfpathlineto{\pgfqpoint{8.008cm}{9.817cm}}
\pgfpathlineto{\pgfqpoint{8.14cm}{9.695cm}}
\pgfpathlineto{\pgfqpoint{8.273cm}{9.559cm}}
\pgfpathlineto{\pgfqpoint{8.406cm}{9.41cm}}
\pgfpathlineto{\pgfqpoint{8.538cm}{9.247cm}}
\pgfpathlineto{\pgfqpoint{8.671cm}{9.071cm}}
\pgfpathlineto{\pgfqpoint{8.804cm}{8.881cm}}
\pgfpathlineto{\pgfqpoint{8.936cm}{8.678cm}}
\pgfpathlineto{\pgfqpoint{9.069cm}{8.462cm}}
\pgfpathlineto{\pgfqpoint{9.202cm}{8.232cm}}
\pgfpathlineto{\pgfqpoint{9.334cm}{7.989cm}}
\pgfpathlineto{\pgfqpoint{9.467cm}{7.733cm}}
\pgfpathlineto{\pgfqpoint{9.599cm}{7.463cm}}
\pgfpathlineto{\pgfqpoint{9.732cm}{7.18cm}}
\pgfpathlineto{\pgfqpoint{9.865cm}{6.883cm}}
\pgfpathlineto{\pgfqpoint{9.997cm}{6.574cm}}
\pgfpathlineto{\pgfqpoint{10.13cm}{6.25cm}}
\pgfpathlineto{\pgfqpoint{10.263cm}{5.914cm}}
\pgfpathlineto{\pgfqpoint{10.395cm}{5.564cm}}
\pgfpathlineto{\pgfqpoint{10.528cm}{5.2cm}}
\pgfpathlineto{\pgfqpoint{10.661cm}{4.823cm}}
\pgfpathlineto{\pgfqpoint{10.793cm}{4.433cm}}
\pgfpathlineto{\pgfqpoint{10.926cm}{4.03cm}}
\pgfpathlineto{\pgfqpoint{11.059cm}{3.613cm}}
\pgfpathlineto{\pgfqpoint{11.191cm}{3.182cm}}
\pgfpathlineto{\pgfqpoint{11.324cm}{2.739cm}}
\pgfpathlineto{\pgfqpoint{11.456cm}{2.282cm}}
\pgfpathlineto{\pgfqpoint{11.589cm}{1.811cm}}
\pgfpathlineto{\pgfqpoint{11.722cm}{1.328cm}}
\pgfpathlineto{\pgfqpoint{11.854cm}{0.83cm}}
\pgfpathlineto{\pgfqpoint{11.987cm}{0.32cm}}
\pgfpathlineto{\pgfqpoint{12.068cm}{-0cm}}
\pgfusepath{stroke}
\end{pgfscope}
\begin{pgfscope}
\pgfpathmoveto{\pgfqpoint{0cm}{0cm}}
\pgfpathlineto{\pgfqpoint{12.7cm}{0cm}}
\pgfpathlineto{\pgfqpoint{12.7cm}{12.7cm}}
\pgfpathlineto{\pgfqpoint{0cm}{12.7cm}}
\pgfpathclose
\pgfusepath{clip}
\pgfsetdash{}{0cm}
\pgfsetlinewidth{0.265mm}
\pgfsetroundcap
\pgfsetroundjoin
\definecolor{eps2pgf_color}{gray}{0.6}\pgfsetstrokecolor{eps2pgf_color}\pgfsetfillcolor{eps2pgf_color}
\pgfpathmoveto{\pgfqpoint{2.437cm}{2.591cm}}
\pgfpathlineto{\pgfqpoint{10.395cm}{2.591cm}}
\pgfusepath{stroke}
\pgfsetdash{}{0cm}
\pgfpathmoveto{\pgfqpoint{2.437cm}{2.591cm}}
\pgfpathlineto{\pgfqpoint{2.437cm}{2.337cm}}
\pgfusepath{stroke}
\pgfsetdash{}{0cm}
\pgfpathmoveto{\pgfqpoint{3.763cm}{2.591cm}}
\pgfpathlineto{\pgfqpoint{3.763cm}{2.337cm}}
\pgfusepath{stroke}
\pgfsetdash{}{0cm}
\pgfpathmoveto{\pgfqpoint{5.09cm}{2.591cm}}
\pgfpathlineto{\pgfqpoint{5.09cm}{2.337cm}}
\pgfusepath{stroke}
\pgfsetdash{}{0cm}
\pgfpathmoveto{\pgfqpoint{6.416cm}{2.591cm}}
\pgfpathlineto{\pgfqpoint{6.416cm}{2.337cm}}
\pgfusepath{stroke}
\pgfsetdash{}{0cm}
\pgfpathmoveto{\pgfqpoint{7.742cm}{2.591cm}}
\pgfpathlineto{\pgfqpoint{7.742cm}{2.337cm}}
\pgfusepath{stroke}
\pgfsetdash{}{0cm}
\pgfpathmoveto{\pgfqpoint{9.069cm}{2.591cm}}
\pgfpathlineto{\pgfqpoint{9.069cm}{2.337cm}}
\pgfusepath{stroke}
\pgfsetdash{}{0cm}
\pgfpathmoveto{\pgfqpoint{10.395cm}{2.591cm}}
\pgfpathlineto{\pgfqpoint{10.395cm}{2.337cm}}
\pgfusepath{stroke}
\begin{pgfscope}
\definecolor{eps2pgf_color}{gray}{0}\pgfsetstrokecolor{eps2pgf_color}\pgfsetfillcolor{eps2pgf_color}
\pgftext[x=2.437cm,y=1.821cm,rotate=0]{0.0}
\end{pgfscope}
\begin{pgfscope}
\definecolor{eps2pgf_color}{gray}{0}\pgfsetstrokecolor{eps2pgf_color}\pgfsetfillcolor{eps2pgf_color}
\pgftext[x=3.729cm,y=1.821cm,rotate=0]{0.1}
\end{pgfscope}
\begin{pgfscope}
\definecolor{eps2pgf_color}{gray}{0}\pgfsetstrokecolor{eps2pgf_color}\pgfsetfillcolor{eps2pgf_color}
\pgftext[x=5.087cm,y=1.821cm,rotate=0]{0.2}
\end{pgfscope}
\begin{pgfscope}
\definecolor{eps2pgf_color}{gray}{0}\pgfsetstrokecolor{eps2pgf_color}\pgfsetfillcolor{eps2pgf_color}
\pgftext[x=6.417cm,y=1.821cm,rotate=0]{0.3}
\end{pgfscope}
\begin{pgfscope}
\definecolor{eps2pgf_color}{gray}{0}\pgfsetstrokecolor{eps2pgf_color}\pgfsetfillcolor{eps2pgf_color}
\pgftext[x=7.743cm,y=1.821cm,rotate=0]{0.4}
\end{pgfscope}
\begin{pgfscope}
\definecolor{eps2pgf_color}{gray}{0}\pgfsetstrokecolor{eps2pgf_color}\pgfsetfillcolor{eps2pgf_color}
\pgftext[x=9.068cm,y=1.821cm,rotate=0]{0.5}
\end{pgfscope}
\begin{pgfscope}
\definecolor{eps2pgf_color}{gray}{0}\pgfsetstrokecolor{eps2pgf_color}\pgfsetfillcolor{eps2pgf_color}
\pgftext[x=10.395cm,y=1.821cm,rotate=0]{0.6}
\end{pgfscope}
\pgfsetdash{}{0cm}
\pgfpathmoveto{\pgfqpoint{2.083cm}{2.888cm}}
\pgfpathlineto{\pgfqpoint{2.083cm}{9.577cm}}
\pgfusepath{stroke}
\pgfsetdash{}{0cm}
\pgfpathmoveto{\pgfqpoint{2.083cm}{2.888cm}}
\pgfpathlineto{\pgfqpoint{1.829cm}{2.888cm}}
\pgfusepath{stroke}
\pgfsetdash{}{0cm}
\pgfpathmoveto{\pgfqpoint{2.083cm}{4.003cm}}
\pgfpathlineto{\pgfqpoint{1.829cm}{4.003cm}}
\pgfusepath{stroke}
\pgfsetdash{}{0cm}
\pgfpathmoveto{\pgfqpoint{2.083cm}{5.118cm}}
\pgfpathlineto{\pgfqpoint{1.829cm}{5.118cm}}
\pgfusepath{stroke}
\pgfsetdash{}{0cm}
\pgfpathmoveto{\pgfqpoint{2.083cm}{6.233cm}}
\pgfpathlineto{\pgfqpoint{1.829cm}{6.233cm}}
\pgfusepath{stroke}
\pgfsetdash{}{0cm}
\pgfpathmoveto{\pgfqpoint{2.083cm}{7.347cm}}
\pgfpathlineto{\pgfqpoint{1.829cm}{7.347cm}}
\pgfusepath{stroke}
\pgfsetdash{}{0cm}
\pgfpathmoveto{\pgfqpoint{2.083cm}{8.462cm}}
\pgfpathlineto{\pgfqpoint{1.829cm}{8.462cm}}
\pgfusepath{stroke}
\pgfsetdash{}{0cm}
\pgfpathmoveto{\pgfqpoint{2.083cm}{9.577cm}}
\pgfpathlineto{\pgfqpoint{1.829cm}{9.577cm}}
\pgfusepath{stroke}
\begin{pgfscope}
\definecolor{eps2pgf_color}{gray}{0}\pgfsetstrokecolor{eps2pgf_color}\pgfsetfillcolor{eps2pgf_color}
\pgftext[x=1.328cm,y=2.888cm,rotate=90]{0.00}
\end{pgfscope}
\begin{pgfscope}
\definecolor{eps2pgf_color}{gray}{0}\pgfsetstrokecolor{eps2pgf_color}\pgfsetfillcolor{eps2pgf_color}
\pgftext[x=1.328cm,y=5.118cm,rotate=90]{0.10}
\end{pgfscope}
\begin{pgfscope}
\definecolor{eps2pgf_color}{gray}{0}\pgfsetstrokecolor{eps2pgf_color}\pgfsetfillcolor{eps2pgf_color}
\pgftext[x=1.328cm,y=7.347cm,rotate=90]{0.20}
\end{pgfscope}
\begin{pgfscope}
\definecolor{eps2pgf_color}{gray}{0}\pgfsetstrokecolor{eps2pgf_color}\pgfsetfillcolor{eps2pgf_color}
\pgftext[x=1.328cm,y=9.577cm,rotate=90]{0.30}
\end{pgfscope}
\end{pgfscope}
\begin{pgfscope}
\pgfpathmoveto{\pgfqpoint{0cm}{0cm}}
\pgfpathlineto{\pgfqpoint{12.7cm}{0cm}}
\pgfpathlineto{\pgfqpoint{12.7cm}{12.7cm}}
\pgfpathlineto{\pgfqpoint{0cm}{12.7cm}}
\pgfpathclose
\pgfusepath{clip}
\begin{pgfscope}
\definecolor{eps2pgf_color}{gray}{0}\pgfsetstrokecolor{eps2pgf_color}\pgfsetfillcolor{eps2pgf_color}
\pgftext[x=6.857cm,y=0.8cm,rotate=0]{$t$}
\end{pgfscope}
\begin{pgfscope}
\definecolor{eps2pgf_color}{gray}{0}\pgfsetstrokecolor{eps2pgf_color}\pgfsetfillcolor{eps2pgf_color}
\pgftext[x=0.337cm,y=6.603cm,rotate=90]{$f(t)$}
\end{pgfscope}
\end{pgfscope}
\begin{pgfscope}
\pgfpathmoveto{\pgfqpoint{2.083cm}{2.591cm}}
\pgfpathlineto{\pgfqpoint{11.633cm}{2.591cm}}
\pgfpathlineto{\pgfqpoint{11.633cm}{10.617cm}}
\pgfpathlineto{\pgfqpoint{2.083cm}{10.617cm}}
\pgfpathclose
\pgfusepath{clip}
\pgfsetdash{}{0cm}
\pgfsetlinewidth{0.395mm}
\pgfsetroundcap
\pgfsetroundjoin
\definecolor{eps2pgf_color}{gray}{0}\pgfsetstrokecolor{eps2pgf_color}\pgfsetfillcolor{eps2pgf_color}
\pgfpathmoveto{\pgfqpoint{2.083cm}{2.888cm}}
\pgfpathlineto{\pgfqpoint{11.633cm}{2.888cm}}
\pgfusepath{stroke}
\pgfsetdash{}{0cm}
\pgfpathmoveto{\pgfqpoint{2.437cm}{2.591cm}}
\pgfpathlineto{\pgfqpoint{2.437cm}{10.617cm}}
\pgfusepath{stroke}
\pgfsetdash{}{0cm}
\pgfsetlinewidth{0.265mm}
\definecolor{eps2pgf_color}{rgb}{0,0,1}\pgfsetstrokecolor{eps2pgf_color}\pgfsetfillcolor{eps2pgf_color}
\pgfpathmoveto{\pgfqpoint{2.083cm}{2.293cm}}
\pgfpathlineto{\pgfqpoint{8.274cm}{12.7cm}}
\pgfusepath{stroke}
\pgfsetdash{}{0cm}
\definecolor{eps2pgf_color}{rgb}{0,0.3922,0}\pgfsetstrokecolor{eps2pgf_color}\pgfsetfillcolor{eps2pgf_color}
\pgfpathmoveto{\pgfqpoint{4.647cm}{2.888cm}}
\pgfpathlineto{\pgfqpoint{4.647cm}{8.462cm}}
\pgfpathlineto{\pgfqpoint{5.753cm}{8.462cm}}
\pgfpathlineto{\pgfqpoint{5.753cm}{9.856cm}}
\pgfpathlineto{\pgfqpoint{6.582cm}{9.856cm}}
\pgfpathlineto{\pgfqpoint{6.582cm}{10.291cm}}
\pgfpathlineto{\pgfqpoint{6.841cm}{10.291cm}}
\pgfpathlineto{\pgfqpoint{6.841cm}{10.32cm}}
\pgfpathlineto{\pgfqpoint{6.858cm}{10.32cm}}
\pgfpathlineto{\pgfqpoint{6.858cm}{10.32cm}}
\pgfpathlineto{\pgfqpoint{6.858cm}{10.32cm}}
\pgfpathlineto{\pgfqpoint{6.858cm}{10.32cm}}
\pgfpathlineto{\pgfqpoint{6.858cm}{10.32cm}}
\pgfpathlineto{\pgfqpoint{6.858cm}{10.32cm}}
\pgfpathlineto{\pgfqpoint{6.858cm}{10.32cm}}
\pgfpathlineto{\pgfqpoint{6.858cm}{10.32cm}}
\pgfpathlineto{\pgfqpoint{6.858cm}{10.32cm}}
\pgfpathlineto{\pgfqpoint{6.858cm}{10.32cm}}
\pgfpathlineto{\pgfqpoint{6.858cm}{10.32cm}}
\pgfpathlineto{\pgfqpoint{6.858cm}{10.32cm}}
\pgfpathlineto{\pgfqpoint{6.858cm}{10.32cm}}
\pgfpathlineto{\pgfqpoint{6.858cm}{10.32cm}}
\pgfpathlineto{\pgfqpoint{6.858cm}{10.32cm}}
\pgfpathlineto{\pgfqpoint{6.858cm}{10.32cm}}
\pgfpathlineto{\pgfqpoint{6.858cm}{10.32cm}}
\pgfpathlineto{\pgfqpoint{6.858cm}{10.32cm}}
\pgfpathlineto{\pgfqpoint{6.858cm}{10.32cm}}
\pgfpathlineto{\pgfqpoint{6.858cm}{10.32cm}}
\pgfpathlineto{\pgfqpoint{6.858cm}{10.32cm}}
\pgfpathlineto{\pgfqpoint{6.858cm}{10.32cm}}
\pgfpathlineto{\pgfqpoint{6.858cm}{10.32cm}}
\pgfpathlineto{\pgfqpoint{6.858cm}{10.32cm}}
\pgfpathlineto{\pgfqpoint{6.858cm}{10.32cm}}
\pgfpathlineto{\pgfqpoint{6.858cm}{10.32cm}}
\pgfpathlineto{\pgfqpoint{6.858cm}{10.32cm}}
\pgfpathlineto{\pgfqpoint{6.858cm}{10.32cm}}
\pgfpathlineto{\pgfqpoint{6.858cm}{10.32cm}}
\pgfpathlineto{\pgfqpoint{6.858cm}{10.32cm}}
\pgfpathlineto{\pgfqpoint{6.858cm}{10.32cm}}
\pgfpathlineto{\pgfqpoint{6.858cm}{10.32cm}}
\pgfpathlineto{\pgfqpoint{6.858cm}{10.32cm}}
\pgfpathlineto{\pgfqpoint{6.858cm}{10.32cm}}
\pgfpathlineto{\pgfqpoint{6.858cm}{10.32cm}}
\pgfpathlineto{\pgfqpoint{6.858cm}{10.32cm}}
\pgfpathlineto{\pgfqpoint{6.858cm}{10.32cm}}
\pgfpathlineto{\pgfqpoint{6.858cm}{10.32cm}}
\pgfpathlineto{\pgfqpoint{6.858cm}{10.32cm}}
\pgfpathlineto{\pgfqpoint{6.858cm}{10.32cm}}
\pgfpathlineto{\pgfqpoint{6.858cm}{10.32cm}}
\pgfpathlineto{\pgfqpoint{6.858cm}{10.32cm}}
\pgfpathlineto{\pgfqpoint{6.858cm}{10.32cm}}
\pgfpathlineto{\pgfqpoint{6.858cm}{10.32cm}}
\pgfpathlineto{\pgfqpoint{6.858cm}{10.32cm}}
\pgfpathlineto{\pgfqpoint{6.858cm}{10.32cm}}
\pgfpathlineto{\pgfqpoint{6.858cm}{10.32cm}}
\pgfpathlineto{\pgfqpoint{6.858cm}{10.32cm}}
\pgfpathlineto{\pgfqpoint{6.858cm}{10.32cm}}
\pgfpathlineto{\pgfqpoint{6.858cm}{10.32cm}}
\pgfpathlineto{\pgfqpoint{6.858cm}{10.32cm}}
\pgfpathlineto{\pgfqpoint{6.858cm}{10.32cm}}
\pgfpathlineto{\pgfqpoint{6.858cm}{10.32cm}}
\pgfpathlineto{\pgfqpoint{6.858cm}{10.32cm}}
\pgfpathlineto{\pgfqpoint{6.858cm}{10.32cm}}
\pgfpathlineto{\pgfqpoint{6.858cm}{10.32cm}}
\pgfpathlineto{\pgfqpoint{6.858cm}{10.32cm}}
\pgfpathlineto{\pgfqpoint{6.858cm}{10.32cm}}
\pgfpathlineto{\pgfqpoint{6.858cm}{10.32cm}}
\pgfpathlineto{\pgfqpoint{6.858cm}{10.32cm}}
\pgfpathlineto{\pgfqpoint{6.858cm}{10.32cm}}
\pgfpathlineto{\pgfqpoint{6.858cm}{10.32cm}}
\pgfpathlineto{\pgfqpoint{6.858cm}{10.32cm}}
\pgfpathlineto{\pgfqpoint{6.858cm}{10.32cm}}
\pgfpathlineto{\pgfqpoint{6.858cm}{10.32cm}}
\pgfpathlineto{\pgfqpoint{6.858cm}{10.32cm}}
\pgfpathlineto{\pgfqpoint{6.858cm}{10.32cm}}
\pgfpathlineto{\pgfqpoint{6.858cm}{10.32cm}}
\pgfpathlineto{\pgfqpoint{6.858cm}{10.32cm}}
\pgfpathlineto{\pgfqpoint{6.858cm}{10.32cm}}
\pgfpathlineto{\pgfqpoint{6.858cm}{10.32cm}}
\pgfpathlineto{\pgfqpoint{6.858cm}{10.32cm}}
\pgfpathlineto{\pgfqpoint{6.858cm}{10.32cm}}
\pgfpathlineto{\pgfqpoint{6.858cm}{10.32cm}}
\pgfpathlineto{\pgfqpoint{6.858cm}{10.32cm}}
\pgfpathlineto{\pgfqpoint{6.858cm}{10.32cm}}
\pgfpathlineto{\pgfqpoint{6.858cm}{10.32cm}}
\pgfpathlineto{\pgfqpoint{6.858cm}{10.32cm}}
\pgfpathlineto{\pgfqpoint{6.858cm}{10.32cm}}
\pgfpathlineto{\pgfqpoint{6.858cm}{10.32cm}}
\pgfpathlineto{\pgfqpoint{6.858cm}{10.32cm}}
\pgfpathlineto{\pgfqpoint{6.858cm}{10.32cm}}
\pgfpathlineto{\pgfqpoint{6.858cm}{10.32cm}}
\pgfpathlineto{\pgfqpoint{6.858cm}{10.32cm}}
\pgfpathlineto{\pgfqpoint{6.858cm}{10.32cm}}
\pgfpathlineto{\pgfqpoint{6.858cm}{10.32cm}}
\pgfpathlineto{\pgfqpoint{6.858cm}{10.32cm}}
\pgfpathlineto{\pgfqpoint{6.858cm}{10.32cm}}
\pgfpathlineto{\pgfqpoint{6.858cm}{10.32cm}}
\pgfpathlineto{\pgfqpoint{6.858cm}{10.32cm}}
\pgfpathlineto{\pgfqpoint{6.858cm}{10.32cm}}
\pgfpathlineto{\pgfqpoint{6.858cm}{10.32cm}}
\pgfpathlineto{\pgfqpoint{6.858cm}{10.32cm}}
\pgfusepath{stroke}
\end{pgfscope}
\begin{pgfscope}
\pgfpathmoveto{\pgfqpoint{0cm}{0cm}}
\pgfpathlineto{\pgfqpoint{12.7cm}{0cm}}
\pgfpathlineto{\pgfqpoint{12.7cm}{12.7cm}}
\pgfpathlineto{\pgfqpoint{0cm}{12.7cm}}
\pgfpathclose
\pgfusepath{clip}
\begin{pgfscope}
\definecolor{eps2pgf_color}{gray}{0}\pgfsetstrokecolor{eps2pgf_color}\pgfsetfillcolor{eps2pgf_color}
\pgftext[x=6.858cm,y=11.619cm,rotate=0]{{\larger\bf Cobweb plot of $ f(x) =  2  x  (1 - 3  x/2) $ }}
\end{pgfscope}
\end{pgfscope}
\begin{pgfscope}
\pgfpathmoveto{\pgfqpoint{2.083cm}{2.591cm}}
\pgfpathlineto{\pgfqpoint{11.633cm}{2.591cm}}
\pgfpathlineto{\pgfqpoint{11.633cm}{10.617cm}}
\pgfpathlineto{\pgfqpoint{2.083cm}{10.617cm}}
\pgfpathclose
\pgfusepath{clip}
\begin{pgfscope}
\definecolor{eps2pgf_color}{gray}{0}\pgfsetstrokecolor{eps2pgf_color}\pgfsetfillcolor{eps2pgf_color}
\pgftext[x=6.817cm,y=4.844cm,rotate=0]{Fixed points: $ \{  0, 0.333333333333333  $ \} }
\end{pgfscope}
\begin{pgfscope}
\definecolor{eps2pgf_color}{gray}{0}\pgfsetstrokecolor{eps2pgf_color}\pgfsetfillcolor{eps2pgf_color}
\pgftext[x=6.86cm,y=4.334cm,rotate=0]{Carrying capacity $ k= 0.666666666666667  $}
\end{pgfscope}
\begin{pgfscope}
\definecolor{eps2pgf_color}{gray}{0}\pgfsetstrokecolor{eps2pgf_color}\pgfsetfillcolor{eps2pgf_color}
\pgftext[x=6.869cm,y=3.631cm,rotate=0]{Rate: $ r=  2 \quad \rightarrow 1 < r < 3$}
\end{pgfscope}
\begin{pgfscope}
\definecolor{eps2pgf_color}{gray}{0}\pgfsetstrokecolor{eps2pgf_color}\pgfsetfillcolor{eps2pgf_color}
\pgftext[x=6.858cm,y=3.136cm,rotate=0]{{\bf $\therefore $ the fixed point is stable }}
\end{pgfscope}
\end{pgfscope}
\end{pgfscope}
\end{pgfpicture}

\endpgfgraphicnamed

\begin{Schunk}
\begin{Sinput}
 f <- f3
 plotCobWeb(f)
\end{Sinput}
\end{Schunk}
\beginpgfgraphicnamed{solution-functor_def13}
% Created by Eps2pgf 0.7.0 (build on 2008-08-24) on Thu Nov 05 06:11:39 CST 2009
\begin{pgfpicture}
\pgfpathmoveto{\pgfqpoint{0cm}{0cm}}
\pgfpathlineto{\pgfqpoint{12.7cm}{0cm}}
\pgfpathlineto{\pgfqpoint{12.7cm}{12.7cm}}
\pgfpathlineto{\pgfqpoint{0cm}{12.7cm}}
\pgfpathclose
\pgfusepath{clip}
\begin{pgfscope}
\begin{pgfscope}
\end{pgfscope}
\begin{pgfscope}
\pgfpathmoveto{\pgfqpoint{2.083cm}{2.591cm}}
\pgfpathlineto{\pgfqpoint{11.633cm}{2.591cm}}
\pgfpathlineto{\pgfqpoint{11.633cm}{10.617cm}}
\pgfpathlineto{\pgfqpoint{2.083cm}{10.617cm}}
\pgfpathclose
\pgfusepath{clip}
\pgfsetdash{}{0cm}
\pgfsetlinewidth{0.265mm}
\pgfsetroundcap
\pgfsetroundjoin
\definecolor{eps2pgf_color}{rgb}{1,0,0}\pgfsetstrokecolor{eps2pgf_color}\pgfsetfillcolor{eps2pgf_color}
\pgfpathmoveto{\pgfqpoint{2.437cm}{2.888cm}}
\pgfpathlineto{\pgfqpoint{2.542cm}{3.241cm}}
\pgfpathlineto{\pgfqpoint{2.649cm}{3.585cm}}
\pgfpathlineto{\pgfqpoint{2.755cm}{3.92cm}}
\pgfpathlineto{\pgfqpoint{2.861cm}{4.246cm}}
\pgfpathlineto{\pgfqpoint{2.967cm}{4.565cm}}
\pgfpathlineto{\pgfqpoint{3.073cm}{4.874cm}}
\pgfpathlineto{\pgfqpoint{3.179cm}{5.175cm}}
\pgfpathlineto{\pgfqpoint{3.285cm}{5.468cm}}
\pgfpathlineto{\pgfqpoint{3.392cm}{5.752cm}}
\pgfpathlineto{\pgfqpoint{3.498cm}{6.027cm}}
\pgfpathlineto{\pgfqpoint{3.604cm}{6.294cm}}
\pgfpathlineto{\pgfqpoint{3.71cm}{6.552cm}}
\pgfpathlineto{\pgfqpoint{3.816cm}{6.802cm}}
\pgfpathlineto{\pgfqpoint{3.922cm}{7.043cm}}
\pgfpathlineto{\pgfqpoint{4.028cm}{7.276cm}}
\pgfpathlineto{\pgfqpoint{4.134cm}{7.5cm}}
\pgfpathlineto{\pgfqpoint{4.24cm}{7.715cm}}
\pgfpathlineto{\pgfqpoint{4.347cm}{7.922cm}}
\pgfpathlineto{\pgfqpoint{4.453cm}{8.121cm}}
\pgfpathlineto{\pgfqpoint{4.559cm}{8.31cm}}
\pgfpathlineto{\pgfqpoint{4.665cm}{8.492cm}}
\pgfpathlineto{\pgfqpoint{4.771cm}{8.664cm}}
\pgfpathlineto{\pgfqpoint{4.877cm}{8.828cm}}
\pgfpathlineto{\pgfqpoint{4.983cm}{8.984cm}}
\pgfpathlineto{\pgfqpoint{5.09cm}{9.131cm}}
\pgfpathlineto{\pgfqpoint{5.195cm}{9.269cm}}
\pgfpathlineto{\pgfqpoint{5.302cm}{9.399cm}}
\pgfpathlineto{\pgfqpoint{5.408cm}{9.52cm}}
\pgfpathlineto{\pgfqpoint{5.514cm}{9.633cm}}
\pgfpathlineto{\pgfqpoint{5.62cm}{9.737cm}}
\pgfpathlineto{\pgfqpoint{5.726cm}{9.833cm}}
\pgfpathlineto{\pgfqpoint{5.832cm}{9.92cm}}
\pgfpathlineto{\pgfqpoint{5.938cm}{9.998cm}}
\pgfpathlineto{\pgfqpoint{6.044cm}{10.068cm}}
\pgfpathlineto{\pgfqpoint{6.151cm}{10.13cm}}
\pgfpathlineto{\pgfqpoint{6.257cm}{10.183cm}}
\pgfpathlineto{\pgfqpoint{6.363cm}{10.227cm}}
\pgfpathlineto{\pgfqpoint{6.469cm}{10.262cm}}
\pgfpathlineto{\pgfqpoint{6.575cm}{10.289cm}}
\pgfpathlineto{\pgfqpoint{6.681cm}{10.308cm}}
\pgfpathlineto{\pgfqpoint{6.787cm}{10.318cm}}
\pgfpathlineto{\pgfqpoint{6.893cm}{10.319cm}}
\pgfpathlineto{\pgfqpoint{6.999cm}{10.312cm}}
\pgfpathlineto{\pgfqpoint{7.106cm}{10.297cm}}
\pgfpathlineto{\pgfqpoint{7.212cm}{10.273cm}}
\pgfpathlineto{\pgfqpoint{7.318cm}{10.239cm}}
\pgfpathlineto{\pgfqpoint{7.424cm}{10.198cm}}
\pgfpathlineto{\pgfqpoint{7.53cm}{10.148cm}}
\pgfpathlineto{\pgfqpoint{7.636cm}{10.09cm}}
\pgfpathlineto{\pgfqpoint{7.742cm}{10.023cm}}
\pgfpathlineto{\pgfqpoint{7.848cm}{9.947cm}}
\pgfpathlineto{\pgfqpoint{7.954cm}{9.863cm}}
\pgfpathlineto{\pgfqpoint{8.061cm}{9.77cm}}
\pgfpathlineto{\pgfqpoint{8.167cm}{9.669cm}}
\pgfpathlineto{\pgfqpoint{8.273cm}{9.559cm}}
\pgfpathlineto{\pgfqpoint{8.379cm}{9.44cm}}
\pgfpathlineto{\pgfqpoint{8.485cm}{9.313cm}}
\pgfpathlineto{\pgfqpoint{8.591cm}{9.178cm}}
\pgfpathlineto{\pgfqpoint{8.697cm}{9.034cm}}
\pgfpathlineto{\pgfqpoint{8.804cm}{8.881cm}}
\pgfpathlineto{\pgfqpoint{8.909cm}{8.72cm}}
\pgfpathlineto{\pgfqpoint{9.016cm}{8.55cm}}
\pgfpathlineto{\pgfqpoint{9.122cm}{8.372cm}}
\pgfpathlineto{\pgfqpoint{9.228cm}{8.185cm}}
\pgfpathlineto{\pgfqpoint{9.334cm}{7.989cm}}
\pgfpathlineto{\pgfqpoint{9.44cm}{7.785cm}}
\pgfpathlineto{\pgfqpoint{9.546cm}{7.573cm}}
\pgfpathlineto{\pgfqpoint{9.652cm}{7.352cm}}
\pgfpathlineto{\pgfqpoint{9.759cm}{7.122cm}}
\pgfpathlineto{\pgfqpoint{9.865cm}{6.883cm}}
\pgfpathlineto{\pgfqpoint{9.971cm}{6.636cm}}
\pgfpathlineto{\pgfqpoint{10.077cm}{6.381cm}}
\pgfpathlineto{\pgfqpoint{10.183cm}{6.117cm}}
\pgfpathlineto{\pgfqpoint{10.289cm}{5.844cm}}
\pgfpathlineto{\pgfqpoint{10.395cm}{5.564cm}}
\pgfpathlineto{\pgfqpoint{10.501cm}{5.274cm}}
\pgfpathlineto{\pgfqpoint{10.607cm}{4.976cm}}
\pgfpathlineto{\pgfqpoint{10.714cm}{4.669cm}}
\pgfpathlineto{\pgfqpoint{10.82cm}{4.354cm}}
\pgfpathlineto{\pgfqpoint{10.926cm}{4.03cm}}
\pgfpathlineto{\pgfqpoint{11.032cm}{3.697cm}}
\pgfpathlineto{\pgfqpoint{11.138cm}{3.356cm}}
\pgfpathlineto{\pgfqpoint{11.244cm}{3.006cm}}
\pgfpathlineto{\pgfqpoint{11.35cm}{2.648cm}}
\pgfpathlineto{\pgfqpoint{11.456cm}{2.282cm}}
\pgfpathlineto{\pgfqpoint{11.562cm}{1.906cm}}
\pgfpathlineto{\pgfqpoint{11.668cm}{1.523cm}}
\pgfpathlineto{\pgfqpoint{11.775cm}{1.13cm}}
\pgfpathlineto{\pgfqpoint{11.881cm}{0.729cm}}
\pgfpathlineto{\pgfqpoint{11.987cm}{0.32cm}}
\pgfpathlineto{\pgfqpoint{12.068cm}{-0cm}}
\pgfusepath{stroke}
\end{pgfscope}
\begin{pgfscope}
\pgfpathmoveto{\pgfqpoint{0cm}{0cm}}
\pgfpathlineto{\pgfqpoint{12.7cm}{0cm}}
\pgfpathlineto{\pgfqpoint{12.7cm}{12.7cm}}
\pgfpathlineto{\pgfqpoint{0cm}{12.7cm}}
\pgfpathclose
\pgfusepath{clip}
\pgfsetdash{}{0cm}
\pgfsetlinewidth{0.265mm}
\pgfsetroundcap
\pgfsetroundjoin
\definecolor{eps2pgf_color}{gray}{0.6}\pgfsetstrokecolor{eps2pgf_color}\pgfsetfillcolor{eps2pgf_color}
\pgfpathmoveto{\pgfqpoint{2.437cm}{2.591cm}}
\pgfpathlineto{\pgfqpoint{11.279cm}{2.591cm}}
\pgfusepath{stroke}
\pgfsetdash{}{0cm}
\pgfpathmoveto{\pgfqpoint{2.437cm}{2.591cm}}
\pgfpathlineto{\pgfqpoint{2.437cm}{2.337cm}}
\pgfusepath{stroke}
\pgfsetdash{}{0cm}
\pgfpathmoveto{\pgfqpoint{4.205cm}{2.591cm}}
\pgfpathlineto{\pgfqpoint{4.205cm}{2.337cm}}
\pgfusepath{stroke}
\pgfsetdash{}{0cm}
\pgfpathmoveto{\pgfqpoint{5.974cm}{2.591cm}}
\pgfpathlineto{\pgfqpoint{5.974cm}{2.337cm}}
\pgfusepath{stroke}
\pgfsetdash{}{0cm}
\pgfpathmoveto{\pgfqpoint{7.742cm}{2.591cm}}
\pgfpathlineto{\pgfqpoint{7.742cm}{2.337cm}}
\pgfusepath{stroke}
\pgfsetdash{}{0cm}
\pgfpathmoveto{\pgfqpoint{9.511cm}{2.591cm}}
\pgfpathlineto{\pgfqpoint{9.511cm}{2.337cm}}
\pgfusepath{stroke}
\pgfsetdash{}{0cm}
\pgfpathmoveto{\pgfqpoint{11.279cm}{2.591cm}}
\pgfpathlineto{\pgfqpoint{11.279cm}{2.337cm}}
\pgfusepath{stroke}
\begin{pgfscope}
\definecolor{eps2pgf_color}{gray}{0}\pgfsetstrokecolor{eps2pgf_color}\pgfsetfillcolor{eps2pgf_color}
\pgftext[x=2.437cm,y=1.821cm,rotate=0]{0.0}
\end{pgfscope}
\begin{pgfscope}
\definecolor{eps2pgf_color}{gray}{0}\pgfsetstrokecolor{eps2pgf_color}\pgfsetfillcolor{eps2pgf_color}
\pgftext[x=4.204cm,y=1.821cm,rotate=0]{0.5}
\end{pgfscope}
\begin{pgfscope}
\definecolor{eps2pgf_color}{gray}{0}\pgfsetstrokecolor{eps2pgf_color}\pgfsetfillcolor{eps2pgf_color}
\pgftext[x=5.987cm,y=1.821cm,rotate=0]{1.0}
\end{pgfscope}
\begin{pgfscope}
\definecolor{eps2pgf_color}{gray}{0}\pgfsetstrokecolor{eps2pgf_color}\pgfsetfillcolor{eps2pgf_color}
\pgftext[x=7.755cm,y=1.821cm,rotate=0]{1.5}
\end{pgfscope}
\begin{pgfscope}
\definecolor{eps2pgf_color}{gray}{0}\pgfsetstrokecolor{eps2pgf_color}\pgfsetfillcolor{eps2pgf_color}
\pgftext[x=9.509cm,y=1.821cm,rotate=0]{2.0}
\end{pgfscope}
\begin{pgfscope}
\definecolor{eps2pgf_color}{gray}{0}\pgfsetstrokecolor{eps2pgf_color}\pgfsetfillcolor{eps2pgf_color}
\pgftext[x=11.276cm,y=1.821cm,rotate=0]{2.5}
\end{pgfscope}
\pgfsetdash{}{0cm}
\pgfpathmoveto{\pgfqpoint{2.083cm}{2.888cm}}
\pgfpathlineto{\pgfqpoint{2.083cm}{10.023cm}}
\pgfusepath{stroke}
\pgfsetdash{}{0cm}
\pgfpathmoveto{\pgfqpoint{2.083cm}{2.888cm}}
\pgfpathlineto{\pgfqpoint{1.829cm}{2.888cm}}
\pgfusepath{stroke}
\pgfsetdash{}{0cm}
\pgfpathmoveto{\pgfqpoint{2.083cm}{4.077cm}}
\pgfpathlineto{\pgfqpoint{1.829cm}{4.077cm}}
\pgfusepath{stroke}
\pgfsetdash{}{0cm}
\pgfpathmoveto{\pgfqpoint{2.083cm}{5.266cm}}
\pgfpathlineto{\pgfqpoint{1.829cm}{5.266cm}}
\pgfusepath{stroke}
\pgfsetdash{}{0cm}
\pgfpathmoveto{\pgfqpoint{2.083cm}{6.455cm}}
\pgfpathlineto{\pgfqpoint{1.829cm}{6.455cm}}
\pgfusepath{stroke}
\pgfsetdash{}{0cm}
\pgfpathmoveto{\pgfqpoint{2.083cm}{7.644cm}}
\pgfpathlineto{\pgfqpoint{1.829cm}{7.644cm}}
\pgfusepath{stroke}
\pgfsetdash{}{0cm}
\pgfpathmoveto{\pgfqpoint{2.083cm}{8.834cm}}
\pgfpathlineto{\pgfqpoint{1.829cm}{8.834cm}}
\pgfusepath{stroke}
\pgfsetdash{}{0cm}
\pgfpathmoveto{\pgfqpoint{2.083cm}{10.023cm}}
\pgfpathlineto{\pgfqpoint{1.829cm}{10.023cm}}
\pgfusepath{stroke}
\begin{pgfscope}
\definecolor{eps2pgf_color}{gray}{0}\pgfsetstrokecolor{eps2pgf_color}\pgfsetfillcolor{eps2pgf_color}
\pgftext[x=1.328cm,y=2.888cm,rotate=90]{0.00}
\end{pgfscope}
\begin{pgfscope}
\definecolor{eps2pgf_color}{gray}{0}\pgfsetstrokecolor{eps2pgf_color}\pgfsetfillcolor{eps2pgf_color}
\pgftext[x=1.328cm,y=4.076cm,rotate=90]{0.05}
\end{pgfscope}
\begin{pgfscope}
\definecolor{eps2pgf_color}{gray}{0}\pgfsetstrokecolor{eps2pgf_color}\pgfsetfillcolor{eps2pgf_color}
\pgftext[x=1.328cm,y=5.266cm,rotate=90]{0.10}
\end{pgfscope}
\begin{pgfscope}
\definecolor{eps2pgf_color}{gray}{0}\pgfsetstrokecolor{eps2pgf_color}\pgfsetfillcolor{eps2pgf_color}
\pgftext[x=1.328cm,y=6.454cm,rotate=90]{0.15}
\end{pgfscope}
\begin{pgfscope}
\definecolor{eps2pgf_color}{gray}{0}\pgfsetstrokecolor{eps2pgf_color}\pgfsetfillcolor{eps2pgf_color}
\pgftext[x=1.328cm,y=7.644cm,rotate=90]{0.20}
\end{pgfscope}
\begin{pgfscope}
\definecolor{eps2pgf_color}{gray}{0}\pgfsetstrokecolor{eps2pgf_color}\pgfsetfillcolor{eps2pgf_color}
\pgftext[x=1.328cm,y=8.832cm,rotate=90]{0.25}
\end{pgfscope}
\begin{pgfscope}
\definecolor{eps2pgf_color}{gray}{0}\pgfsetstrokecolor{eps2pgf_color}\pgfsetfillcolor{eps2pgf_color}
\pgftext[x=1.328cm,y=10.023cm,rotate=90]{0.30}
\end{pgfscope}
\end{pgfscope}
\begin{pgfscope}
\pgfpathmoveto{\pgfqpoint{0cm}{0cm}}
\pgfpathlineto{\pgfqpoint{12.7cm}{0cm}}
\pgfpathlineto{\pgfqpoint{12.7cm}{12.7cm}}
\pgfpathlineto{\pgfqpoint{0cm}{12.7cm}}
\pgfpathclose
\pgfusepath{clip}
\begin{pgfscope}
\definecolor{eps2pgf_color}{gray}{0}\pgfsetstrokecolor{eps2pgf_color}\pgfsetfillcolor{eps2pgf_color}
\pgftext[x=6.857cm,y=0.8cm,rotate=0]{$t$}
\end{pgfscope}
\begin{pgfscope}
\definecolor{eps2pgf_color}{gray}{0}\pgfsetstrokecolor{eps2pgf_color}\pgfsetfillcolor{eps2pgf_color}
\pgftext[x=0.337cm,y=6.603cm,rotate=90]{$f(t)$}
\end{pgfscope}
\end{pgfscope}
\begin{pgfscope}
\pgfpathmoveto{\pgfqpoint{2.083cm}{2.591cm}}
\pgfpathlineto{\pgfqpoint{11.633cm}{2.591cm}}
\pgfpathlineto{\pgfqpoint{11.633cm}{10.617cm}}
\pgfpathlineto{\pgfqpoint{2.083cm}{10.617cm}}
\pgfpathclose
\pgfusepath{clip}
\pgfsetdash{}{0cm}
\pgfsetlinewidth{0.395mm}
\pgfsetroundcap
\pgfsetroundjoin
\definecolor{eps2pgf_color}{gray}{0}\pgfsetstrokecolor{eps2pgf_color}\pgfsetfillcolor{eps2pgf_color}
\pgfpathmoveto{\pgfqpoint{2.083cm}{2.888cm}}
\pgfpathlineto{\pgfqpoint{11.633cm}{2.888cm}}
\pgfusepath{stroke}
\pgfsetdash{}{0cm}
\pgfpathmoveto{\pgfqpoint{2.437cm}{2.591cm}}
\pgfpathlineto{\pgfqpoint{2.437cm}{10.617cm}}
\pgfusepath{stroke}
\pgfsetdash{}{0cm}
\pgfsetlinewidth{0.265mm}
\definecolor{eps2pgf_color}{rgb}{0,0,1}\pgfsetstrokecolor{eps2pgf_color}\pgfsetfillcolor{eps2pgf_color}
\pgfpathmoveto{\pgfqpoint{2.083cm}{0.51cm}}
\pgfpathlineto{\pgfqpoint{3.896cm}{12.7cm}}
\pgfusepath{stroke}
\pgfsetdash{}{0cm}
\definecolor{eps2pgf_color}{rgb}{0,0.3922,0}\pgfsetstrokecolor{eps2pgf_color}\pgfsetfillcolor{eps2pgf_color}
\pgfpathmoveto{\pgfqpoint{4.647cm}{2.888cm}}
\pgfpathlineto{\pgfqpoint{4.647cm}{8.462cm}}
\pgfpathlineto{\pgfqpoint{3.266cm}{8.462cm}}
\pgfpathlineto{\pgfqpoint{3.266cm}{5.414cm}}
\pgfpathlineto{\pgfqpoint{2.812cm}{5.414cm}}
\pgfpathlineto{\pgfqpoint{2.812cm}{4.097cm}}
\pgfpathlineto{\pgfqpoint{2.616cm}{4.097cm}}
\pgfpathlineto{\pgfqpoint{2.616cm}{3.481cm}}
\pgfpathlineto{\pgfqpoint{2.524cm}{3.481cm}}
\pgfpathlineto{\pgfqpoint{2.524cm}{3.181cm}}
\pgfpathlineto{\pgfqpoint{2.48cm}{3.181cm}}
\pgfpathlineto{\pgfqpoint{2.48cm}{3.034cm}}
\pgfpathlineto{\pgfqpoint{2.458cm}{3.034cm}}
\pgfpathlineto{\pgfqpoint{2.458cm}{2.961cm}}
\pgfpathlineto{\pgfqpoint{2.447cm}{2.961cm}}
\pgfpathlineto{\pgfqpoint{2.447cm}{2.925cm}}
\pgfpathlineto{\pgfqpoint{2.442cm}{2.925cm}}
\pgfpathlineto{\pgfqpoint{2.442cm}{2.906cm}}
\pgfpathlineto{\pgfqpoint{2.439cm}{2.906cm}}
\pgfpathlineto{\pgfqpoint{2.439cm}{2.897cm}}
\pgfpathlineto{\pgfqpoint{2.438cm}{2.897cm}}
\pgfpathlineto{\pgfqpoint{2.438cm}{2.893cm}}
\pgfpathlineto{\pgfqpoint{2.437cm}{2.893cm}}
\pgfpathlineto{\pgfqpoint{2.437cm}{2.89cm}}
\pgfpathlineto{\pgfqpoint{2.437cm}{2.89cm}}
\pgfpathlineto{\pgfqpoint{2.437cm}{2.889cm}}
\pgfpathlineto{\pgfqpoint{2.437cm}{2.889cm}}
\pgfpathlineto{\pgfqpoint{2.437cm}{2.889cm}}
\pgfpathlineto{\pgfqpoint{2.437cm}{2.889cm}}
\pgfpathlineto{\pgfqpoint{2.437cm}{2.888cm}}
\pgfpathlineto{\pgfqpoint{2.437cm}{2.888cm}}
\pgfpathlineto{\pgfqpoint{2.437cm}{2.888cm}}
\pgfpathlineto{\pgfqpoint{2.437cm}{2.888cm}}
\pgfpathlineto{\pgfqpoint{2.437cm}{2.888cm}}
\pgfpathlineto{\pgfqpoint{2.437cm}{2.888cm}}
\pgfpathlineto{\pgfqpoint{2.437cm}{2.888cm}}
\pgfpathlineto{\pgfqpoint{2.437cm}{2.888cm}}
\pgfpathlineto{\pgfqpoint{2.437cm}{2.888cm}}
\pgfpathlineto{\pgfqpoint{2.437cm}{2.888cm}}
\pgfpathlineto{\pgfqpoint{2.437cm}{2.888cm}}
\pgfpathlineto{\pgfqpoint{2.437cm}{2.888cm}}
\pgfpathlineto{\pgfqpoint{2.437cm}{2.888cm}}
\pgfpathlineto{\pgfqpoint{2.437cm}{2.888cm}}
\pgfpathlineto{\pgfqpoint{2.437cm}{2.888cm}}
\pgfpathlineto{\pgfqpoint{2.437cm}{2.888cm}}
\pgfpathlineto{\pgfqpoint{2.437cm}{2.888cm}}
\pgfpathlineto{\pgfqpoint{2.437cm}{2.888cm}}
\pgfpathlineto{\pgfqpoint{2.437cm}{2.888cm}}
\pgfpathlineto{\pgfqpoint{2.437cm}{2.888cm}}
\pgfpathlineto{\pgfqpoint{2.437cm}{2.888cm}}
\pgfpathlineto{\pgfqpoint{2.437cm}{2.888cm}}
\pgfpathlineto{\pgfqpoint{2.437cm}{2.888cm}}
\pgfpathlineto{\pgfqpoint{2.437cm}{2.888cm}}
\pgfpathlineto{\pgfqpoint{2.437cm}{2.888cm}}
\pgfpathlineto{\pgfqpoint{2.437cm}{2.888cm}}
\pgfpathlineto{\pgfqpoint{2.437cm}{2.888cm}}
\pgfpathlineto{\pgfqpoint{2.437cm}{2.888cm}}
\pgfpathlineto{\pgfqpoint{2.437cm}{2.888cm}}
\pgfpathlineto{\pgfqpoint{2.437cm}{2.888cm}}
\pgfpathlineto{\pgfqpoint{2.437cm}{2.888cm}}
\pgfpathlineto{\pgfqpoint{2.437cm}{2.888cm}}
\pgfpathlineto{\pgfqpoint{2.437cm}{2.888cm}}
\pgfpathlineto{\pgfqpoint{2.437cm}{2.888cm}}
\pgfpathlineto{\pgfqpoint{2.437cm}{2.888cm}}
\pgfpathlineto{\pgfqpoint{2.437cm}{2.888cm}}
\pgfpathlineto{\pgfqpoint{2.437cm}{2.888cm}}
\pgfpathlineto{\pgfqpoint{2.437cm}{2.888cm}}
\pgfpathlineto{\pgfqpoint{2.437cm}{2.888cm}}
\pgfpathlineto{\pgfqpoint{2.437cm}{2.888cm}}
\pgfpathlineto{\pgfqpoint{2.437cm}{2.888cm}}
\pgfpathlineto{\pgfqpoint{2.437cm}{2.888cm}}
\pgfpathlineto{\pgfqpoint{2.437cm}{2.888cm}}
\pgfpathlineto{\pgfqpoint{2.437cm}{2.888cm}}
\pgfpathlineto{\pgfqpoint{2.437cm}{2.888cm}}
\pgfpathlineto{\pgfqpoint{2.437cm}{2.888cm}}
\pgfpathlineto{\pgfqpoint{2.437cm}{2.888cm}}
\pgfpathlineto{\pgfqpoint{2.437cm}{2.888cm}}
\pgfpathlineto{\pgfqpoint{2.437cm}{2.888cm}}
\pgfpathlineto{\pgfqpoint{2.437cm}{2.888cm}}
\pgfpathlineto{\pgfqpoint{2.437cm}{2.888cm}}
\pgfpathlineto{\pgfqpoint{2.437cm}{2.888cm}}
\pgfpathlineto{\pgfqpoint{2.437cm}{2.888cm}}
\pgfpathlineto{\pgfqpoint{2.437cm}{2.888cm}}
\pgfpathlineto{\pgfqpoint{2.437cm}{2.888cm}}
\pgfpathlineto{\pgfqpoint{2.437cm}{2.888cm}}
\pgfpathlineto{\pgfqpoint{2.437cm}{2.888cm}}
\pgfpathlineto{\pgfqpoint{2.437cm}{2.888cm}}
\pgfpathlineto{\pgfqpoint{2.437cm}{2.888cm}}
\pgfpathlineto{\pgfqpoint{2.437cm}{2.888cm}}
\pgfpathlineto{\pgfqpoint{2.437cm}{2.888cm}}
\pgfpathlineto{\pgfqpoint{2.437cm}{2.888cm}}
\pgfpathlineto{\pgfqpoint{2.437cm}{2.888cm}}
\pgfpathlineto{\pgfqpoint{2.437cm}{2.888cm}}
\pgfpathlineto{\pgfqpoint{2.437cm}{2.888cm}}
\pgfpathlineto{\pgfqpoint{2.437cm}{2.888cm}}
\pgfpathlineto{\pgfqpoint{2.437cm}{2.888cm}}
\pgfpathlineto{\pgfqpoint{2.437cm}{2.888cm}}
\pgfpathlineto{\pgfqpoint{2.437cm}{2.888cm}}
\pgfpathlineto{\pgfqpoint{2.437cm}{2.888cm}}
\pgfpathlineto{\pgfqpoint{2.437cm}{2.888cm}}
\pgfpathlineto{\pgfqpoint{2.437cm}{2.888cm}}
\pgfusepath{stroke}
\end{pgfscope}
\begin{pgfscope}
\pgfpathmoveto{\pgfqpoint{0cm}{0cm}}
\pgfpathlineto{\pgfqpoint{12.7cm}{0cm}}
\pgfpathlineto{\pgfqpoint{12.7cm}{12.7cm}}
\pgfpathlineto{\pgfqpoint{0cm}{12.7cm}}
\pgfpathclose
\pgfusepath{clip}
\begin{pgfscope}
\definecolor{eps2pgf_color}{gray}{0}\pgfsetstrokecolor{eps2pgf_color}\pgfsetfillcolor{eps2pgf_color}
\pgftext[x=6.858cm,y=11.619cm,rotate=0]{{\larger\bf Cobweb plot of $ f(x) =  x/2 - x^2/5 $ }}
\end{pgfscope}
\end{pgfscope}
\begin{pgfscope}
\pgfpathmoveto{\pgfqpoint{2.083cm}{2.591cm}}
\pgfpathlineto{\pgfqpoint{11.633cm}{2.591cm}}
\pgfpathlineto{\pgfqpoint{11.633cm}{10.617cm}}
\pgfpathlineto{\pgfqpoint{2.083cm}{10.617cm}}
\pgfpathclose
\pgfusepath{clip}
\begin{pgfscope}
\definecolor{eps2pgf_color}{gray}{0}\pgfsetstrokecolor{eps2pgf_color}\pgfsetfillcolor{eps2pgf_color}
\pgftext[x=6.817cm,y=4.844cm,rotate=0]{Fixed points: $ \{  -2.5, 0  $ \} }
\end{pgfscope}
\begin{pgfscope}
\definecolor{eps2pgf_color}{gray}{0}\pgfsetstrokecolor{eps2pgf_color}\pgfsetfillcolor{eps2pgf_color}
\pgftext[x=6.86cm,y=4.334cm,rotate=0]{Carrying capacity $ k= 2.5  $}
\end{pgfscope}
\begin{pgfscope}
\definecolor{eps2pgf_color}{gray}{0}\pgfsetstrokecolor{eps2pgf_color}\pgfsetfillcolor{eps2pgf_color}
\pgftext[x=6.869cm,y=3.631cm,rotate=0]{Rate: $r =  0.5 \quad \rightarrow r > 3$}
\end{pgfscope}
\begin{pgfscope}
\definecolor{eps2pgf_color}{gray}{0}\pgfsetstrokecolor{eps2pgf_color}\pgfsetfillcolor{eps2pgf_color}
\pgftext[x=6.858cm,y=3.136cm,rotate=0]{{\bf $\therefore $ the fixed point is unstable }}
\end{pgfscope}
\end{pgfscope}
\end{pgfscope}
\end{pgfpicture}

\endpgfgraphicnamed
\begin{Schunk}
\begin{Sinput}
 f <- f4
 plotCobWeb(f)
\end{Sinput}
\end{Schunk}
\beginpgfgraphicnamed{solution-functor_def14}
% Created by Eps2pgf 0.7.0 (build on 2008-08-24) on Thu Nov 05 06:11:42 CST 2009
\begin{pgfpicture}
\pgfpathmoveto{\pgfqpoint{0cm}{0cm}}
\pgfpathlineto{\pgfqpoint{12.7cm}{0cm}}
\pgfpathlineto{\pgfqpoint{12.7cm}{12.7cm}}
\pgfpathlineto{\pgfqpoint{0cm}{12.7cm}}
\pgfpathclose
\pgfusepath{clip}
\begin{pgfscope}
\begin{pgfscope}
\end{pgfscope}
\begin{pgfscope}
\pgfpathmoveto{\pgfqpoint{2.083cm}{2.591cm}}
\pgfpathlineto{\pgfqpoint{11.633cm}{2.591cm}}
\pgfpathlineto{\pgfqpoint{11.633cm}{10.617cm}}
\pgfpathlineto{\pgfqpoint{2.083cm}{10.617cm}}
\pgfpathclose
\pgfusepath{clip}
\pgfsetdash{}{0cm}
\pgfsetlinewidth{0.265mm}
\pgfsetroundcap
\pgfsetroundjoin
\definecolor{eps2pgf_color}{rgb}{1,0,0}\pgfsetstrokecolor{eps2pgf_color}\pgfsetfillcolor{eps2pgf_color}
\pgfpathmoveto{\pgfqpoint{2.437cm}{2.888cm}}
\pgfpathlineto{\pgfqpoint{2.684cm}{3.697cm}}
\pgfpathlineto{\pgfqpoint{2.932cm}{4.459cm}}
\pgfpathlineto{\pgfqpoint{3.179cm}{5.175cm}}
\pgfpathlineto{\pgfqpoint{3.427cm}{5.844cm}}
\pgfpathlineto{\pgfqpoint{3.675cm}{6.467cm}}
\pgfpathlineto{\pgfqpoint{3.922cm}{7.043cm}}
\pgfpathlineto{\pgfqpoint{4.17cm}{7.573cm}}
\pgfpathlineto{\pgfqpoint{4.417cm}{8.055cm}}
\pgfpathlineto{\pgfqpoint{4.665cm}{8.492cm}}
\pgfpathlineto{\pgfqpoint{4.912cm}{8.881cm}}
\pgfpathlineto{\pgfqpoint{5.16cm}{9.224cm}}
\pgfpathlineto{\pgfqpoint{5.408cm}{9.52cm}}
\pgfpathlineto{\pgfqpoint{5.655cm}{9.77cm}}
\pgfpathlineto{\pgfqpoint{5.903cm}{9.973cm}}
\pgfpathlineto{\pgfqpoint{6.151cm}{10.13cm}}
\pgfpathlineto{\pgfqpoint{6.398cm}{10.239cm}}
\pgfpathlineto{\pgfqpoint{6.646cm}{10.303cm}}
\pgfpathlineto{\pgfqpoint{6.893cm}{10.319cm}}
\pgfpathlineto{\pgfqpoint{7.141cm}{10.289cm}}
\pgfpathlineto{\pgfqpoint{7.389cm}{10.213cm}}
\pgfpathlineto{\pgfqpoint{7.636cm}{10.09cm}}
\pgfpathlineto{\pgfqpoint{7.884cm}{9.92cm}}
\pgfpathlineto{\pgfqpoint{8.132cm}{9.704cm}}
\pgfpathlineto{\pgfqpoint{8.379cm}{9.44cm}}
\pgfpathlineto{\pgfqpoint{8.626cm}{9.131cm}}
\pgfpathlineto{\pgfqpoint{8.874cm}{8.775cm}}
\pgfpathlineto{\pgfqpoint{9.122cm}{8.372cm}}
\pgfpathlineto{\pgfqpoint{9.369cm}{7.922cm}}
\pgfpathlineto{\pgfqpoint{9.617cm}{7.426cm}}
\pgfpathlineto{\pgfqpoint{9.865cm}{6.883cm}}
\pgfpathlineto{\pgfqpoint{10.112cm}{6.294cm}}
\pgfpathlineto{\pgfqpoint{10.36cm}{5.658cm}}
\pgfpathlineto{\pgfqpoint{10.607cm}{4.976cm}}
\pgfpathlineto{\pgfqpoint{10.855cm}{4.246cm}}
\pgfpathlineto{\pgfqpoint{11.103cm}{3.471cm}}
\pgfpathlineto{\pgfqpoint{11.35cm}{2.648cm}}
\pgfpathlineto{\pgfqpoint{11.598cm}{1.779cm}}
\pgfpathlineto{\pgfqpoint{11.846cm}{0.864cm}}
\pgfpathlineto{\pgfqpoint{12.068cm}{0cm}}
\pgfusepath{stroke}
\end{pgfscope}
\begin{pgfscope}
\pgfpathmoveto{\pgfqpoint{0cm}{0cm}}
\pgfpathlineto{\pgfqpoint{12.7cm}{0cm}}
\pgfpathlineto{\pgfqpoint{12.7cm}{12.7cm}}
\pgfpathlineto{\pgfqpoint{0cm}{12.7cm}}
\pgfpathclose
\pgfusepath{clip}
\pgfsetdash{}{0cm}
\pgfsetlinewidth{0.265mm}
\pgfsetroundcap
\pgfsetroundjoin
\definecolor{eps2pgf_color}{gray}{0.6}\pgfsetstrokecolor{eps2pgf_color}\pgfsetfillcolor{eps2pgf_color}
\pgfpathmoveto{\pgfqpoint{2.437cm}{2.591cm}}
\pgfpathlineto{\pgfqpoint{11.103cm}{2.591cm}}
\pgfusepath{stroke}
\pgfsetdash{}{0cm}
\pgfpathmoveto{\pgfqpoint{2.437cm}{2.591cm}}
\pgfpathlineto{\pgfqpoint{2.437cm}{2.337cm}}
\pgfusepath{stroke}
\pgfsetdash{}{0cm}
\pgfpathmoveto{\pgfqpoint{3.675cm}{2.591cm}}
\pgfpathlineto{\pgfqpoint{3.675cm}{2.337cm}}
\pgfusepath{stroke}
\pgfsetdash{}{0cm}
\pgfpathmoveto{\pgfqpoint{4.912cm}{2.591cm}}
\pgfpathlineto{\pgfqpoint{4.912cm}{2.337cm}}
\pgfusepath{stroke}
\pgfsetdash{}{0cm}
\pgfpathmoveto{\pgfqpoint{6.151cm}{2.591cm}}
\pgfpathlineto{\pgfqpoint{6.151cm}{2.337cm}}
\pgfusepath{stroke}
\pgfsetdash{}{0cm}
\pgfpathmoveto{\pgfqpoint{7.389cm}{2.591cm}}
\pgfpathlineto{\pgfqpoint{7.389cm}{2.337cm}}
\pgfusepath{stroke}
\pgfsetdash{}{0cm}
\pgfpathmoveto{\pgfqpoint{8.626cm}{2.591cm}}
\pgfpathlineto{\pgfqpoint{8.626cm}{2.337cm}}
\pgfusepath{stroke}
\pgfsetdash{}{0cm}
\pgfpathmoveto{\pgfqpoint{9.865cm}{2.591cm}}
\pgfpathlineto{\pgfqpoint{9.865cm}{2.337cm}}
\pgfusepath{stroke}
\pgfsetdash{}{0cm}
\pgfpathmoveto{\pgfqpoint{11.103cm}{2.591cm}}
\pgfpathlineto{\pgfqpoint{11.103cm}{2.337cm}}
\pgfusepath{stroke}
\begin{pgfscope}
\definecolor{eps2pgf_color}{gray}{0}\pgfsetstrokecolor{eps2pgf_color}\pgfsetfillcolor{eps2pgf_color}
\pgftext[x=2.437cm,y=1.821cm,rotate=0]{0.00}
\end{pgfscope}
\begin{pgfscope}
\definecolor{eps2pgf_color}{gray}{0}\pgfsetstrokecolor{eps2pgf_color}\pgfsetfillcolor{eps2pgf_color}
\pgftext[x=3.673cm,y=1.821cm,rotate=0]{0.05}
\end{pgfscope}
\begin{pgfscope}
\definecolor{eps2pgf_color}{gray}{0}\pgfsetstrokecolor{eps2pgf_color}\pgfsetfillcolor{eps2pgf_color}
\pgftext[x=4.912cm,y=1.821cm,rotate=0]{0.10}
\end{pgfscope}
\begin{pgfscope}
\definecolor{eps2pgf_color}{gray}{0}\pgfsetstrokecolor{eps2pgf_color}\pgfsetfillcolor{eps2pgf_color}
\pgftext[x=6.15cm,y=1.821cm,rotate=0]{0.15}
\end{pgfscope}
\begin{pgfscope}
\definecolor{eps2pgf_color}{gray}{0}\pgfsetstrokecolor{eps2pgf_color}\pgfsetfillcolor{eps2pgf_color}
\pgftext[x=7.389cm,y=1.821cm,rotate=0]{0.20}
\end{pgfscope}
\begin{pgfscope}
\definecolor{eps2pgf_color}{gray}{0}\pgfsetstrokecolor{eps2pgf_color}\pgfsetfillcolor{eps2pgf_color}
\pgftext[x=8.625cm,y=1.821cm,rotate=0]{0.25}
\end{pgfscope}
\begin{pgfscope}
\definecolor{eps2pgf_color}{gray}{0}\pgfsetstrokecolor{eps2pgf_color}\pgfsetfillcolor{eps2pgf_color}
\pgftext[x=9.865cm,y=1.821cm,rotate=0]{0.30}
\end{pgfscope}
\begin{pgfscope}
\definecolor{eps2pgf_color}{gray}{0}\pgfsetstrokecolor{eps2pgf_color}\pgfsetfillcolor{eps2pgf_color}
\pgftext[x=11.102cm,y=1.821cm,rotate=0]{0.35}
\end{pgfscope}
\pgfsetdash{}{0cm}
\pgfpathmoveto{\pgfqpoint{2.083cm}{2.888cm}}
\pgfpathlineto{\pgfqpoint{2.083cm}{9.547cm}}
\pgfusepath{stroke}
\pgfsetdash{}{0cm}
\pgfpathmoveto{\pgfqpoint{2.083cm}{2.888cm}}
\pgfpathlineto{\pgfqpoint{1.829cm}{2.888cm}}
\pgfusepath{stroke}
\pgfsetdash{}{0cm}
\pgfpathmoveto{\pgfqpoint{2.083cm}{4.553cm}}
\pgfpathlineto{\pgfqpoint{1.829cm}{4.553cm}}
\pgfusepath{stroke}
\pgfsetdash{}{0cm}
\pgfpathmoveto{\pgfqpoint{2.083cm}{6.218cm}}
\pgfpathlineto{\pgfqpoint{1.829cm}{6.218cm}}
\pgfusepath{stroke}
\pgfsetdash{}{0cm}
\pgfpathmoveto{\pgfqpoint{2.083cm}{7.882cm}}
\pgfpathlineto{\pgfqpoint{1.829cm}{7.882cm}}
\pgfusepath{stroke}
\pgfsetdash{}{0cm}
\pgfpathmoveto{\pgfqpoint{2.083cm}{9.547cm}}
\pgfpathlineto{\pgfqpoint{1.829cm}{9.547cm}}
\pgfusepath{stroke}
\begin{pgfscope}
\definecolor{eps2pgf_color}{gray}{0}\pgfsetstrokecolor{eps2pgf_color}\pgfsetfillcolor{eps2pgf_color}
\pgftext[x=1.328cm,y=2.888cm,rotate=90]{0.00}
\end{pgfscope}
\begin{pgfscope}
\definecolor{eps2pgf_color}{gray}{0}\pgfsetstrokecolor{eps2pgf_color}\pgfsetfillcolor{eps2pgf_color}
\pgftext[x=1.328cm,y=4.552cm,rotate=90]{0.05}
\end{pgfscope}
\begin{pgfscope}
\definecolor{eps2pgf_color}{gray}{0}\pgfsetstrokecolor{eps2pgf_color}\pgfsetfillcolor{eps2pgf_color}
\pgftext[x=1.328cm,y=6.218cm,rotate=90]{0.10}
\end{pgfscope}
\begin{pgfscope}
\definecolor{eps2pgf_color}{gray}{0}\pgfsetstrokecolor{eps2pgf_color}\pgfsetfillcolor{eps2pgf_color}
\pgftext[x=1.328cm,y=7.881cm,rotate=90]{0.15}
\end{pgfscope}
\begin{pgfscope}
\definecolor{eps2pgf_color}{gray}{0}\pgfsetstrokecolor{eps2pgf_color}\pgfsetfillcolor{eps2pgf_color}
\pgftext[x=1.328cm,y=9.547cm,rotate=90]{0.20}
\end{pgfscope}
\end{pgfscope}
\begin{pgfscope}
\pgfpathmoveto{\pgfqpoint{0cm}{0cm}}
\pgfpathlineto{\pgfqpoint{12.7cm}{0cm}}
\pgfpathlineto{\pgfqpoint{12.7cm}{12.7cm}}
\pgfpathlineto{\pgfqpoint{0cm}{12.7cm}}
\pgfpathclose
\pgfusepath{clip}
\begin{pgfscope}
\definecolor{eps2pgf_color}{gray}{0}\pgfsetstrokecolor{eps2pgf_color}\pgfsetfillcolor{eps2pgf_color}
\pgftext[x=6.857cm,y=0.8cm,rotate=0]{$t$}
\end{pgfscope}
\begin{pgfscope}
\definecolor{eps2pgf_color}{gray}{0}\pgfsetstrokecolor{eps2pgf_color}\pgfsetfillcolor{eps2pgf_color}
\pgftext[x=0.337cm,y=6.603cm,rotate=90]{$f(t)$}
\end{pgfscope}
\end{pgfscope}
\begin{pgfscope}
\pgfpathmoveto{\pgfqpoint{2.083cm}{2.591cm}}
\pgfpathlineto{\pgfqpoint{11.633cm}{2.591cm}}
\pgfpathlineto{\pgfqpoint{11.633cm}{10.617cm}}
\pgfpathlineto{\pgfqpoint{2.083cm}{10.617cm}}
\pgfpathclose
\pgfusepath{clip}
\pgfsetdash{}{0cm}
\pgfsetlinewidth{0.395mm}
\pgfsetroundcap
\pgfsetroundjoin
\definecolor{eps2pgf_color}{gray}{0}\pgfsetstrokecolor{eps2pgf_color}\pgfsetfillcolor{eps2pgf_color}
\pgfpathmoveto{\pgfqpoint{2.083cm}{2.888cm}}
\pgfpathlineto{\pgfqpoint{11.633cm}{2.888cm}}
\pgfusepath{stroke}
\pgfsetdash{}{0cm}
\pgfpathmoveto{\pgfqpoint{2.437cm}{2.591cm}}
\pgfpathlineto{\pgfqpoint{2.437cm}{10.617cm}}
\pgfusepath{stroke}
\pgfsetdash{}{0cm}
\pgfsetlinewidth{0.265mm}
\definecolor{eps2pgf_color}{rgb}{0,0,1}\pgfsetstrokecolor{eps2pgf_color}\pgfsetfillcolor{eps2pgf_color}
\pgfpathmoveto{\pgfqpoint{2.083cm}{2.412cm}}
\pgfpathlineto{\pgfqpoint{9.733cm}{12.7cm}}
\pgfusepath{stroke}
\pgfsetdash{}{0cm}
\definecolor{eps2pgf_color}{rgb}{0,0.3922,0}\pgfsetstrokecolor{eps2pgf_color}\pgfsetfillcolor{eps2pgf_color}
\pgfpathmoveto{\pgfqpoint{4.647cm}{2.888cm}}
\pgfpathlineto{\pgfqpoint{4.647cm}{8.462cm}}
\pgfpathlineto{\pgfqpoint{6.582cm}{8.462cm}}
\pgfpathlineto{\pgfqpoint{6.582cm}{10.291cm}}
\pgfpathlineto{\pgfqpoint{7.942cm}{10.291cm}}
\pgfpathlineto{\pgfqpoint{7.942cm}{9.874cm}}
\pgfpathlineto{\pgfqpoint{7.631cm}{9.874cm}}
\pgfpathlineto{\pgfqpoint{7.631cm}{10.093cm}}
\pgfpathlineto{\pgfqpoint{7.794cm}{10.093cm}}
\pgfpathlineto{\pgfqpoint{7.794cm}{9.987cm}}
\pgfpathlineto{\pgfqpoint{7.716cm}{9.987cm}}
\pgfpathlineto{\pgfqpoint{7.716cm}{10.04cm}}
\pgfpathlineto{\pgfqpoint{7.755cm}{10.04cm}}
\pgfpathlineto{\pgfqpoint{7.755cm}{10.014cm}}
\pgfpathlineto{\pgfqpoint{7.736cm}{10.014cm}}
\pgfpathlineto{\pgfqpoint{7.736cm}{10.027cm}}
\pgfpathlineto{\pgfqpoint{7.746cm}{10.027cm}}
\pgfpathlineto{\pgfqpoint{7.746cm}{10.02cm}}
\pgfpathlineto{\pgfqpoint{7.741cm}{10.02cm}}
\pgfpathlineto{\pgfqpoint{7.741cm}{10.024cm}}
\pgfpathlineto{\pgfqpoint{7.743cm}{10.024cm}}
\pgfpathlineto{\pgfqpoint{7.743cm}{10.022cm}}
\pgfpathlineto{\pgfqpoint{7.742cm}{10.022cm}}
\pgfpathlineto{\pgfqpoint{7.742cm}{10.023cm}}
\pgfpathlineto{\pgfqpoint{7.742cm}{10.023cm}}
\pgfpathlineto{\pgfqpoint{7.742cm}{10.022cm}}
\pgfpathlineto{\pgfqpoint{7.742cm}{10.022cm}}
\pgfpathlineto{\pgfqpoint{7.742cm}{10.023cm}}
\pgfpathlineto{\pgfqpoint{7.742cm}{10.023cm}}
\pgfpathlineto{\pgfqpoint{7.742cm}{10.023cm}}
\pgfpathlineto{\pgfqpoint{7.742cm}{10.023cm}}
\pgfpathlineto{\pgfqpoint{7.742cm}{10.023cm}}
\pgfpathlineto{\pgfqpoint{7.742cm}{10.023cm}}
\pgfpathlineto{\pgfqpoint{7.742cm}{10.023cm}}
\pgfpathlineto{\pgfqpoint{7.742cm}{10.023cm}}
\pgfpathlineto{\pgfqpoint{7.742cm}{10.023cm}}
\pgfpathlineto{\pgfqpoint{7.742cm}{10.023cm}}
\pgfpathlineto{\pgfqpoint{7.742cm}{10.023cm}}
\pgfpathlineto{\pgfqpoint{7.742cm}{10.023cm}}
\pgfpathlineto{\pgfqpoint{7.742cm}{10.023cm}}
\pgfpathlineto{\pgfqpoint{7.742cm}{10.023cm}}
\pgfpathlineto{\pgfqpoint{7.742cm}{10.023cm}}
\pgfpathlineto{\pgfqpoint{7.742cm}{10.023cm}}
\pgfpathlineto{\pgfqpoint{7.742cm}{10.023cm}}
\pgfpathlineto{\pgfqpoint{7.742cm}{10.023cm}}
\pgfpathlineto{\pgfqpoint{7.742cm}{10.023cm}}
\pgfpathlineto{\pgfqpoint{7.742cm}{10.023cm}}
\pgfpathlineto{\pgfqpoint{7.742cm}{10.023cm}}
\pgfpathlineto{\pgfqpoint{7.742cm}{10.023cm}}
\pgfpathlineto{\pgfqpoint{7.742cm}{10.023cm}}
\pgfpathlineto{\pgfqpoint{7.742cm}{10.023cm}}
\pgfpathlineto{\pgfqpoint{7.742cm}{10.023cm}}
\pgfpathlineto{\pgfqpoint{7.742cm}{10.023cm}}
\pgfpathlineto{\pgfqpoint{7.742cm}{10.023cm}}
\pgfpathlineto{\pgfqpoint{7.742cm}{10.023cm}}
\pgfpathlineto{\pgfqpoint{7.742cm}{10.023cm}}
\pgfpathlineto{\pgfqpoint{7.742cm}{10.023cm}}
\pgfpathlineto{\pgfqpoint{7.742cm}{10.023cm}}
\pgfpathlineto{\pgfqpoint{7.742cm}{10.023cm}}
\pgfpathlineto{\pgfqpoint{7.742cm}{10.023cm}}
\pgfpathlineto{\pgfqpoint{7.742cm}{10.023cm}}
\pgfpathlineto{\pgfqpoint{7.742cm}{10.023cm}}
\pgfpathlineto{\pgfqpoint{7.742cm}{10.023cm}}
\pgfpathlineto{\pgfqpoint{7.742cm}{10.023cm}}
\pgfpathlineto{\pgfqpoint{7.742cm}{10.023cm}}
\pgfpathlineto{\pgfqpoint{7.742cm}{10.023cm}}
\pgfpathlineto{\pgfqpoint{7.742cm}{10.023cm}}
\pgfpathlineto{\pgfqpoint{7.742cm}{10.023cm}}
\pgfpathlineto{\pgfqpoint{7.742cm}{10.023cm}}
\pgfpathlineto{\pgfqpoint{7.742cm}{10.023cm}}
\pgfpathlineto{\pgfqpoint{7.742cm}{10.023cm}}
\pgfpathlineto{\pgfqpoint{7.742cm}{10.023cm}}
\pgfpathlineto{\pgfqpoint{7.742cm}{10.023cm}}
\pgfpathlineto{\pgfqpoint{7.742cm}{10.023cm}}
\pgfpathlineto{\pgfqpoint{7.742cm}{10.023cm}}
\pgfpathlineto{\pgfqpoint{7.742cm}{10.023cm}}
\pgfpathlineto{\pgfqpoint{7.742cm}{10.023cm}}
\pgfpathlineto{\pgfqpoint{7.742cm}{10.023cm}}
\pgfpathlineto{\pgfqpoint{7.742cm}{10.023cm}}
\pgfpathlineto{\pgfqpoint{7.742cm}{10.023cm}}
\pgfpathlineto{\pgfqpoint{7.742cm}{10.023cm}}
\pgfpathlineto{\pgfqpoint{7.742cm}{10.023cm}}
\pgfpathlineto{\pgfqpoint{7.742cm}{10.023cm}}
\pgfpathlineto{\pgfqpoint{7.742cm}{10.023cm}}
\pgfpathlineto{\pgfqpoint{7.742cm}{10.023cm}}
\pgfpathlineto{\pgfqpoint{7.742cm}{10.023cm}}
\pgfpathlineto{\pgfqpoint{7.742cm}{10.023cm}}
\pgfpathlineto{\pgfqpoint{7.742cm}{10.023cm}}
\pgfpathlineto{\pgfqpoint{7.742cm}{10.023cm}}
\pgfpathlineto{\pgfqpoint{7.742cm}{10.023cm}}
\pgfpathlineto{\pgfqpoint{7.742cm}{10.023cm}}
\pgfpathlineto{\pgfqpoint{7.742cm}{10.023cm}}
\pgfpathlineto{\pgfqpoint{7.742cm}{10.023cm}}
\pgfpathlineto{\pgfqpoint{7.742cm}{10.023cm}}
\pgfpathlineto{\pgfqpoint{7.742cm}{10.023cm}}
\pgfpathlineto{\pgfqpoint{7.742cm}{10.023cm}}
\pgfpathlineto{\pgfqpoint{7.742cm}{10.023cm}}
\pgfpathlineto{\pgfqpoint{7.742cm}{10.023cm}}
\pgfpathlineto{\pgfqpoint{7.742cm}{10.023cm}}
\pgfpathlineto{\pgfqpoint{7.742cm}{10.023cm}}
\pgfpathlineto{\pgfqpoint{7.742cm}{10.023cm}}
\pgfusepath{stroke}
\end{pgfscope}
\begin{pgfscope}
\pgfpathmoveto{\pgfqpoint{0cm}{0cm}}
\pgfpathlineto{\pgfqpoint{12.7cm}{0cm}}
\pgfpathlineto{\pgfqpoint{12.7cm}{12.7cm}}
\pgfpathlineto{\pgfqpoint{0cm}{12.7cm}}
\pgfpathclose
\pgfusepath{clip}
\begin{pgfscope}
\definecolor{eps2pgf_color}{gray}{0}\pgfsetstrokecolor{eps2pgf_color}\pgfsetfillcolor{eps2pgf_color}
\pgftext[x=6.858cm,y=11.619cm,rotate=0]{{\larger\bf Cobweb plot of $ f(x) =  x  (5/2 - 7  x) $ }}
\end{pgfscope}
\end{pgfscope}
\begin{pgfscope}
\pgfpathmoveto{\pgfqpoint{2.083cm}{2.591cm}}
\pgfpathlineto{\pgfqpoint{11.633cm}{2.591cm}}
\pgfpathlineto{\pgfqpoint{11.633cm}{10.617cm}}
\pgfpathlineto{\pgfqpoint{2.083cm}{10.617cm}}
\pgfpathclose
\pgfusepath{clip}
\begin{pgfscope}
\definecolor{eps2pgf_color}{gray}{0}\pgfsetstrokecolor{eps2pgf_color}\pgfsetfillcolor{eps2pgf_color}
\pgftext[x=6.817cm,y=4.844cm,rotate=0]{Fixed points: $ \{  0, 0.214285714285714  $ \} }
\end{pgfscope}
\begin{pgfscope}
\definecolor{eps2pgf_color}{gray}{0}\pgfsetstrokecolor{eps2pgf_color}\pgfsetfillcolor{eps2pgf_color}
\pgftext[x=6.86cm,y=4.334cm,rotate=0]{Carrying capacity $ k= 0.357142857142857  $}
\end{pgfscope}
\begin{pgfscope}
\definecolor{eps2pgf_color}{gray}{0}\pgfsetstrokecolor{eps2pgf_color}\pgfsetfillcolor{eps2pgf_color}
\pgftext[x=6.869cm,y=3.631cm,rotate=0]{Rate: $ r=  2.5 \quad \rightarrow 1 < r < 3$}
\end{pgfscope}
\begin{pgfscope}
\definecolor{eps2pgf_color}{gray}{0}\pgfsetstrokecolor{eps2pgf_color}\pgfsetfillcolor{eps2pgf_color}
\pgftext[x=6.858cm,y=3.136cm,rotate=0]{{\bf $\therefore $ the fixed point is stable }}
\end{pgfscope}
\end{pgfscope}
\end{pgfscope}
\end{pgfpicture}

\endpgfgraphicnamed
\end{center}

\begin{enumerate}
  \setcounter{enumi}{4}
  
  \item In physiology, maintaining a steady level of glucose in the bloodstream is necessary for the proper functioning of all organs. To study
  this process, define $G(t)$ to be the amount of glucose in the bloodstream of a person at time $t$. Assume that glucose is absorbed from the bloodstream at a rate proportional to the concentration $G(t)$, with rate parameter $k$.
  \begin{enumerate}
    \item Write down a differential equation to describe this situation. What kind of ODE is it?
    
    \begin{Solution}
      $\dot G = k G(t) $, a homogenous linear differential equation.
    \end{Solution}
    
    \item Find the analytical solution for this equation, in terms of an initial value $G_0$ and the rate parameter $k$.
    
    \begin{Solution}
      \begin{align*}
        \frac{d G}{d t} &= k * G(t) \\
        \frac{d G}{d t} \cdot \frac{1}{G(t)} &= k \\
        \int \left( \frac{d G}{d t} \cdot \frac{1}{G(t)} \right) d t &= \int k \ d t \\
        \int \frac{1}{G(t)} dG &= \int k \ dt \\
        \log G(t) &= k t + C \\
        G(t) &= e^{k t + C} \\
      \end{align*}
    \end{Solution}
        
    \item Let the initial glucose concentration be $G_0 = 100 mg/dl$, and the glucose removal rate be $k = 0.01 / min$. Use R to plot the solution as a graph over a reasonable time interval, with properly labeled axes.
    
    \begin{Solution}
      \begin{eqnarray*}
        G(t) &= e^{k t +  C} \\
        G(0) &= e^{C} &= 100 \text{mg/dl} \\
        G(t) &= e^{0.01 \text{ min$^{-1}$} \cdot t \text{ min} + 100 \text{ mg/dl} } \\
        G(t) &= e^{100 - t / 100}
      \end{eqnarray*}
      
      
    \end{Solution}
    \item What is the equilibrium concentration of blood sugar in this model? Is the equilibrium stable or unstable?
    
    \begin{Solution}
      Under the assumption that $k$ is negative (inferred from the text), the equilibrium concentration is $0$ mg/dl.
    \end{Solution}
    
    
    \end{enumerate}
  \item Now let us assume that glucose is added to to bloodstream at a constant rate $a$, independent of glucose concentration.
  \begin{enumerate}
    \item Write down a differential equation to describe this situation. What kind of ODE is it?
    
    \begin{Solution}
      $$ \dot G = k G(t) + a $$
    \end{Solution}
    
    \item Find the analytical solution for this equation, in terms of an initial value $G_0$ and the parameters $k$ and $a$.
    \begin{Solution}
      \begin{align*}
        \dot G &= k G(t) + a \\
        \frac{d G}{d t} &= k G(t) + a \\
        G'(t) - k G(t) &= a \\
        \\
        \text{Let } \mu &= e^{\int -k dt} \tag{integrating factor} \\
        &= e^{-kt} \\
        \frac{d \mu}{d t} &= -k e^{-kt} \\
        \\
        \mu \left( G'(t) - k G(t) \right) &= \mu \, a \\
        = e^{-kt} G'(t) - k e^{-kt} G(t) &= a e^{\int k dt}\\
        = \mu G'(t) + \mu' G(t) & \\
        = (G(t) \cdot \mu)'& \tag{product rule} \\
         \\
        \int \left(G(t) \cdot \mu \right)' dt &= \int {a e^{-kt} } dt \\
        G(t) \cdot \mu + C &=\int {a e^{-kt} } dt \\
        G(t) \cdot \mu &= - \frac{a}{k} e^{-kt} + C \\
        G(t) &= - \frac{1}{e^{-kt}} \left(\frac{a}{k}  e^{-kt} + C \right) \\
        &= e^{k t} \left(C - \frac{a}{k}\right) \\
        G(0) &= C - \frac{a}{k}
      \end{align*}
    \end{Solution}
    \item Let the initial glucose concentration be $G_0 = 100 mg/dl$, the glucose removal rate be $k = 0.01 / min$, and $a = 4 mg/dl/min$. Use R to plot the solution as a graph over a reasonable time interval, with properly labeled axes.
    \begin{Solution}
      \begin{align}
        G(0) &= C - \frac{\mbox{a}}{k} \\
        100 &= C - \frac{4}{-0.01} \\
        C &= -300 \\
        100 \, {(4 \, e^{\frac{1}{100} \, t} - 3)} e^{-\frac{1}{100} \, t}
      \end{align}

\begin{Schunk}
\begin{Sinput}
 glucose <- function(t) {
   100 *  exp(-1 * t / 100) * (4 * exp(t / 100) - 3)
 }
 xlab <- "time $(t)$ {\\smaller (minutes)}"
 ylab <- "Blood glucose concentration $G(t)$ {\\smaller(mg / dl)}"
 main <- "$100 \\, {(4 \\, e^{\\frac{1}{100} \\, t} - 3)} e^{-\\frac{1}{100} \\, t}$"
 plot(glucose, from=0, to=800, xlab=xlab, ylab=ylab, main=main);
\end{Sinput}
\end{Schunk}
\end{Solution}
\bc
\beginpgfgraphicnamed{solution-nonhomogenous_part3}
% Created by Eps2pgf 0.7.0 (build on 2008-08-24) on Thu Nov 05 06:11:44 CST 2009
\begin{pgfpicture}
\pgfpathmoveto{\pgfqpoint{0cm}{0cm}}
\pgfpathlineto{\pgfqpoint{8.89cm}{0cm}}
\pgfpathlineto{\pgfqpoint{8.89cm}{8.89cm}}
\pgfpathlineto{\pgfqpoint{0cm}{8.89cm}}
\pgfpathclose
\pgfusepath{clip}
\begin{pgfscope}
\begin{pgfscope}
\end{pgfscope}
\begin{pgfscope}
\pgfpathmoveto{\pgfqpoint{2.083cm}{2.591cm}}
\pgfpathlineto{\pgfqpoint{7.823cm}{2.591cm}}
\pgfpathlineto{\pgfqpoint{7.823cm}{6.807cm}}
\pgfpathlineto{\pgfqpoint{2.083cm}{6.807cm}}
\pgfpathclose
\pgfusepath{clip}
\pgfsetdash{}{0cm}
\pgfsetlinewidth{0.265mm}
\pgfsetroundcap
\pgfsetroundjoin
\definecolor{eps2pgf_color}{gray}{0}\pgfsetstrokecolor{eps2pgf_color}\pgfsetfillcolor{eps2pgf_color}
\pgfpathmoveto{\pgfqpoint{2.296cm}{2.747cm}}
\pgfpathlineto{\pgfqpoint{2.348cm}{3.047cm}}
\pgfpathlineto{\pgfqpoint{2.402cm}{3.324cm}}
\pgfpathlineto{\pgfqpoint{2.455cm}{3.58cm}}
\pgfpathlineto{\pgfqpoint{2.508cm}{3.816cm}}
\pgfpathlineto{\pgfqpoint{2.561cm}{4.034cm}}
\pgfpathlineto{\pgfqpoint{2.614cm}{4.236cm}}
\pgfpathlineto{\pgfqpoint{2.667cm}{4.422cm}}
\pgfpathlineto{\pgfqpoint{2.721cm}{4.593cm}}
\pgfpathlineto{\pgfqpoint{2.774cm}{4.752cm}}
\pgfpathlineto{\pgfqpoint{2.827cm}{4.898cm}}
\pgfpathlineto{\pgfqpoint{2.88cm}{5.032cm}}
\pgfpathlineto{\pgfqpoint{2.933cm}{5.157cm}}
\pgfpathlineto{\pgfqpoint{2.986cm}{5.272cm}}
\pgfpathlineto{\pgfqpoint{3.04cm}{5.378cm}}
\pgfpathlineto{\pgfqpoint{3.093cm}{5.476cm}}
\pgfpathlineto{\pgfqpoint{3.146cm}{5.566cm}}
\pgfpathlineto{\pgfqpoint{3.199cm}{5.65cm}}
\pgfpathlineto{\pgfqpoint{3.252cm}{5.727cm}}
\pgfpathlineto{\pgfqpoint{3.305cm}{5.798cm}}
\pgfpathlineto{\pgfqpoint{3.358cm}{5.864cm}}
\pgfpathlineto{\pgfqpoint{3.412cm}{5.925cm}}
\pgfpathlineto{\pgfqpoint{3.465cm}{5.98cm}}
\pgfpathlineto{\pgfqpoint{3.518cm}{6.032cm}}
\pgfpathlineto{\pgfqpoint{3.571cm}{6.08cm}}
\pgfpathlineto{\pgfqpoint{3.624cm}{6.124cm}}
\pgfpathlineto{\pgfqpoint{3.677cm}{6.164cm}}
\pgfpathlineto{\pgfqpoint{3.731cm}{6.202cm}}
\pgfpathlineto{\pgfqpoint{3.784cm}{6.237cm}}
\pgfpathlineto{\pgfqpoint{3.837cm}{6.269cm}}
\pgfpathlineto{\pgfqpoint{3.89cm}{6.298cm}}
\pgfpathlineto{\pgfqpoint{3.943cm}{6.325cm}}
\pgfpathlineto{\pgfqpoint{3.996cm}{6.35cm}}
\pgfpathlineto{\pgfqpoint{4.05cm}{6.374cm}}
\pgfpathlineto{\pgfqpoint{4.102cm}{6.395cm}}
\pgfpathlineto{\pgfqpoint{4.156cm}{6.415cm}}
\pgfpathlineto{\pgfqpoint{4.209cm}{6.433cm}}
\pgfpathlineto{\pgfqpoint{4.262cm}{6.45cm}}
\pgfpathlineto{\pgfqpoint{4.315cm}{6.465cm}}
\pgfpathlineto{\pgfqpoint{4.368cm}{6.48cm}}
\pgfpathlineto{\pgfqpoint{4.421cm}{6.493cm}}
\pgfpathlineto{\pgfqpoint{4.475cm}{6.505cm}}
\pgfpathlineto{\pgfqpoint{4.528cm}{6.517cm}}
\pgfpathlineto{\pgfqpoint{4.581cm}{6.527cm}}
\pgfpathlineto{\pgfqpoint{4.634cm}{6.537cm}}
\pgfpathlineto{\pgfqpoint{4.687cm}{6.546cm}}
\pgfpathlineto{\pgfqpoint{4.74cm}{6.554cm}}
\pgfpathlineto{\pgfqpoint{4.794cm}{6.561cm}}
\pgfpathlineto{\pgfqpoint{4.847cm}{6.568cm}}
\pgfpathlineto{\pgfqpoint{4.9cm}{6.575cm}}
\pgfpathlineto{\pgfqpoint{4.953cm}{6.581cm}}
\pgfpathlineto{\pgfqpoint{5.006cm}{6.586cm}}
\pgfpathlineto{\pgfqpoint{5.059cm}{6.591cm}}
\pgfpathlineto{\pgfqpoint{5.112cm}{6.596cm}}
\pgfpathlineto{\pgfqpoint{5.166cm}{6.6cm}}
\pgfpathlineto{\pgfqpoint{5.219cm}{6.604cm}}
\pgfpathlineto{\pgfqpoint{5.272cm}{6.608cm}}
\pgfpathlineto{\pgfqpoint{5.325cm}{6.611cm}}
\pgfpathlineto{\pgfqpoint{5.378cm}{6.615cm}}
\pgfpathlineto{\pgfqpoint{5.431cm}{6.617cm}}
\pgfpathlineto{\pgfqpoint{5.485cm}{6.62cm}}
\pgfpathlineto{\pgfqpoint{5.538cm}{6.623cm}}
\pgfpathlineto{\pgfqpoint{5.591cm}{6.625cm}}
\pgfpathlineto{\pgfqpoint{5.644cm}{6.627cm}}
\pgfpathlineto{\pgfqpoint{5.697cm}{6.629cm}}
\pgfpathlineto{\pgfqpoint{5.75cm}{6.631cm}}
\pgfpathlineto{\pgfqpoint{5.804cm}{6.633cm}}
\pgfpathlineto{\pgfqpoint{5.856cm}{6.634cm}}
\pgfpathlineto{\pgfqpoint{5.91cm}{6.635cm}}
\pgfpathlineto{\pgfqpoint{5.963cm}{6.637cm}}
\pgfpathlineto{\pgfqpoint{6.016cm}{6.638cm}}
\pgfpathlineto{\pgfqpoint{6.069cm}{6.639cm}}
\pgfpathlineto{\pgfqpoint{6.122cm}{6.64cm}}
\pgfpathlineto{\pgfqpoint{6.175cm}{6.641cm}}
\pgfpathlineto{\pgfqpoint{6.229cm}{6.642cm}}
\pgfpathlineto{\pgfqpoint{6.282cm}{6.643cm}}
\pgfpathlineto{\pgfqpoint{6.335cm}{6.644cm}}
\pgfpathlineto{\pgfqpoint{6.388cm}{6.644cm}}
\pgfpathlineto{\pgfqpoint{6.441cm}{6.645cm}}
\pgfpathlineto{\pgfqpoint{6.494cm}{6.645cm}}
\pgfpathlineto{\pgfqpoint{6.548cm}{6.646cm}}
\pgfpathlineto{\pgfqpoint{6.601cm}{6.646cm}}
\pgfpathlineto{\pgfqpoint{6.654cm}{6.647cm}}
\pgfpathlineto{\pgfqpoint{6.707cm}{6.647cm}}
\pgfpathlineto{\pgfqpoint{6.76cm}{6.648cm}}
\pgfpathlineto{\pgfqpoint{6.813cm}{6.648cm}}
\pgfpathlineto{\pgfqpoint{6.866cm}{6.648cm}}
\pgfpathlineto{\pgfqpoint{6.92cm}{6.649cm}}
\pgfpathlineto{\pgfqpoint{6.973cm}{6.649cm}}
\pgfpathlineto{\pgfqpoint{7.026cm}{6.649cm}}
\pgfpathlineto{\pgfqpoint{7.079cm}{6.65cm}}
\pgfpathlineto{\pgfqpoint{7.132cm}{6.65cm}}
\pgfpathlineto{\pgfqpoint{7.185cm}{6.65cm}}
\pgfpathlineto{\pgfqpoint{7.239cm}{6.65cm}}
\pgfpathlineto{\pgfqpoint{7.292cm}{6.65cm}}
\pgfpathlineto{\pgfqpoint{7.345cm}{6.651cm}}
\pgfpathlineto{\pgfqpoint{7.398cm}{6.651cm}}
\pgfpathlineto{\pgfqpoint{7.451cm}{6.651cm}}
\pgfpathlineto{\pgfqpoint{7.504cm}{6.651cm}}
\pgfpathlineto{\pgfqpoint{7.558cm}{6.651cm}}
\pgfpathlineto{\pgfqpoint{7.61cm}{6.651cm}}
\pgfusepath{stroke}
\end{pgfscope}
\begin{pgfscope}
\pgfpathmoveto{\pgfqpoint{0cm}{0cm}}
\pgfpathlineto{\pgfqpoint{8.89cm}{0cm}}
\pgfpathlineto{\pgfqpoint{8.89cm}{8.89cm}}
\pgfpathlineto{\pgfqpoint{0cm}{8.89cm}}
\pgfpathclose
\pgfusepath{clip}
\pgfsetdash{}{0cm}
\pgfsetlinewidth{0.265mm}
\pgfsetroundcap
\pgfsetroundjoin
\definecolor{eps2pgf_color}{gray}{0}\pgfsetstrokecolor{eps2pgf_color}\pgfsetfillcolor{eps2pgf_color}
\pgfpathmoveto{\pgfqpoint{2.296cm}{2.591cm}}
\pgfpathlineto{\pgfqpoint{7.61cm}{2.591cm}}
\pgfusepath{stroke}
\pgfsetdash{}{0cm}
\pgfpathmoveto{\pgfqpoint{2.296cm}{2.591cm}}
\pgfpathlineto{\pgfqpoint{2.296cm}{2.337cm}}
\pgfusepath{stroke}
\pgfsetdash{}{0cm}
\pgfpathmoveto{\pgfqpoint{3.624cm}{2.591cm}}
\pgfpathlineto{\pgfqpoint{3.624cm}{2.337cm}}
\pgfusepath{stroke}
\pgfsetdash{}{0cm}
\pgfpathmoveto{\pgfqpoint{4.953cm}{2.591cm}}
\pgfpathlineto{\pgfqpoint{4.953cm}{2.337cm}}
\pgfusepath{stroke}
\pgfsetdash{}{0cm}
\pgfpathmoveto{\pgfqpoint{6.282cm}{2.591cm}}
\pgfpathlineto{\pgfqpoint{6.282cm}{2.337cm}}
\pgfusepath{stroke}
\pgfsetdash{}{0cm}
\pgfpathmoveto{\pgfqpoint{7.61cm}{2.591cm}}
\pgfpathlineto{\pgfqpoint{7.61cm}{2.337cm}}
\pgfusepath{stroke}
\begin{pgfscope}
\pgftext[x=2.296cm,y=1.821cm,rotate=0]{0}
\end{pgfscope}
\begin{pgfscope}
\pgftext[x=3.622cm,y=1.821cm,rotate=0]{200}
\end{pgfscope}
\begin{pgfscope}
\pgftext[x=4.95cm,y=1.821cm,rotate=0]{400}
\end{pgfscope}
\begin{pgfscope}
\pgftext[x=6.282cm,y=1.821cm,rotate=0]{600}
\end{pgfscope}
\begin{pgfscope}
\pgftext[x=7.611cm,y=1.821cm,rotate=0]{800}
\end{pgfscope}
\pgfsetdash{}{0cm}
\pgfpathmoveto{\pgfqpoint{2.083cm}{2.747cm}}
\pgfpathlineto{\pgfqpoint{2.083cm}{6.652cm}}
\pgfusepath{stroke}
\pgfsetdash{}{0cm}
\pgfpathmoveto{\pgfqpoint{2.083cm}{2.747cm}}
\pgfpathlineto{\pgfqpoint{1.829cm}{2.747cm}}
\pgfusepath{stroke}
\pgfsetdash{}{0cm}
\pgfpathmoveto{\pgfqpoint{2.083cm}{3.398cm}}
\pgfpathlineto{\pgfqpoint{1.829cm}{3.398cm}}
\pgfusepath{stroke}
\pgfsetdash{}{0cm}
\pgfpathmoveto{\pgfqpoint{2.083cm}{4.049cm}}
\pgfpathlineto{\pgfqpoint{1.829cm}{4.049cm}}
\pgfusepath{stroke}
\pgfsetdash{}{0cm}
\pgfpathmoveto{\pgfqpoint{2.083cm}{4.7cm}}
\pgfpathlineto{\pgfqpoint{1.829cm}{4.7cm}}
\pgfusepath{stroke}
\pgfsetdash{}{0cm}
\pgfpathmoveto{\pgfqpoint{2.083cm}{5.351cm}}
\pgfpathlineto{\pgfqpoint{1.829cm}{5.351cm}}
\pgfusepath{stroke}
\pgfsetdash{}{0cm}
\pgfpathmoveto{\pgfqpoint{2.083cm}{6.001cm}}
\pgfpathlineto{\pgfqpoint{1.829cm}{6.001cm}}
\pgfusepath{stroke}
\pgfsetdash{}{0cm}
\pgfpathmoveto{\pgfqpoint{2.083cm}{6.652cm}}
\pgfpathlineto{\pgfqpoint{1.829cm}{6.652cm}}
\pgfusepath{stroke}
\begin{pgfscope}
\pgftext[x=1.328cm,y=2.761cm,rotate=90]{100}
\end{pgfscope}
\begin{pgfscope}
\pgftext[x=1.328cm,y=4.047cm,rotate=90]{200}
\end{pgfscope}
\begin{pgfscope}
\pgftext[x=1.328cm,y=5.35cm,rotate=90]{300}
\end{pgfscope}
\begin{pgfscope}
\pgftext[x=1.328cm,y=6.65cm,rotate=90]{400}
\end{pgfscope}
\pgfsetdash{}{0cm}
\pgfpathmoveto{\pgfqpoint{2.083cm}{2.591cm}}
\pgfpathlineto{\pgfqpoint{7.823cm}{2.591cm}}
\pgfpathlineto{\pgfqpoint{7.823cm}{6.807cm}}
\pgfpathlineto{\pgfqpoint{2.083cm}{6.807cm}}
\pgfpathlineto{\pgfqpoint{2.083cm}{2.591cm}}
\pgfusepath{stroke}
\end{pgfscope}
\begin{pgfscope}
\pgfpathmoveto{\pgfqpoint{0cm}{0cm}}
\pgfpathlineto{\pgfqpoint{8.89cm}{0cm}}
\pgfpathlineto{\pgfqpoint{8.89cm}{8.89cm}}
\pgfpathlineto{\pgfqpoint{0cm}{8.89cm}}
\pgfpathclose
\pgfusepath{clip}
\begin{pgfscope}
\definecolor{eps2pgf_color}{gray}{0}\pgfsetstrokecolor{eps2pgf_color}\pgfsetfillcolor{eps2pgf_color}
\pgftext[x=4.952cm,y=7.811cm,rotate=0]{$100 \, {(4 \, e^{\frac{1}{100} \, t} - 3)} e^{-\frac{1}{100} \, t}$}
\end{pgfscope}
\begin{pgfscope}
\definecolor{eps2pgf_color}{gray}{0}\pgfsetstrokecolor{eps2pgf_color}\pgfsetfillcolor{eps2pgf_color}
\pgftext[x=4.947cm,y=0.781cm,rotate=0]{time $(t)$ {\smaller (minutes)}}
\end{pgfscope}
\begin{pgfscope}
\definecolor{eps2pgf_color}{gray}{0}\pgfsetstrokecolor{eps2pgf_color}\pgfsetfillcolor{eps2pgf_color}
\pgftext[x=0.34cm,y=4.706cm,rotate=90]{Blood glucose concentration $G(t)$ {\smaller(mg / dl)}}
\end{pgfscope}
\end{pgfscope}
\end{pgfscope}
\end{pgfpicture}

\endpgfgraphicnamed
\ec    
    \item What is the equilibrium concentration of blood sugar in the model with glucose infusion? Is the equilibrium stable or unstable?
    \begin{Solution}
      \begin{align}
        G(t) &= 400 - 300 e^{\frac{-t}{100}} \\
        \frac{dG}{dt} &= 3 e^{\frac{-t}{100}} \\
        \lim_{t \rightarrow \infty} e^{\frac{-t}{100}} = 0 \\
        \lim_{t \rightarrow \infty} G(t) &= 400 - 300 \cdot 0 \\
          &= 100 \\
        \lim_{t \rightarrow \infty} G'(t) &= 0 \\
      \end{align}
      Thus equilibrium is reached and stable at 400 mg / dl.
    \end{Solution}
  \end{enumerate}
%  \item \emph{Analytically} find all fixed points for the following logistic models.  You may use a pencil and paper to solve these problems (hand in for credit).


%   \item \emph{Enzyme kinetics.}  Consider the following reaction and observations.
%   \begin{center}
%     \ce{A + 2B <=> AB2}
%   \end{center}
%   \bc
%     \begin{tabular}{c c c c c c c c}
%     \toprule
%     & & \multicolumn{6}{c}{Concentration (mM)} \\
%     \multicolumn{2}{c}{\hspace{0.8cm}Time (ms)}\hspace{1.6cm} & \multicolumn{3}{c}{Before} & \multicolumn{3}{c}{After} \\
%     & & [\ce{A}] & [\ce{B}] & [\ce{AB2}] & [\ce{A}] & [\ce{B}] & [\ce{AB2}] \\
%     \midrule
%     1.177 & 1.253 & 1.021 & 1.011 & 0.014 & 1.012 & 0.992 & 0.023 \\ 
%     2.662 & 2.820 & 1.012 & 2.015 & 0.006 & 0.943 & 1.879 & 0.074 \\ 
%     3.184 & 3.670 & 1.023 & 3.015 & 0.018 & 0.918 & 2.805 & 0.123 \\ 
%     4.495 & 4.586 & 2.040 & 1.003 & 0.016 & 1.937 & 0.797 & 0.119 \\ 
%     5.492 & 5.827 & 2.019 & 2.018 & 0.012 & 1.679 & 1.339 & 0.351 \\ 
%     6.678 & 6.769 & 2.026 & 3.009 & 0.016 & 1.608 & 2.174 & 0.434 \\ 
%     7.221 & 7.805 & 0.021 & 1.008 & 1.005 & 0.061 & 1.089 & 0.965 \\ 
%     8.220 & 8.256 & 0.014 & 2.016 & 2.018 & 0.040 & 2.068 & 1.992 \\ 
%     9.260 & 9.802 & 0.032 & 3.014 & 3.002 & 0.071 & 3.092 & 2.962 \\ 
%     10.380 & 10.698 & 0.015 & 1.018 & 4.013 & 0.091 & 1.170 & 3.937 \\ 
%     \bottomrule
%   \end{tabular}
%   \ec
%   \begin{enumerate}
%     \item Calculate the rate law for the forward and reverse reactions. (\emph{Hint: the file c3e4.R on the website contains this raw data}).
%     \item Calculate $K_1$ and $K_{-1}$, the forward and backwards rate coefficients.
%   \end{enumerate}
%   \item \emph{Goldman equation and Action potentials}.  The \emph{Nernst potential} of a single ion \ce{A^z} with charge $z$ across a cell membrane is given by
%   \begin{align}
%     E_{\ce{A}} &= \frac{RT}{zF} \log \frac{[\ce{A^z}_o]}{[\ce{A^z}_i]} \\
%     R &= 8.314 J K^{-1} mol^{-1} \text{(Universal gas constant)} \\
%     T &= \text{Temperature in kelvins.  Assume $T=298$} \\
%     F &= 9.648 \times 10^4 C mol^{-1} \text{(Faraday constant; i.e. charge per mole of electrons)} \\
%     [\ce{A^z_o}] &= \text{Extracellular concentration} \\
%     [\ce{A^z_i}] &= \text{Intracellular concentration}
%   \end{align}
%   Thus for sodium (\ce{NA+}), the Nernst potential, given $[\ce{NA+}_o] = 100$ mM and $[\ce{NA+}_i] = 10$ mM is
%   \begin{align*}
%     E_{\ce{NA+}} &= \frac{RT}{zF} \log \frac{[\ce{NA+}_o]}{[\ce{NA+}_i]} \\
%     &= \frac{8.314 J K^{-1} mol^{-1} \cdot 298 K}{1 \cdot 9.68 \times 10^4 C mol^{-1}} * \log{\frac{100}{10}} \\
%     &= 25.3 mV \log 10 \\
%     &= 58 mV
%   \end{align*}
%   Furthermore, the total potential across a membrane by combining the products of the individual ionic Nernst potentials and ionic conductances.
%   \begin{equation}
%     E_{m} = \frac{RT}{F} \ln{ \left( \frac{ \sum_{i}^{N} P_{M^{+}_{i}}[M^{+}_{i}]_\text{out} + \sum_{j}^{M} P_{A^{-}_{j}}[A^{-}_{j}]_\text{in}}{ \sum_{i}^{N} P_{M^{+}_{i}}[M^{+}_{i}]_\text{in} + \sum_{j}^{M} P_{A^{-}_{j}}[A^{-}_{j}]_\text{out}} \right) }
%     \end{equation} 
  \item \emph{Not required.  A good exercise which relates to numerical approximation.} \begin{enumerate}
    \item \emph{Binary Search.} Write a function which, given a sorted vector $X$ of $n$ unique integers, and an integer $k$, returns the index of $k$ in vector $X$ if $k \in X$ and $-1$ otherwise.  Your solution should perform approximately $\log n$ comparisons in the ``worst case'' scenario. \emph{Hint:}  one example of a ``worst case'' is when the $X_0 = k$.
  \end{enumerate}
  \end{enumerate}
\begin{Solution}
\begin{Schunk}
\begin{Sinput}
 bSearch1 <- function(intVec, value) {
   nIter <- 0;
   bottom <- 1;
   top <- length(intVec);
   testVal <- intVec[floor(mean(c(top, bottom)))];
   while (testVal != value) {
     nIter <- nIter + 1;   # Just to count steps
   if (testVal < value) {
     bottom <- ceiling(mean(c(top, bottom)));
   } else {
     top <- floor(mean(c(top, bottom)));
   }
   testVal <- intVec[floor(mean(c(top, bottom)))];
   }
   cat(sprintf("Match found in %d steps\n", nIter));
   return(floor(mean(c(top, bottom))));
 }
 iVec <- sort(sample(1:1000, 10));
 for (value in iVec) {
   print(bSearch1(iVec, value));
 }
\end{Sinput}
\begin{Soutput}
Match found in 3 steps
[1] 1
Match found in 2 steps
[1] 2
Match found in 1 steps
[1] 3
Match found in 2 steps
[1] 4
Match found in 0 steps
[1] 5
Match found in 3 steps
[1] 6
Match found in 2 steps
[1] 7
Match found in 1 steps
[1] 8
Match found in 2 steps
[1] 9
Match found in 4 steps
[1] 10
\end{Soutput}
\end{Schunk}
\end{Solution}

\begin{enumerate}
\begin{enumerate}
\setcounter{enumii}{1}

  \item Modify the previous solution so that the requirement of \emph{unique} integers may be relaxed.  In the case that multiple solutions exist return a vector of all such solutions.
  \end{enumerate}
\end{enumerate}

\begin{boxedminipage}{\textwidth}
\begin{Schunk}
\begin{Sinput}
 bSearch2 <- function(intVals, value) {
   ## Sanity Check, bail if unsure 
   if (!(is.numeric(c(intVals, value)))) {
     stop('Error: parameters must be numeric!\n');
   } else if (length(value) != 1) {
     stop('Error: value must be a vector of length 1!\n');
   } else if (is.unsorted(intVals)) {
     stop('Error: intVals must be sorted!\n');
   } else if (!(all(is.finite(c(intVals, value))))) {
     stop('Error: parameters may not contain infinite or undefined values.\n');
   }
 
   ## Cheating, a bit - ensure value is in vector
   if (!(value %in% intVals)) {
     return(numeric(length=0));
     #TODO: Remove this hack and code the case loop properly 10/13/2009 DMR
   }
   
   ## There are two 'breakpoints' - consecutive pairs (a,b) in 
   ##   intVals to find such that exactly one of a or b == value
 
   vRange <- numeric(length=2);
 
   ## Find top breakpoint
   bottom <- 1; top <- length(intVals);
   idx1 <- floor(mean(c(top, bottom)));
   idx2 <- min(c(idx1 + 1, top));
   while (TRUE) {          ## Infinite; use 'break' to escape
     if(intVals[idx1] < value) {
       bottom <- idx1;
     } else if (intVals[idx1] > value) {
       top <- idx1;
     ## All remaining cases must have intVals[idx1] == value
     } else if (idx2 == top) {   ## Can't go any higher
       vRange[2] <- top;         ##   must have found it!
     ## Note: vRange is referenced by lexical scope
       break;                    ## Breaks the while loop
     } else if (intVals[idx2] == value) {  ## 'in' range but not at top
       bottom <- idx1
     } else {                    ## Found it
       vRange[2] <- idx1;
       break;                    ## Break the while loop
     }
     idx1 <- floor(mean(c(top, bottom)));  ## Refine the interval
     idx2 <- min(c(idx1 + 1, top));
   }
 
   ## Find bottom breakpoint - similar to above;
   bottom <- 1; top <- length(intVals);
   idx1 <- ceiling(mean(c(top, bottom)));
   idx2 <- max(c(idx1 - 1, 1));
   while (TRUE) {
     if(intVals[idx1] < value) {
       bottom <- idx1;
     } else if (intVals[idx1] > value) {
       top <- idx1;
     } else if (idx2 == 1) {
       vRange[1] <- 1;
       return(vRange[1]:vRange[2])
     } else if (intVals[idx2] == value){
       top <- idx1
     } else {
       vRange[1] <- idx1;
       return(vRange[1]:vRange[2]);
     }
     idx1 <- ceiling(mean(c(top, bottom)));
     idx2 <- max(c(idx1 -1, 1));
   }
 }
\end{Sinput}
\end{Schunk}
\end{boxedminipage}
\begin{Solution}
\begin{Schunk}
\begin{Sinput}
 iVec2 <- c();
 iVec <- sort(sample(unique(floor(runif(n=50, min=-250, max=250))), 25));
 for (i in 1:5) { 
   bnds <- sort(sample(1:25, 2));
   iVec2 <- c(iVec2, iVec[bnds[1]:bnds[2]])
 }
 iVec2 <- sort(iVec2)
 iVec2
\end{Sinput}
\begin{Soutput}
 [1] -215 -210 -199 -187 -187 -184 -184 -158 -158
[10] -158 -153 -153 -153 -148 -148 -148 -148 -136
[19] -136 -136 -103 -103 -103 -103  -70  -70  -70
[28]  -70  -58  -58  -58  -58    8    8    8    8
[37]   67   67   67   73   73   73  102  102  102
[46]  111  111  111  123  123  123  176  176  190
[55]  201
\end{Soutput}
\begin{Sinput}
 for(i in 1:5) {
   value <- sample(iVec2[-1], 1);
   cat(sprintf('Iteration %d of %d: searching for value %s ...\n', 
               i, 5, as.character(value)));
   print(bSearch2(iVec2, value));
 }
\end{Sinput}
\begin{Soutput}
Iteration 1 of 5: searching for value -153 ...
[1] 11 12 13 14
Iteration 2 of 5: searching for value -153 ...
[1] 11 12 13 14
Iteration 3 of 5: searching for value -103 ...
[1] 21 22 23 24
Iteration 4 of 5: searching for value -70 ...
[1] 25 26 27 28
Iteration 5 of 5: searching for value 123 ...
[1] 49 50 51
\end{Soutput}
\end{Schunk}
\end{Solution}

\end{document}
