

\documentclass[9pt]{beamer}

\usepackage[Rbeamer]{Rosenberg}

%\usepackage{Sweave}



\definecolor{ricolor}{rgb}{0.101, 0.043, 0.432}
\newcommand{\ri}[1]{\textcolor{blue}{\tt\smaller #1}}


\title{Common questions (and answers) about \R  }
\author{David M. Rosenberg}
\institute[University of Chicago] % (optional, but mostly needed)
{
  Committee on Neurobiology\\
  University of Chicago
}
% - Use the \inst command only if there are several affiliations.
% - Keep it simple, no one is interested in your street address.

\date{\today}

\subject{Talks}
% This is only inserted into the PDF information catalog. Can be left
% out. 



% If you have a file called "university-logo-filename.xxx", where xxx
% is a graphic format that can be processed by latex or pdflatex,
% resp., then you can add a logo as follows:

% \pgfdeclareimage[height=0.5cm]{university-logo}{university-logo-filename}
% \logo{\pgfuseimage{university-logo}}



% Delete this, if you do not want the table of contents to pop up at
% the beginning of each subsection:
%\AtBeginSection[]
\newcommand{\makeOutline}{%
  \begin{frame}<beamer>{Outline}
    \begin{columns}[c]
      \column{0.5\textwidth}
        \tableofcontents[currentsection,currentsubsection]
    \end{columns}
  \end{frame}
}

\AtBeginSection[]
{
  \show\section
  \begin{frame}<beamer>{Outline}
  \begin{columns}[c]
  \column{0.5\textwidth}
    \tableofcontents[currentsection,sections={<1-3>}]
  \column{0.5\textwidth}
    \tableofcontents[currentsection,sections={<4-5>}]
  \end{columns}
  \end{frame}
}


% If you wish to uncover everything in a step-wise fashion, uncomment
% the following command: 

%\beamerdefaultoverlayspecification{<+->}


\begin{document}

\begin{frame}
  \titlepage
\end{frame}

\begin{frame}{Outline}
  \begin{columns}[c]
    \column{0.5\textwidth}
      \tableofcontents
    \end{columns}
\end{frame}


% Since this a solution template for a generic talk, very little can
% be said about how it should be structured. However, the talk length
% of between 15min and 45min and the theme suggest that you stick to
% the following rules:  

% - Exactly two or three sections (other than the summary).
% - At *most* three subsections per section.
% - Talk about 30s to 2min per frame. So there should be between about
%   15 and 30 frames, all told.

\section{Admin issues}
\subsection{Things to ask for}

\begin{frame}{Admin issues}{Things to ask for}
\begin{itemize}[<+->]
  \item \textbf{Email addresses:}\\
    If you use a non-University of Chicago email account, let me know either via an email from your {\tt uchicago.edu} email account or a written note.
  \end{itemize}
  \begin{onlyenv}<2->\begin{itemize}[<+->]
    \item \textbf{Use of lab time:}\\
      What do you think would be the best use of your time in lab.
      \begin{itemize}[<+->]
        \item Status quo.  Labs are principally time for you to try to work on the lab.
        \item \emph{Short} lectures. Presentations like this (but shorter; 10-20 minutes) at the start / end / middle of labs seems helpful.
        \item Review of the previous week's assignment and / or group discussion of the different ways you solved the problems.
        \item Discussion / question-and-answer time about the topics and concepts presented in lecture \emph{not} directly pertaining to computational programming.
      \end{itemize}
\end{itemize}
\end{onlyenv}
\end{frame}
  

\subsection{Things}

\end{document}
