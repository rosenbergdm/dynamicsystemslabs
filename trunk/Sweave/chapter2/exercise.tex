%
%  Created by David Rosenberg on 2009-09-10.
%  Copyright (c) 2009 University of Chicago. All rights reserved.
%
\documentclass[10pt,letterpaper]{article}
\usepackage[Rpdflatex]{Rosenberg}
\usepackage{svn-multi}
\usepackage{tikz}
\usepackage[version=3]{mhchem}

\svnidlong
{$LastChangedDate: 2009-10-20 19:41:04 -0500 (Tue, 20 Oct 2009) $}
{$LastChangedRevision: 161 $}
{$LastChangedBy: root $}
% \svnid{$Id: example_main.tex 146 2008-12-03 13:29:19Z martin $}
% Don't forget to set the svn property 'svn:keywords' to
% 'HeadURL LastChangedDate LastChangedRevision LastChangedBy' or
% 'Id' or both depending if you use \svnidlong and/or \svnid
%
\newcommand{\svnfooter}{Last Changed Rev: \svnkw{LastChangedRevision}}
\svnRegisterAuthor{davidrosenberg}{David M. Rosenberg}


%\usepackage{Sweave}





\title{Introduction to computational programming\\\smaller Chapter 2 Lab Exercise \\\smaller Cobweb plots and stability}
\author{David M. Rosenberg\\\small University of Chicago\\\small Committee on Neurobiology\medskip\\
{\footnotesize \parbox[t]{10cm} {
Version control information:
\begin{tabbing}
\footnotesize\sffamily
 Last changes revision: \= \kill
 Last changed date: \> \svndate\\
 Last changes revision: \> \svnrev\\
 Version: \> \svnFullRevision*{\svnrev}\\
 Last changed by: \> David M. Rosenberg\\
\end{tabbing}
}
}}
\begin{document}

\maketitle

\section*{Overview}

In this exercise we will build more complicated (cobweb) plots and continue our analysis of stability and fixed points.

\part{Tutorial}

\section{Example: Cobweb plots} % (fold)
\label{sec:example_cobweb_plots}

% section example_cobweb_plots (end)

Lets start by examining the provided pseudocode for cobweb plot generation.

\textbf{Pseudocode for producing a cobweb plot for an iterated map:}
\begin{enumerate}
\item Define the iterated map function \texttt{F(x)}
\item Choose an initial condition $x_0$ and store it as the first element of the array $x$ (we will start at $x[0]$, but in some languages the first index has to be 1)
\item Store in \texttt{y[0]} the value 0 (start the graph on the x-axis)
\item For $i$ starting at 1, repeat until $i$ exceeds the specified number of iterations
\begin{enumerate}
\item store in \texttt{y[2i-1]} the value of \texttt{F(x[2i-2])}  (draw a vertical line to the graph of $y=f(x)$)
\item store in \texttt{y[2i]} the value of \texttt{y[2i-1]}  (draw a horizontal line to the graph of $y=x$)
\item store in \texttt{x[2i-1]} the value \texttt{x[2i-2]} (draw a vertical line to the graph of $y=f(x)$)
\item store in \texttt{x[2i]} the value of \texttt{y[2i]} (draw a horizontal line to the graph of $y=x$)
\item increment $i$ by 1
\end{enumerate}
\item plot the graph of $F(x)$
\item plot the graph of $y=x$
\item plot the array $y$ versus the array $x$
\end{enumerate}

In the more compact form, this becomes

\begin{pcode}[Cobweb plot algorithm][lst:cobwebplot1]
  function F(x: real)   // iterated map function
  var x0 : real         // initial condition
  var N : integer       // maximum number of iterations
  var X,Y : real array [1..N]
  x[0] := x0
  Y[0] := 0
  for i := 1 to N do
  begin
     Y[2i-1] := F(X[2i-2])
     Y[2i] := F(Y[2i-1])
     X[2i-1] := F(X[2i-2])
     X[2i] := F(Y[2i])
  end
  plot F(x)
  plot y = x
  plot X,Y
\end{pcode}

The following example generates a ``cobweb'' plot for the logistic model given by $f(x) = x \left( 2-\frac{x}{2} \right)$

\begin{Rlistfloat}[h]
\begin{multicols}{2}
  \begin{minipage}{0.4\textwidth}
\begin{Schunk}
\begin{Sinput}
> cobImapFun <- function(x) {
+   return(2 * x - 0.5 * x ** 2);
+ }
> max_iter <- 50;
> y_vector <- x_vector <- numeric(
+   max_iter * 2);
> y_vector[1] <- 0;
> x_vector[1] <- 0.5;
> for(ii in 1:50) {
+   y_vector[2 * ii] <- cobImapFun(
+     x_vector[2 * ii - 1]);
+   y_vector[2 * ii + 1] <-
+     y_vector[2 * ii];
+   x_vector[2 * ii] <- 
+     x_vector[2 * ii - 1];
+   x_vector[2 * ii + 1] <- 
+     y_vector[2 * ii + 1];
+ }
> plot( -10:50 / 10, 
+       cobImapFun(-10:50 / 10),
+       xlab='x', 
+       ylab='y', 
+       main='Example cobweb',
+       type='l', 
+       xlim=c(0, 4),
+       ylim=c(0, 2.5), 
+       xaxs='i', 
+       yaxs='i');
> lines(-2:6, -2:6, col='red');
> lines(x_vector, y_vector, 
+       col='blue');
\end{Sinput}
\end{Schunk}
  \end{minipage}
\begin{minipage}{0.6\textwidth}
  \bc
\beginpgfgraphicnamed{exercise-cobwebplot_func_11}
% Created by Eps2pgf 0.7.0 (build on 2008-08-24) on Fri Nov 06 11:50:19 CST 2009
\begin{pgfpicture}
\pgfpathmoveto{\pgfqpoint{0cm}{0cm}}
\pgfpathlineto{\pgfqpoint{10.16cm}{0cm}}
\pgfpathlineto{\pgfqpoint{10.16cm}{10.16cm}}
\pgfpathlineto{\pgfqpoint{0cm}{10.16cm}}
\pgfpathclose
\pgfusepath{clip}
\begin{pgfscope}
\begin{pgfscope}
\end{pgfscope}
\begin{pgfscope}
\pgfpathmoveto{\pgfqpoint{2.083cm}{2.591cm}}
\pgfpathlineto{\pgfqpoint{9.093cm}{2.591cm}}
\pgfpathlineto{\pgfqpoint{9.093cm}{8.077cm}}
\pgfpathlineto{\pgfqpoint{2.083cm}{8.077cm}}
\pgfpathclose
\pgfusepath{clip}
\pgfsetdash{}{0cm}
\pgfsetlinewidth{0.265mm}
\pgfsetroundcap
\pgfsetroundjoin
\definecolor{eps2pgf_color}{gray}{0}\pgfsetstrokecolor{eps2pgf_color}\pgfsetfillcolor{eps2pgf_color}
\pgfpathmoveto{\pgfqpoint{1.168cm}{0cm}}
\pgfpathlineto{\pgfqpoint{1.206cm}{0.122cm}}
\pgfpathlineto{\pgfqpoint{1.382cm}{0.66cm}}
\pgfpathlineto{\pgfqpoint{1.557cm}{1.175cm}}
\pgfpathlineto{\pgfqpoint{1.732cm}{1.669cm}}
\pgfpathlineto{\pgfqpoint{1.907cm}{2.141cm}}
\pgfpathlineto{\pgfqpoint{2.083cm}{2.591cm}}
\pgfpathlineto{\pgfqpoint{2.258cm}{3.019cm}}
\pgfpathlineto{\pgfqpoint{2.433cm}{3.425cm}}
\pgfpathlineto{\pgfqpoint{2.608cm}{3.809cm}}
\pgfpathlineto{\pgfqpoint{2.784cm}{4.171cm}}
\pgfpathlineto{\pgfqpoint{2.959cm}{4.511cm}}
\pgfpathlineto{\pgfqpoint{3.134cm}{4.829cm}}
\pgfpathlineto{\pgfqpoint{3.31cm}{5.126cm}}
\pgfpathlineto{\pgfqpoint{3.485cm}{5.4cm}}
\pgfpathlineto{\pgfqpoint{3.66cm}{5.652cm}}
\pgfpathlineto{\pgfqpoint{3.835cm}{5.883cm}}
\pgfpathlineto{\pgfqpoint{4.011cm}{6.091cm}}
\pgfpathlineto{\pgfqpoint{4.186cm}{6.278cm}}
\pgfpathlineto{\pgfqpoint{4.361cm}{6.442cm}}
\pgfpathlineto{\pgfqpoint{4.536cm}{6.585cm}}
\pgfpathlineto{\pgfqpoint{4.712cm}{6.706cm}}
\pgfpathlineto{\pgfqpoint{4.887cm}{6.804cm}}
\pgfpathlineto{\pgfqpoint{5.062cm}{6.881cm}}
\pgfpathlineto{\pgfqpoint{5.237cm}{6.936cm}}
\pgfpathlineto{\pgfqpoint{5.413cm}{6.969cm}}
\pgfpathlineto{\pgfqpoint{5.588cm}{6.98cm}}
\pgfpathlineto{\pgfqpoint{5.763cm}{6.969cm}}
\pgfpathlineto{\pgfqpoint{5.939cm}{6.936cm}}
\pgfpathlineto{\pgfqpoint{6.114cm}{6.881cm}}
\pgfpathlineto{\pgfqpoint{6.289cm}{6.804cm}}
\pgfpathlineto{\pgfqpoint{6.464cm}{6.706cm}}
\pgfpathlineto{\pgfqpoint{6.64cm}{6.585cm}}
\pgfpathlineto{\pgfqpoint{6.815cm}{6.442cm}}
\pgfpathlineto{\pgfqpoint{6.99cm}{6.278cm}}
\pgfpathlineto{\pgfqpoint{7.165cm}{6.091cm}}
\pgfpathlineto{\pgfqpoint{7.341cm}{5.883cm}}
\pgfpathlineto{\pgfqpoint{7.516cm}{5.652cm}}
\pgfpathlineto{\pgfqpoint{7.691cm}{5.4cm}}
\pgfpathlineto{\pgfqpoint{7.866cm}{5.126cm}}
\pgfpathlineto{\pgfqpoint{8.042cm}{4.829cm}}
\pgfpathlineto{\pgfqpoint{8.217cm}{4.511cm}}
\pgfpathlineto{\pgfqpoint{8.392cm}{4.171cm}}
\pgfpathlineto{\pgfqpoint{8.568cm}{3.809cm}}
\pgfpathlineto{\pgfqpoint{8.743cm}{3.425cm}}
\pgfpathlineto{\pgfqpoint{8.918cm}{3.019cm}}
\pgfpathlineto{\pgfqpoint{9.093cm}{2.591cm}}
\pgfpathlineto{\pgfqpoint{9.269cm}{2.141cm}}
\pgfpathlineto{\pgfqpoint{9.444cm}{1.669cm}}
\pgfpathlineto{\pgfqpoint{9.619cm}{1.175cm}}
\pgfpathlineto{\pgfqpoint{9.794cm}{0.66cm}}
\pgfpathlineto{\pgfqpoint{9.97cm}{0.122cm}}
\pgfpathlineto{\pgfqpoint{10.008cm}{0cm}}
\pgfusepath{stroke}
\end{pgfscope}
\begin{pgfscope}
\pgfpathmoveto{\pgfqpoint{0cm}{0cm}}
\pgfpathlineto{\pgfqpoint{10.16cm}{0cm}}
\pgfpathlineto{\pgfqpoint{10.16cm}{10.16cm}}
\pgfpathlineto{\pgfqpoint{0cm}{10.16cm}}
\pgfpathclose
\pgfusepath{clip}
\pgfsetdash{}{0cm}
\pgfsetlinewidth{0.265mm}
\pgfsetroundcap
\pgfsetroundjoin
\definecolor{eps2pgf_color}{gray}{0}\pgfsetstrokecolor{eps2pgf_color}\pgfsetfillcolor{eps2pgf_color}
\pgfpathmoveto{\pgfqpoint{2.083cm}{2.591cm}}
\pgfpathlineto{\pgfqpoint{9.093cm}{2.591cm}}
\pgfusepath{stroke}
\pgfsetdash{}{0cm}
\pgfpathmoveto{\pgfqpoint{2.083cm}{2.591cm}}
\pgfpathlineto{\pgfqpoint{2.083cm}{2.337cm}}
\pgfusepath{stroke}
\pgfsetdash{}{0cm}
\pgfpathmoveto{\pgfqpoint{3.835cm}{2.591cm}}
\pgfpathlineto{\pgfqpoint{3.835cm}{2.337cm}}
\pgfusepath{stroke}
\pgfsetdash{}{0cm}
\pgfpathmoveto{\pgfqpoint{5.588cm}{2.591cm}}
\pgfpathlineto{\pgfqpoint{5.588cm}{2.337cm}}
\pgfusepath{stroke}
\pgfsetdash{}{0cm}
\pgfpathmoveto{\pgfqpoint{7.341cm}{2.591cm}}
\pgfpathlineto{\pgfqpoint{7.341cm}{2.337cm}}
\pgfusepath{stroke}
\pgfsetdash{}{0cm}
\pgfpathmoveto{\pgfqpoint{9.093cm}{2.591cm}}
\pgfpathlineto{\pgfqpoint{9.093cm}{2.337cm}}
\pgfusepath{stroke}
\begin{pgfscope}
\pgftext[x=2.083cm,y=1.821cm,rotate=0]{0}
\end{pgfscope}
\begin{pgfscope}
\pgftext[x=3.815cm,y=1.825cm,rotate=0]{1}
\end{pgfscope}
\begin{pgfscope}
\pgftext[x=5.583cm,y=1.825cm,rotate=0]{2}
\end{pgfscope}
\begin{pgfscope}
\pgftext[x=7.341cm,y=1.821cm,rotate=0]{3}
\end{pgfscope}
\begin{pgfscope}
\pgftext[x=9.092cm,y=1.825cm,rotate=0]{4}
\end{pgfscope}
\pgfsetdash{}{0cm}
\pgfpathmoveto{\pgfqpoint{2.083cm}{2.591cm}}
\pgfpathlineto{\pgfqpoint{2.083cm}{8.077cm}}
\pgfusepath{stroke}
\pgfsetdash{}{0cm}
\pgfpathmoveto{\pgfqpoint{2.083cm}{2.591cm}}
\pgfpathlineto{\pgfqpoint{1.829cm}{2.591cm}}
\pgfusepath{stroke}
\pgfsetdash{}{0cm}
\pgfpathmoveto{\pgfqpoint{2.083cm}{3.688cm}}
\pgfpathlineto{\pgfqpoint{1.829cm}{3.688cm}}
\pgfusepath{stroke}
\pgfsetdash{}{0cm}
\pgfpathmoveto{\pgfqpoint{2.083cm}{4.785cm}}
\pgfpathlineto{\pgfqpoint{1.829cm}{4.785cm}}
\pgfusepath{stroke}
\pgfsetdash{}{0cm}
\pgfpathmoveto{\pgfqpoint{2.083cm}{5.883cm}}
\pgfpathlineto{\pgfqpoint{1.829cm}{5.883cm}}
\pgfusepath{stroke}
\pgfsetdash{}{0cm}
\pgfpathmoveto{\pgfqpoint{2.083cm}{6.98cm}}
\pgfpathlineto{\pgfqpoint{1.829cm}{6.98cm}}
\pgfusepath{stroke}
\pgfsetdash{}{0cm}
\pgfpathmoveto{\pgfqpoint{2.083cm}{8.077cm}}
\pgfpathlineto{\pgfqpoint{1.829cm}{8.077cm}}
\pgfusepath{stroke}
\begin{pgfscope}
\pgftext[x=1.328cm,y=2.591cm,rotate=90]{0.0}
\end{pgfscope}
\begin{pgfscope}
\pgftext[x=1.328cm,y=3.687cm,rotate=90]{0.5}
\end{pgfscope}
\begin{pgfscope}
\pgftext[x=1.328cm,y=4.799cm,rotate=90]{1.0}
\end{pgfscope}
\begin{pgfscope}
\pgftext[x=1.328cm,y=5.895cm,rotate=90]{1.5}
\end{pgfscope}
\begin{pgfscope}
\pgftext[x=1.328cm,y=6.978cm,rotate=90]{2.0}
\end{pgfscope}
\begin{pgfscope}
\pgftext[x=1.328cm,y=8.074cm,rotate=90]{2.5}
\end{pgfscope}
\pgfsetdash{}{0cm}
\pgfpathmoveto{\pgfqpoint{2.083cm}{2.591cm}}
\pgfpathlineto{\pgfqpoint{9.093cm}{2.591cm}}
\pgfpathlineto{\pgfqpoint{9.093cm}{8.077cm}}
\pgfpathlineto{\pgfqpoint{2.083cm}{8.077cm}}
\pgfpathlineto{\pgfqpoint{2.083cm}{2.591cm}}
\pgfusepath{stroke}
\end{pgfscope}
\begin{pgfscope}
\pgfpathmoveto{\pgfqpoint{0cm}{0cm}}
\pgfpathlineto{\pgfqpoint{10.16cm}{0cm}}
\pgfpathlineto{\pgfqpoint{10.16cm}{10.16cm}}
\pgfpathlineto{\pgfqpoint{0cm}{10.16cm}}
\pgfpathclose
\pgfusepath{clip}
\begin{pgfscope}
\definecolor{eps2pgf_color}{gray}{0}\pgfsetstrokecolor{eps2pgf_color}\pgfsetfillcolor{eps2pgf_color}
\pgftext[x=5.599cm,y=9.067cm,rotate=0]{Example cobweb}
\end{pgfscope}
\begin{pgfscope}
\definecolor{eps2pgf_color}{gray}{0}\pgfsetstrokecolor{eps2pgf_color}\pgfsetfillcolor{eps2pgf_color}
\pgftext[x=5.588cm,y=0.771cm,rotate=0]{x}
\end{pgfscope}
\begin{pgfscope}
\definecolor{eps2pgf_color}{gray}{0}\pgfsetstrokecolor{eps2pgf_color}\pgfsetfillcolor{eps2pgf_color}
\pgftext[x=0.392cm,y=5.334cm,rotate=90]{y}
\end{pgfscope}
\end{pgfscope}
\begin{pgfscope}
\pgfpathmoveto{\pgfqpoint{2.083cm}{2.591cm}}
\pgfpathlineto{\pgfqpoint{9.093cm}{2.591cm}}
\pgfpathlineto{\pgfqpoint{9.093cm}{8.077cm}}
\pgfpathlineto{\pgfqpoint{2.083cm}{8.077cm}}
\pgfpathclose
\pgfusepath{clip}
\pgfsetdash{}{0cm}
\pgfsetlinewidth{0.265mm}
\pgfsetroundcap
\pgfsetroundjoin
\definecolor{eps2pgf_color}{rgb}{1,0,0}\pgfsetstrokecolor{eps2pgf_color}\pgfsetfillcolor{eps2pgf_color}
\pgfpathmoveto{\pgfqpoint{0.014cm}{0cm}}
\pgfpathlineto{\pgfqpoint{0.33cm}{0.396cm}}
\pgfpathlineto{\pgfqpoint{2.083cm}{2.591cm}}
\pgfpathlineto{\pgfqpoint{3.835cm}{4.785cm}}
\pgfpathlineto{\pgfqpoint{5.588cm}{6.98cm}}
\pgfpathlineto{\pgfqpoint{7.341cm}{9.174cm}}
\pgfpathlineto{\pgfqpoint{8.128cm}{10.16cm}}
\pgfusepath{stroke}
\pgfsetdash{}{0cm}
\definecolor{eps2pgf_color}{rgb}{0,0,1}\pgfsetstrokecolor{eps2pgf_color}\pgfsetfillcolor{eps2pgf_color}
\pgfpathmoveto{\pgfqpoint{2.959cm}{2.591cm}}
\pgfpathlineto{\pgfqpoint{2.959cm}{4.511cm}}
\pgfpathlineto{\pgfqpoint{3.616cm}{4.511cm}}
\pgfpathlineto{\pgfqpoint{3.616cm}{5.591cm}}
\pgfpathlineto{\pgfqpoint{4.479cm}{5.591cm}}
\pgfpathlineto{\pgfqpoint{4.479cm}{6.541cm}}
\pgfpathlineto{\pgfqpoint{5.237cm}{6.541cm}}
\pgfpathlineto{\pgfqpoint{5.237cm}{6.936cm}}
\pgfpathlineto{\pgfqpoint{5.553cm}{6.936cm}}
\pgfpathlineto{\pgfqpoint{5.553cm}{6.979cm}}
\pgfpathlineto{\pgfqpoint{5.588cm}{6.979cm}}
\pgfpathlineto{\pgfqpoint{5.588cm}{6.98cm}}
\pgfpathlineto{\pgfqpoint{5.588cm}{6.98cm}}
\pgfpathlineto{\pgfqpoint{5.588cm}{6.98cm}}
\pgfpathlineto{\pgfqpoint{5.588cm}{6.98cm}}
\pgfpathlineto{\pgfqpoint{5.588cm}{6.98cm}}
\pgfpathlineto{\pgfqpoint{5.588cm}{6.98cm}}
\pgfpathlineto{\pgfqpoint{5.588cm}{6.98cm}}
\pgfpathlineto{\pgfqpoint{5.588cm}{6.98cm}}
\pgfpathlineto{\pgfqpoint{5.588cm}{6.98cm}}
\pgfpathlineto{\pgfqpoint{5.588cm}{6.98cm}}
\pgfpathlineto{\pgfqpoint{5.588cm}{6.98cm}}
\pgfpathlineto{\pgfqpoint{5.588cm}{6.98cm}}
\pgfpathlineto{\pgfqpoint{5.588cm}{6.98cm}}
\pgfpathlineto{\pgfqpoint{5.588cm}{6.98cm}}
\pgfpathlineto{\pgfqpoint{5.588cm}{6.98cm}}
\pgfpathlineto{\pgfqpoint{5.588cm}{6.98cm}}
\pgfpathlineto{\pgfqpoint{5.588cm}{6.98cm}}
\pgfpathlineto{\pgfqpoint{5.588cm}{6.98cm}}
\pgfpathlineto{\pgfqpoint{5.588cm}{6.98cm}}
\pgfpathlineto{\pgfqpoint{5.588cm}{6.98cm}}
\pgfpathlineto{\pgfqpoint{5.588cm}{6.98cm}}
\pgfpathlineto{\pgfqpoint{5.588cm}{6.98cm}}
\pgfpathlineto{\pgfqpoint{5.588cm}{6.98cm}}
\pgfpathlineto{\pgfqpoint{5.588cm}{6.98cm}}
\pgfpathlineto{\pgfqpoint{5.588cm}{6.98cm}}
\pgfpathlineto{\pgfqpoint{5.588cm}{6.98cm}}
\pgfpathlineto{\pgfqpoint{5.588cm}{6.98cm}}
\pgfpathlineto{\pgfqpoint{5.588cm}{6.98cm}}
\pgfpathlineto{\pgfqpoint{5.588cm}{6.98cm}}
\pgfpathlineto{\pgfqpoint{5.588cm}{6.98cm}}
\pgfpathlineto{\pgfqpoint{5.588cm}{6.98cm}}
\pgfpathlineto{\pgfqpoint{5.588cm}{6.98cm}}
\pgfpathlineto{\pgfqpoint{5.588cm}{6.98cm}}
\pgfpathlineto{\pgfqpoint{5.588cm}{6.98cm}}
\pgfpathlineto{\pgfqpoint{5.588cm}{6.98cm}}
\pgfpathlineto{\pgfqpoint{5.588cm}{6.98cm}}
\pgfpathlineto{\pgfqpoint{5.588cm}{6.98cm}}
\pgfpathlineto{\pgfqpoint{5.588cm}{6.98cm}}
\pgfpathlineto{\pgfqpoint{5.588cm}{6.98cm}}
\pgfpathlineto{\pgfqpoint{5.588cm}{6.98cm}}
\pgfpathlineto{\pgfqpoint{5.588cm}{6.98cm}}
\pgfpathlineto{\pgfqpoint{5.588cm}{6.98cm}}
\pgfpathlineto{\pgfqpoint{5.588cm}{6.98cm}}
\pgfpathlineto{\pgfqpoint{5.588cm}{6.98cm}}
\pgfpathlineto{\pgfqpoint{5.588cm}{6.98cm}}
\pgfpathlineto{\pgfqpoint{5.588cm}{6.98cm}}
\pgfpathlineto{\pgfqpoint{5.588cm}{6.98cm}}
\pgfpathlineto{\pgfqpoint{5.588cm}{6.98cm}}
\pgfpathlineto{\pgfqpoint{5.588cm}{6.98cm}}
\pgfpathlineto{\pgfqpoint{5.588cm}{6.98cm}}
\pgfpathlineto{\pgfqpoint{5.588cm}{6.98cm}}
\pgfpathlineto{\pgfqpoint{5.588cm}{6.98cm}}
\pgfpathlineto{\pgfqpoint{5.588cm}{6.98cm}}
\pgfpathlineto{\pgfqpoint{5.588cm}{6.98cm}}
\pgfpathlineto{\pgfqpoint{5.588cm}{6.98cm}}
\pgfpathlineto{\pgfqpoint{5.588cm}{6.98cm}}
\pgfpathlineto{\pgfqpoint{5.588cm}{6.98cm}}
\pgfpathlineto{\pgfqpoint{5.588cm}{6.98cm}}
\pgfpathlineto{\pgfqpoint{5.588cm}{6.98cm}}
\pgfpathlineto{\pgfqpoint{5.588cm}{6.98cm}}
\pgfpathlineto{\pgfqpoint{5.588cm}{6.98cm}}
\pgfpathlineto{\pgfqpoint{5.588cm}{6.98cm}}
\pgfpathlineto{\pgfqpoint{5.588cm}{6.98cm}}
\pgfpathlineto{\pgfqpoint{5.588cm}{6.98cm}}
\pgfpathlineto{\pgfqpoint{5.588cm}{6.98cm}}
\pgfpathlineto{\pgfqpoint{5.588cm}{6.98cm}}
\pgfpathlineto{\pgfqpoint{5.588cm}{6.98cm}}
\pgfpathlineto{\pgfqpoint{5.588cm}{6.98cm}}
\pgfpathlineto{\pgfqpoint{5.588cm}{6.98cm}}
\pgfpathlineto{\pgfqpoint{5.588cm}{6.98cm}}
\pgfpathlineto{\pgfqpoint{5.588cm}{6.98cm}}
\pgfpathlineto{\pgfqpoint{5.588cm}{6.98cm}}
\pgfpathlineto{\pgfqpoint{5.588cm}{6.98cm}}
\pgfpathlineto{\pgfqpoint{5.588cm}{6.98cm}}
\pgfpathlineto{\pgfqpoint{5.588cm}{6.98cm}}
\pgfpathlineto{\pgfqpoint{5.588cm}{6.98cm}}
\pgfpathlineto{\pgfqpoint{5.588cm}{6.98cm}}
\pgfpathlineto{\pgfqpoint{5.588cm}{6.98cm}}
\pgfpathlineto{\pgfqpoint{5.588cm}{6.98cm}}
\pgfpathlineto{\pgfqpoint{5.588cm}{6.98cm}}
\pgfpathlineto{\pgfqpoint{5.588cm}{6.98cm}}
\pgfpathlineto{\pgfqpoint{5.588cm}{6.98cm}}
\pgfpathlineto{\pgfqpoint{5.588cm}{6.98cm}}
\pgfpathlineto{\pgfqpoint{5.588cm}{6.98cm}}
\pgfpathlineto{\pgfqpoint{5.588cm}{6.98cm}}
\pgfpathlineto{\pgfqpoint{5.588cm}{6.98cm}}
\pgfpathlineto{\pgfqpoint{5.588cm}{6.98cm}}
\pgfpathlineto{\pgfqpoint{5.588cm}{6.98cm}}
\pgfpathlineto{\pgfqpoint{5.588cm}{6.98cm}}
\pgfpathlineto{\pgfqpoint{5.588cm}{6.98cm}}
\pgfusepath{stroke}
\end{pgfscope}
\end{pgfscope}
\end{pgfpicture}

\endpgfgraphicnamed
\ec
\end{minipage}
\end{multicols}
\end{Rlistfloat}

\pagebreak

\part{Exercise} % (fold)
\label{prt:exercise}

\begin{enumerate}
  \item Find the equilibrium values of populations governed by the following equations, and determine their stability analytically. Is there is a stable nonzero equilibrium  (carrying capacity) that the population may approach in the long run?
  \begin{enumerate}
    \item $N_{t+1} = 41N_t -10N_t^2$
    \item $N_{t+1} = 41N_t +2N_t^2$
  \end{enumerate}
  \item Generate cobweb plots for the following logistic models.  Graphically identify all fixed points, and state whether they are ``stable.''
  \begin{enumerate}
    \item $f(x) = 3x - \frac{3x^2}{4} + 1$
    \item $f(x) = 100x -2x^2$
    \item $f(x) = -100 x + \frac{x^2}{2}$
  \end{enumerate}
  \item Write a \emph{functor} which takes as an argument a function describing a logistic model and generates a cobweb plot.
  \item In physiology, maintaining a steady level of glucose in the bloodstream is necessary for the proper functioning of all organs. To study
  this process, define $G(t)$ to be the amount of glucose in the bloodstream of a person at time $t$. Assume that glucose is absorbed from the bloodstream at a rate proportional to the concentration $G(t)$, with rate parameter $k$.
  \begin{enumerate}
    \item Write down a differential equation to describe this situation. What kind of ODE is it?
    \item Find the analytical solution for this equation, in terms of an initial value $G_0$ and the rate parameter $k$.
    \item Let the initial glucose concentration be $G_0 = 100 mg/dl$, and the glucose removal rate be $k = 0.01 / min$. Use R to plot the solution as a graph over a reasonable time interval, with properly labeled axes.
    \item What is the equilibrium concentration of blood sugar in this model? Is the equilibrium stable or unstable?
    \end{enumerate}
  \item Now let us assume that glucose is added to to bloodstream at a constant rate $a$, independent of glucose concentration.
  \begin{enumerate}
    \item Write down a differential equation to describe this situation. What kind of ODE is it?
    \item Find the analytical solution for this equation, in terms of an initial value $G_0$ and the parameters $k$ and $a$.
    \item Let the initial glucose concentration be $G_0 = 100 mg/dl$, the glucose removal rate be $k = 0.01 / min$, and $a = 4 mg/dl/min$. Use R to plot the solution as a graph over a reasonable time interval, with properly labeled axes.
    \item What is the equilibrium concentration of blood sugar in the model with glucose infusion? Is the equilibrium stable or unstable?
  \end{enumerate}
%  \item \emph{Analytically} find all fixed points for the following logistic models.  You may use a pencil and paper to solve these problems (hand in for credit).


%   \item \emph{Enzyme kinetics.}  Consider the following reaction and observations.
%   \begin{center}
%     \ce{A + 2B <=> AB2}
%   \end{center}
%   \bc
%     \begin{tabular}{c c c c c c c c}
%     \toprule
%     & & \multicolumn{6}{c}{Concentration (mM)} \\
%     \multicolumn{2}{c}{\hspace{0.8cm}Time (ms)}\hspace{1.6cm} & \multicolumn{3}{c}{Before} & \multicolumn{3}{c}{After} \\
%     & & [\ce{A}] & [\ce{B}] & [\ce{AB2}] & [\ce{A}] & [\ce{B}] & [\ce{AB2}] \\
%     \midrule
%     1.177 & 1.253 & 1.021 & 1.011 & 0.014 & 1.012 & 0.992 & 0.023 \\ 
%     2.662 & 2.820 & 1.012 & 2.015 & 0.006 & 0.943 & 1.879 & 0.074 \\ 
%     3.184 & 3.670 & 1.023 & 3.015 & 0.018 & 0.918 & 2.805 & 0.123 \\ 
%     4.495 & 4.586 & 2.040 & 1.003 & 0.016 & 1.937 & 0.797 & 0.119 \\ 
%     5.492 & 5.827 & 2.019 & 2.018 & 0.012 & 1.679 & 1.339 & 0.351 \\ 
%     6.678 & 6.769 & 2.026 & 3.009 & 0.016 & 1.608 & 2.174 & 0.434 \\ 
%     7.221 & 7.805 & 0.021 & 1.008 & 1.005 & 0.061 & 1.089 & 0.965 \\ 
%     8.220 & 8.256 & 0.014 & 2.016 & 2.018 & 0.040 & 2.068 & 1.992 \\ 
%     9.260 & 9.802 & 0.032 & 3.014 & 3.002 & 0.071 & 3.092 & 2.962 \\ 
%     10.380 & 10.698 & 0.015 & 1.018 & 4.013 & 0.091 & 1.170 & 3.937 \\ 
%     \bottomrule
%   \end{tabular}
%   \ec
%   \begin{enumerate}
%     \item Calculate the rate law for the forward and reverse reactions. (\emph{Hint: the file c3e4.R on the website contains this raw data}).
%     \item Calculate $K_1$ and $K_{-1}$, the forward and backwards rate coefficients.
%   \end{enumerate}
%   \item \emph{Goldman equation and Action potentials}.  The \emph{Nernst potential} of a single ion \ce{A^z} with charge $z$ across a cell membrane is given by
%   \begin{align}
%     E_{\ce{A}} &= \frac{RT}{zF} \log \frac{[\ce{A^z}_o]}{[\ce{A^z}_i]} \\
%     R &= 8.314 J K^{-1} mol^{-1} \text{(Universal gas constant)} \\
%     T &= \text{Temperature in kelvins.  Assume $T=298$} \\
%     F &= 9.648 \times 10^4 C mol^{-1} \text{(Faraday constant; i.e. charge per mole of electrons)} \\
%     [\ce{A^z_o}] &= \text{Extracellular concentration} \\
%     [\ce{A^z_i}] &= \text{Intracellular concentration}
%   \end{align}
%   Thus for sodium (\ce{NA+}), the Nernst potential, given $[\ce{NA+}_o] = 100$ mM and $[\ce{NA+}_i] = 10$ mM is
%   \begin{align*}
%     E_{\ce{NA+}} &= \frac{RT}{zF} \log \frac{[\ce{NA+}_o]}{[\ce{NA+}_i]} \\
%     &= \frac{8.314 J K^{-1} mol^{-1} \cdot 298 K}{1 \cdot 9.68 \times 10^4 C mol^{-1}} * \log{\frac{100}{10}} \\
%     &= 25.3 mV \log 10 \\
%     &= 58 mV
%   \end{align*}
%   Furthermore, the total potential across a membrane by combining the products of the individual ionic Nernst potentials and ionic conductances.
%   \begin{equation}
%     E_{m} = \frac{RT}{F} \ln{ \left( \frac{ \sum_{i}^{N} P_{M^{+}_{i}}[M^{+}_{i}]_\text{out} + \sum_{j}^{M} P_{A^{-}_{j}}[A^{-}_{j}]_\text{in}}{ \sum_{i}^{N} P_{M^{+}_{i}}[M^{+}_{i}]_\text{in} + \sum_{j}^{M} P_{A^{-}_{j}}[A^{-}_{j}]_\text{out}} \right) }
%     \end{equation} 
  \item \emph{Not required.  A good exercise which relates to numerical approximation.} \begin{enumerate}
    \item \emph{Binary Search.} Write a function which, given a sorted vector $X$ of $n$ unique integers, and an integer $k$, returns the index of $k$ in vector $X$ if $k \in X$ and $-1$ otherwise.  Your solution should perform approximately $\log n$ comparisons in the ``worst case'' scenario. \emph{Hint:}  one example of a ``worst case'' is when the $X_0 = k$.
    \item Modify the previous solution so that the requirement of \emph{unique} integers may be relaxed.  In the case that multiple solutions exist return a vector of all such solutions.
  \end{enumerate}
\end{enumerate}
% part exercise (end)


\end{document}
