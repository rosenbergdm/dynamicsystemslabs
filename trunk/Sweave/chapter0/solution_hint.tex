%
%  untitled
%
%  Created by David Rosenberg on 2009-10-04.
%  Copyright (c) 2009 University of Chicago. All rights reserved.
%
\documentclass[]{article}

% Use utf-8 encoding for foreign characters
\usepackage[utf8]{inputenc}

% Setup for fullpage use
\usepackage{fullpage}

% Uncomment some of the following if you use the features
%
% Running Headers and footers
%\usepackage{fancyhdr}

% Multipart figures
%\usepackage{subfigure}

% More symbols
%\usepackage{amsmath}
%\usepackage{amssymb}
%\usepackage{latexsym}

% Surround parts of graphics with box
\usepackage{boxedminipage}

% Package for including code in the document
\usepackage{listings}

% If you want to generate a toc for each chapter (use with book)
\usepackage{minitoc}
\usepackage{amsmath,amsthm,amssymb}

% This is now the recommended way for checking for PDFLaTeX:
\usepackage{ifpdf}

%\newif\ifpdf
%\ifx\pdfoutput\undefined
%\pdffalse % we are not running PDFLaTeX
%\else
%\pdfoutput=1 % we are running PDFLaTeX
%\pdftrue
%\fi

\ifpdf
\usepackage[pdftex]{graphicx}
\else
\usepackage{graphicx}
\fi
\title{A LaTeX Article}
\author{  }

\date{2009-10-04}

\begin{document}

\ifpdf
\DeclareGraphicsExtensions{.pdf, .jpg, .tif}
\else
\DeclareGraphicsExtensions{.eps, .jpg}
\fi

Hint for exercise 3 on the homework.

\begin{displaymath}
\begin{aligned}
  (a_k) &= a_1, a_2, \ldots \\
  \text{ and } a_k &= b_k + c_k \, i,\quad b_k, c_k \in \mathbb{Q}\\
  &= \sqrt{b_k^2 + c_k^2} e^{i\ \left( \arctan \left( \dfrac{b_k}{c_k} \right) \right) } \\
  \text{Let } r_k &= \sqrt{b_k^2 + c_k^2} \text{ and } \psi_k = \arctan \left( \dfrac{b_k}{c_k} \right) \\
  \text{Then } a_k &= r_k e^{\psi_k i} \\
  a_k^n &= \left(r_k e^{\psi_k i}\right) ^n \\
  &=r_k {}^n e^{n i \psi_k} \\
\end{aligned}
\end{displaymath}
Since $x^n - 1, r_k {}^n = 1$ and $e^{n i \psi_k} = 0 \pm 2c \pi$, where $c \in \mathbb{Z}$, we know that \( r_k {}^n = 1 \) and further, \( r_k = 1 \).  We also know that \( 0 \leq \psi_k \leq 2\pi \).  The first two solutions, then, are given by
\begin{displaymath}
\begin{aligned}
	n \psi_k &= 2 \pi \quad &n \psi_k &= 0 \\
	\psi_k &= \dfrac{2 \pi}{n} \quad &\psi_k &= 0 \\
\end{aligned}
\end{displaymath}
Furthermore, the set of all solutions is the set of linear combinations of these two solutions, modulo $2 \pi$.

\begin{displaymath}
\begin{aligned}
	\psi_0 &= 0 \\
	\psi_1 &= \dfrac{2 \pi}{n} \\
	\psi_2 &= 2 \left( \dfrac{2 \pi}{n} \right) \\
	\vdots \\
	\psi_n &= n \left( \dfrac{2 \pi}{n} \right) = 2 \pi \equiv 0 \\
\end{aligned}
\end{displaymath}
Therefore \( \forall k \in \mathbb{Z}: 0 \leq k < n \), the complex number \( e^{\frac{2 k i \pi}{n}} \) is a solution for \( x^k - 1 = 0 \).

\end{document}
