%%%%%%%%%%%%%%%%%%%%%%%%%%%%%%%%%%%%%%%%%%%%%%%%%%%%%%%%%%%%%%%%%%%%%%%%%%%%%%%
%
% Vignette  "How to work with SweaveListingUtils"
%
%
%%%%%%%%%%%%%%%%%%%%%%%%%%%%%%%%%%%%%%%%%%%%%%%%%%%%%%%%%%%%%%%%%%%%%%%%%%%%%%%
%
%\VignetteIndexEntry{ExampleSweaveListingUtils}
%\VignetteDepends{startupmsg}
%\VignetteKeywords{Sweave,TeX package listings, listings, R-forge}
%\VignettePackage{SweaveListingUtils}
%
\documentclass[10pt]{article}
%
%use svn-multi to fill in revision information
%
\usepackage{svn-multi}
% Version control information:
\svnidlong
{$HeadURL: file:///srv/svn/distr/pkg/SweaveListingUtils/inst/doc/ExampleSweaveListingUtils.Rnw $}
{$LastChangedDate: 2009-09-04 01:07:35 +0200 (Fri, 04 Sep 2009) $}
{$LastChangedRevision: 562 $}
{$LastChangedBy: ruckdeschel $}
%\svnid{$Id: example_main.tex 146 2008-12-03 13:29:19Z martin $}
% Don't forget to set the svn property 'svn:keywords' to
% 'HeadURL LastChangedDate LastChangedRevision LastChangedBy' or
% 'Id' or both depending if you use \svnidlong and/or \svnid
%
\newcommand{\svnfooter}{Last Changed Rev: \svnkw{LastChangedRevision}}
\svnRegisterAuthor{ruckdeschel}{Peter Ruckdeschel}
\svnRegisterAuthor{stamats}{Matthias Kohl}
\svnRegisterAuthor{florian}{Florian Camphausen}
\svnRegisterAuthor{stabla}{Thomas Stabla}
\svnRegisterAuthor{anhuel}{Anja H{\"u}ller}
\svnRegisterAuthor{ifrin}{Eleonara Feist}
\svnRegisterAuthor{jdospina}{Juan David Ospina}
\svnRegisterAuthor{kowzar}{Kouros Owzar}
%
\usepackage{geometry}
%
\usepackage{color}
\definecolor{darkblue}{rgb}{0.0,0.0,0.75}
\definecolor{distrCol}{rgb}{0.0,0.4,0.4}
%
\usepackage{ifpdf}
%
\usepackage{amssymb}
\usepackage{url}
%
\usepackage[authoryear,round,longnamesfirst]{natbib}
%
\usepackage[%
baseurl={r-forge.r-project.org},%
pdftitle={How to work with SweaveListingUtils},%
pdfauthor={Peter Ruckdeschel},%
pdfsubject={SweaveListingUtils},%
pdfkeywords={Sweave,TeX package listings, listings, R-forge},%
pagebackref,bookmarks,colorlinks,linkcolor=darkblue,citecolor=darkblue,%
pagecolor=darkblue,raiselinks,plainpages,pdftex]{hyperref}
%
\markboth{\sl How to work with SweaveListingUtils}%
{\sl How to work with SweaveListingUtils}
%
% -------------------------------------------------------------------------------
\newcommand{\Reals}{\mathbb{R}}
\newcommand{\R}{\mathbb{R}}
\newcommand{\N}{\mathbb{N}}
\newcommand{\hreft}[1]{\href{#1}{\tt\small\url{#1}}}
\definecolor{gray90}{rgb}{0.900,0.900,0.900}
\definecolor{gray95}{rgb}{0.950,0.950,0.950}
%
% -------------------------------------------------------------------------------
\RequirePackage{fancyvrb}
\RequirePackage{listings}
%\usepackage{Sweave}

% -------------------------------------------------------------------------------
%------------------------------------------------------------------------------%
%Preparations for Sweave and Listings
%------------------------------------------------------------------------------%
%
\RequirePackage{color}
\definecolor{Rcolor}{rgb}{0, 0.5, 0.5}
\definecolor{RRecomdcolor}{rgb}{0, 0.6, 0.4}
\definecolor{Rbcolor}{rgb}{0, 0.6, 0.6}
\definecolor{Routcolor}{rgb}{0.461, 0.039, 0.102}
\definecolor{Rcommentcolor}{rgb}{0.101, 0.043, 0.432}
%------------------------------------------------------------------------------%
\lstdefinelanguage{Rd}[common]{TeX}%
{moretexcs={acronym,alias,arguments,author,bold,cite,%
          code,command,concept,cr,deqn,describe,%
          description,details,dfn,doctype,dots,%
          dontrun,dontshow,donttest,dQuote,%
          email,emph,enc,encoding,enumerate,env,eqn,%
          examples,file,format,item,itemize,kbd,keyword,%
          ldots,link,linkS4class,method,name,note,%
          option,pkg,preformatted,R,Rdopts,Rdversion,%
          references,S3method,S4method,Sexpr,samp,section,%
          seealso,source,sp,special,%
          sQuote,strong,synopsis,tab,tabular,testonly,%
          title,url,usage,value,var,verb},
   sensitive=true,%
   morecomment=[l]\%% 2008/9 Peter Ruckdeschel
}[keywords,comments]%%
%------------------------------------------------------------------------------%

%----------------
\lstdefinestyle{RstyleO1}{fancyvrb=true,escapechar=`,extendedchars=false,%
                          language=R,%
                          basicstyle={\color{Rcolor}\small},%
                          keywordstyle={\bf\color{Rcolor}},%
                          commentstyle={\color{Rcommentcolor}\ttfamily\itshape},%
                          literate={<-}{{$\leftarrow$}}2{<<-}{{$\twoheadleftarrow$}}2{~}{{$\sim$}}1{<=}{{$\leq$}}2{>=}{{$\geq$}}2{^}{{$\scriptstyle\wedge$}}1,%
                          alsoother={$},%
                          alsoletter={.<-},%
                          otherkeywords={!,!=,~,$,*,\&,\%/\%,\%*\%,\%\%,<-,<<-,/},%
                          escapeinside={(*}{*)}}%
%----------------
\lstdefinestyle{Rstyle}{style=RstyleO1}

%----------------
\lstdefinestyle{Rdstyle}{fancyvrb=true,language=Rd,keywordstyle={\bf},%
                         basicstyle={\color{black}\footnotesize},%
                         commentstyle={\ttfamily\itshape},%
                         alsolanguage=R}%
%----------------
%------------------------------------------------------------------------------%
\global\def\Rlstset{\lstset{style=Rstyle}}%
\global\def\Rdlstset{\lstset{style=Rdstyle}}%
%------------------------------------------------------------------------------%
\global\def\Rinlstset{\lstset{style=Rinstyle}}%
\global\def\Routlstset{\lstset{style=Routstyle}}%
\global\def\Rcodelstset{\lstset{style=Rcodestyle}}%
%------------------------------------------------------------------------------%
\Rlstset
%------------------------------------------------------------------------------%
%copying relevant parts of Sweave.sty
%------------------------------------------------------------------------------%
%
\RequirePackage{graphicx,fancyvrb}%
\IfFileExists{upquote.sty}{\RequirePackage{upquote}}{}%

\RequirePackage{ifthen}%
\newboolean{Sweave@gin}%
\setboolean{Sweave@gin}{true}%
\setkeys{Gin}{width=0.8\textwidth}%
\newboolean{Sweave@ae}
\setboolean{Sweave@ae}{true}%
\RequirePackage[T1]{fontenc}
\RequirePackage{ae}
%
\newenvironment{Schunk}{}{}

\newcommand{\Sconcordance}[1]{% 
\ifx\pdfoutput\undefined% 
\csname newcount\endcsname\pdfoutput\fi% 
\ifcase\pdfoutput\special{#1}% 
\else\immediate\pdfobj{#1}\fi} 

%------------------------------------------------------------------------------%
% ---- end of parts of Sweave.sty
%------------------------------------------------------------------------------%
%
% ---- input 
\lstdefinestyle{RinstyleO}{style=Rstyle,fancyvrb=true,%
                           basicstyle=\color{Rcolor}\small}%
\lstdefinestyle{Rinstyle}{style=RinstyleO}
\lstnewenvironment{Sinput}{\Rinlstset}{\Rlstset}
%
% ---- output 
\lstdefinestyle{RoutstyleO}{
fancyvrb=false,basicstyle=\color{Routcolor}\small}%
\lstdefinestyle{Routstyle}{style=RoutstyleO}
\lstnewenvironment{Soutput}{\Routlstset}{\Rlstset}
%
% ---- code 
\lstdefinestyle{RcodestyleO}{style=Rstyle,fancyvrb=true,fontshape=sl,%
                             basicstyle=\color{Rcolor}}%
\lstdefinestyle{Rcodestyle}{style=RcodestyleO}
\lstnewenvironment{Scode}{\Rcodelstset}{\Rlstset}
%
%------------------------------------------------------------------------------%
\let\code\lstinline
\def\Code#1{{\tt\color{Rcolor} #1}}
\def\file#1{{\tt #1}} 
\def\pkg#1{{\tt "#1"}} 
\newcommand{\pkgversion}{{\tt 2.1.3}}
%------------------------------------------------------------------------------%

\lstdefinestyle{RstyleO2}{style=RstyleO1,%
% --------------------------
% Registration of package SweaveListingUtils
% --------------------------
morekeywords={[2]taglist,SweaveListingPreparations,SweaveListingOptions,SweaveListingoptions,SweaveListingMASK,%
setToBeDefinedPkgs,setBaseOrRecommended,readSourceFromRForge,readPkgVersion,lstsetRout,%
lstsetRin,lstsetRd,lstsetRcode,lstsetRall,lstsetR,%
lstsetLanguage,lstset,lstinputSourceFromRForge,lstdefRstyle,isBaseOrRecommended,%
getSweaveListingOption,copySourceFromRForge,changeKeywordstyles%
},%
keywordstyle={[2]{\bf}},%
%
% --------------------------
% Registration of package startupmsg
% --------------------------
morekeywords={[3]suppressStartupMessages,startupType,startupPackage,StartupMessage,startupMessage,%
startupEndline,readVersionInformation,readURLInformation,pointertoNEWS,onlytypeStartupMessages,%
NEWS,mystartupMessage,mySMHandler,infoShow,buildStartupMessage%
},%
keywordstyle={[3]{\bf}},%
%
% --------------------------
% Registration of package stats [recommended or base] 
% --------------------------
morekeywords={[4]xtabs,write.ftable,window<-,wilcox.test,weighted.residuals,%
weighted.mean,vcov,varimax,variable.names,var.test,%
update.formula,update.default,TukeyHSD.aov,TukeyHSD,tsSmooth,%
tsp<-,tsdiag,ts.union,ts.plot,ts.intersect,%
toeplitz,terms.terms,terms.formula,terms.default,terms.aovlist,%
termplot,t.test,supsmu,summary.stepfun,summary.mlm,%
summary.manova,summary.lm,summary.infl,summary.glm,summary.aovlist,%
summary.aov,StructTS,stl,stepfun,stat.anova,%
SSweibull,SSmicmen,SSlogis,SSgompertz,SSfpl,%
SSfol,SSD,SSbiexp,SSasympOrig,SSasympOff,%
SSasymp,splinefunH,spectrum,spec.taper,spec.pgram,%
spec.ar,sortedXyData,smoothEnds,smooth.spline,smooth,%
simulate,shapiro.test,setNames,selfStart,se.contrast,%
screeplot,scatter.smooth,runmed,rstudent.lm,rstudent.glm,%
rstandard.lm,rstandard.glm,rmultinom,residuals.lm,residuals.glm,%
residuals.default,reshapeWide,reshapeLong,reshape,reorder,%
rect.hclust,read.ftable,r2dtable,quasipoisson,quasibinomial,%
quantile.default,quade.test,qqnorm.default,qbirthday,prop.trend.test,%
prop.test,promax,printCoefmat,print.ts,print.terms,%
print.logLik,print.lm,print.integrate,print.infl,print.glm,%
print.ftable,print.formula,print.family,print.density,print.coefmat,%
print.anova,princomp,predict.poly,predict.mlm,predict.lm,%
predict.glm,prcomp,ppr,PP.test,power.t.test,%
power.prop.test,power.anova.test,polym,poisson.test,plot.TukeyHSD,%
plot.ts,plot.stepfun,plot.spec.phase,plot.spec.coherency,plot.spec,%
plot.mlm,plot.lm,plot.ecdf,plot.density,plclust,%
pbirthday,pairwise.wilcox.test,pairwise.table,pairwise.t.test,pairwise.prop.test,%
pacf,p.adjust.methods,p.adjust,order.dendrogram,oneway.test,%
numericDeriv,NLSstRtAsymptote,NLSstLfAsymptote,NLSstClosestX,NLSstAsymptotic,%
nls.control,nls,nlminb,naresid,naprint,%
napredict,na.pass,na.omit,na.fail,na.exclude,%
na.contiguous,na.action,mood.test,monthplot,model.weights,%
model.tables,model.response,model.offset,model.matrix.lm,model.matrix.default,%
model.matrix,model.frame.lm,model.frame.glm,model.frame.default,model.frame.aovlist,%
model.frame,model.extract,medpolish,median.default,mcnemar.test,%
mauchly.test,mauchley.test,mantelhaen.test,manova,makepredictcall,%
makeARIMA,make.link,ls.print,ls.diag,logLik,%
loess.smooth,loess.control,loess,loadings,lm.wfit.null,%
lm.wfit,lm.influence,lm.fit.null,lm.fit,lines.ts,%
line,lag.plot,lag,ksmooth,ks.test,%
kruskal.test,knots,kmeans,kernel,kernapply,%
KalmanSmooth,KalmanRun,KalmanLike,KalmanForecast,isoreg,%
is.tskernel,is.ts,is.stepfun,is.mts,is.leaf,%
is.empty.model,inverse.gaussian,interaction.plot,integrate,influence.measures,%
HoltWinters,heatmap,hclust,hatvalues.lm,hatvalues,%
glm.fit.null,glm.fit,glm.control,getInitial,get_all_vars,%
friedman.test,fligner.test,fitted.values,fisher.test,filter,%
factor.scope,factanal,expand.model.frame,estVar,embed,%
eff.aovlist,ecdf,dummy.coef,drop.terms,drop.scope,%
dmultinom,dist,diffinv,diff.ts,dfbeta,%
df.residual,df.kernel,deriv3.formula,deriv3.default,deriv3,%
deriv.formula,deriv.default,density.default,dendrapply,delete.response,%
decompose,cutree,cpgram,cov2cor,cov.wt,%
cor.test,cophenetic,cooks.distance,contrasts<-,contr.treatment,%
contr.sum,contr.SAS,contr.poly,contr.helmert,constrOptim,%
confint.default,confint,complete.cases,cmdscale,clearNames,%
chisq.test,ccf,case.names,cancor,bw.ucv,%
bw.SJ,bw.nrd0,bw.nrd,bw.bcv,Box.test,%
biplot,binom.test,bartlett.test,bandwidth.kernel,asOneSidedFormula,%
as.ts,as.stepfun,as.hclust,as.formula,as.dist,%
as.dendrogram,ARMAtoMA,ARMAacf,arima0.diag,arima0,%
arima.sim,arima,ar.yw,ar.ols,ar.mle,%
ar.burg,ar,ansari.test,anovalist.lm,anova.mlm,%
anova.lmlist,anova.lm,anova.glmlist,anova.glm,AIC,%
aggregate.ts,aggregate.default,aggregate.data.frame,addmargins,add.scope,%
acf2AR,acf%
},%
keywordstyle={[4]{\bf\color{RRecomdcolor}}},%
%
% --------------------------
% Registration of package graphics [recommended or base] 
% --------------------------
morekeywords={[5]xspline,text.default,stripchart,strheight,split.screen,%
spineplot,smoothScatter,points.default,plot.xy,plot.window,%
plot.new,plot.design,plot.default,pie,panel.smooth,%
pairs.default,lines.default,layout.show,image.default,hist.default,%
grconvertY,grconvertX,fourfoldplot,filled.contour,erase.screen,%
dotchart,contour.default,co.intervals,close.screen,clip,%
cdplot,boxplot.matrix,boxplot.default,barplot.default,axTicks,%
axis.POSIXct,axis.Date,Axis,assocplot%
},%
keywordstyle={[5]{\bf\color{RRecomdcolor}}},%
%
% --------------------------
% Registration of package grDevices [recommended or base] 
% --------------------------
morekeywords={[6]xyz.coords,xyTable,xy.coords,xfig,X11Fonts,%
X11Font,X11.options,Type1Font,trans3d,topo.colors,%
tiff,terrain.colors,svg,setPS,setEPS,%
savePlot,rgb2hsv,replayPlot,recordPlot,recordGraphics,%
quartzFonts,quartzFont,quartz.options,quartz,ps.options,%
postscriptFonts,postscriptFont,png,pdfFonts,pdf.options,%
pdf,nclass.Sturges,nclass.scott,nclass.FD,n2mfrow,%
make.rgb,jpeg,Hershey,heat.colors,hcl,%
grey.colors,gray.colors,graphics.off,getGraphicsEvent,extendrange,%
embedFonts,deviceIsInteractive,devAskNewPage,dev.size,dev.set,%
dev.print,dev.prev,dev.off,dev.next,dev.new,%
dev.list,dev.interactive,dev.cur,dev.copy2pdf,dev.copy2eps,%
dev.copy,dev.control,densCols,convertColor,contourLines,%
colorspaces,colorRampPalette,colorRamp,colorConverter,col2rgb,%
cm.colors,CIDFont,check.options,cairo_ps,cairo_pdf,%
boxplot.stats,bmp,blues9,bitmap,as.graphicsAnnot%
},%
keywordstyle={[6]{\bf\color{RRecomdcolor}}},%
%
% --------------------------
% Registration of package utils [recommended or base] 
% --------------------------
morekeywords={[7]zip.file.extract,wsbrowser,write.table,write.socket,write.csv2,%
write.csv,vignette,View,URLencode,URLdecode,%
url.show,upgrade,update.packageStatus,update.packages,unzip,%
unstack,type.convert,txtProgressBar,toLatex,toBibtex,%
timestamp,tail.matrix,tail,SweaveSyntConv,SweaveSyntaxNoweb,%
SweaveSyntaxLatex,SweaveHooks,Sweave,summaryRprof,strOptions,%
str,Stangle,stack,setTxtProgressBar,setRepositories,%
sessionInfo,select.list,savehistory,RweaveTryStop,RweaveLatexWritedoc,%
RweaveLatexSetup,RweaveLatexOptions,RweaveLatexFinish,RweaveLatex,RweaveEvalWithOpt,%
RweaveChunkPrefix,RtangleWritedoc,RtangleSetup,Rtangle,rtags,%
RSiteSearch,RShowDoc,Rprofmem,Rprof,remove.packages,%
relist,recover,readCitationFile,read.table,read.socket,%
read.fwf,read.fortran,read.DIF,read.delim2,read.delim,%
read.csv2,read.csv,rc.status,rc.settings,rc.options,%
rc.getOption,promptPackage,promptData,personList,person,%
packageStatus,packageDescription,package.skeleton,package.contents,old.packages,%
object.size,nsl,normalizePath,new.packages,modifyList,%
mirror2html,memory.size,memory.limit,makeRweaveLatexCodeRunner,make.socket,%
make.packages.html,lsf.str,ls.str,localeToCharset,loadhistory,%
limitedLabels,is.relistable,installed.packages,install.packages,index.search,%
history,help.start,help.search,help.request,head.matrix,%
head,glob2rx,getTxtProgressBar,getS3method,getFromNamespace,%
getCRANmirrors,getAnywhere,formatUL,formatOL,flush.console,%
fixInNamespace,file.edit,file_test,dump.frames,download.packages,%
download.file,de.setup,de.restore,de.ncols,data.entry,%
CRAN.packages,count.fields,contrib.url,compareVersion,combn,%
close.socket,citHeader,citFooter,citEntry,citation,%
chooseCRANmirror,checkCRAN,capture.output,bug.report,browseVignettes,%
browseURL,browseEnv,available.packages,assignInNamespace,as.roman,%
as.relistable,as.personList,as.person,argsAnywhere,alarm%
},%
keywordstyle={[7]{\bf\color{RRecomdcolor}}},%
%
% --------------------------
% Registration of package datasets [recommended or base] 
% --------------------------
morekeywords={[8]WWWusage,WorldPhones,women,warpbreaks,volcano,%
VADeaths,uspop,USPersonalExpenditure,USJudgeRatings,USArrests,%
USAccDeaths,UKgas,UKDriverDeaths,UCBAdmissions,trees,%
treering,ToothGrowth,Titanic,Theoph,swiss,%
sunspots,sunspot.year,sunspot.month,state.x77,state.region,%
state.name,state.division,state.center,state.area,state.abb,%
stackloss,stack.x,stack.loss,sleep,Seatbelts,%
rock,rivers,randu,quakes,Puromycin,%
pressure,presidents,precip,PlantGrowth,OrchardSprays,%
Orange,occupationalStatus,nottem,Nile,nhtemp,%
mtcars,morley,mdeaths,lynx,longley,%
Loblolly,LifeCycleSavings,lh,ldeaths,LakeHuron,%
JohnsonJohnson,islands,iris3,iris,InsectSprays,%
infert,Indometh,Harman74.cor,Harman23.cor,HairEyeColor,%
freeny.y,freeny.x,freeny,Formaldehyde,fdeaths,%
faithful,EuStockMarkets,eurodist,euro.cross,euro,%
esoph,DNase,discoveries,crimtab,CO2,%
co2,chickwts,ChickWeight,cars,BOD,%
BJsales.lead,BJsales,beaver2,beaver1,austres,%
attitude,attenu,anscombe,airquality,AirPassengers,%
airmiles,ability.cov%
},%
keywordstyle={[8]{\bf\color{RRecomdcolor}}},%
%
% --------------------------
% Registration of package methods [recommended or base] 
% --------------------------
morekeywords={[9]validSlotNames,validObject,unRematchDefinition,trySilent,tryNew,%
traceOn,traceOff,testVirtual,testInheritedMethods,superClassDepth,%
Summary,substituteFunctionArgs,substituteDirect,slotsFromS3,slotNames,%
slot<-,slot,sigToEnv,SignatureMethod,signature,%
showMlist,showMethods,showExtends,showDefault,showClass,%
setValidity,setReplaceMethod,setPrimitiveMethods,setPackageName,setOldClass,%
setMethod,setIs,setGroupGeneric,setGenericImplicit,setGeneric,%
setDataPart,setClassUnion,setClass,setAs,sessionData,%
selectSuperClasses,selectMethod,seemsS4Object,sealClass,S3Part<-,%
S3Part,S3Class<-,S3Class,resetGeneric,resetClass,%
requireMethods,representation,removeMethodsObject,removeMethods,removeMethod,%
removeGeneric,removeClass,rematchDefinition,registerImplicitGenerics,reconcilePropertiesAndPrototype,%
rbind2,Quote,prototype,promptMethods,promptClass,%
prohibitGeneric,possibleExtends,packageSlot<-,packageSlot,newEmptyObject,%
newClassRepresentation,newBasic,mlistMetaName,missingArg,methodsPackageMetaName,%
MethodsListSelect,MethodsList,methodSignatureMatrix,MethodAddCoerce,method.skeleton,%
metaNameUndo,mergeMethods,Math2,matchSignature,makeStandardGeneric,%
makePrototypeFromClassDef,makeMethodsList,makeGeneric,makeExtends,makeClassRepresentation,%
Logic,loadMethod,listFromMlist,listFromMethods,linearizeMlist,%
languageEl<-,languageEl,isXS3Class,isVirtualClass,isSealedMethod,%
isSealedClass,isGroup,isGrammarSymbol,isGeneric,isClassUnion,%
isClassDef,isClass,insertMethod,initialize,implicitGeneric,%
hasMethods,hasMethod,hasArg,getVirtual,getValidity,%
getSubclasses,getSlots,getPrototype,getProperties,getPackageName,%
getMethodsMetaData,getMethodsForDispatch,getMethods,getMethod,getGroupMembers,%
getGroup,getGenerics,getGeneric,getFunction,getExtends,%
getDataPart,getClassPackage,getClassName,getClasses,getClassDef,%
getClass,getAllSuperClasses,getAllMethods,getAccess,generic.skeleton,%
functionBody<-,functionBody,formalArgs,fixPre1.8,findUnique,%
findMethodSignatures,findMethods,findMethod,findFunction,findClass,%
finalDefaultMethod,extends,existsMethod,existsFunction,emptyMethodsList,%
empty.dump,elNamed<-,elNamed,el<-,el,%
dumpMethods,dumpMethod,doPrimitiveMethod,defaultPrototype,defaultDumpName,%
conformMethod,Complex,completeSubclasses,completeExtends,completeClassDefinition,%
Compare,coerce<-,coerce,classMetaName,classesToAM,%
checkSlotAssignment,cbind2,canCoerce,callNextMethod,callGeneric,%
cacheMethod,cacheMetaData,cacheGenericsMetaData,body<-,balanceMethodsList,%
assignMethodsMetaData,assignClassDef,asMethodDefinition,as<-,Arith,%
allNames,allGenerics,addNextMethod%
},%
keywordstyle={[9]{\bf\color{RRecomdcolor}}},%
%
% --------------------------
% Registration of package base [recommended or base] 
% --------------------------
morekeywords={[10]xtfrm.Surv,xtfrm.POSIXlt,xtfrm.POSIXct,xtfrm.numeric_version,xtfrm.factor,%
xtfrm.default,xtfrm.Date,xtfrm,xpdrows.data.frame,writeLines,%
writeChar,writeBin,write.table0,write.dcf,withVisible,%
withRestarts,within.list,within.data.frame,within,withCallingHandlers,%
with.default,with,which.min,which.max,weekdays.POSIXt,%
weekdays.Date,weekdays,version,Vectorize,utf8ToInt,%
upper.tri,unz,untracemem,unsplit,unserialize,%
unlockBinding,unloadNamespace,unix.time,units<-.difftime,units<-,%
units.difftime,units,unique.POSIXlt,unique.numeric_version,unique.matrix,%
unique.default,unique.data.frame,unique.array,tryCatch,truncate.connection,%
truncate,trunc.POSIXt,trunc.Date,transform.default,transform.data.frame,%
tracingState,tracemem,toupper,toString.default,toString,%
topenv,tolower,textConnectionValue,textConnection,testPlatformEquivalence,%
tempdir,tcrossprod,taskCallbackManager,t.default,t.data.frame,%
T,system.time,system.file,Sys.which,Sys.unsetenv,%
Sys.umask,Sys.timezone,Sys.time,sys.status,sys.source,%
Sys.sleep,Sys.setlocale,Sys.setenv,sys.save.image,Sys.putenv,%
sys.parents,sys.parent,sys.on.exit,sys.nframe,Sys.localeconv,%
sys.load.image,Sys.info,Sys.glob,Sys.getpid,Sys.getlocale,%
Sys.getenv,sys.function,sys.frames,sys.frame,Sys.Date,%
Sys.chmod,sys.calls,sys.call,symbol.For,symbol.C,%
suppressWarnings,suppressPackageStartupMessages,suppressMessages,summary.table,Summary.POSIXlt,%
summary.POSIXlt,Summary.POSIXct,summary.POSIXct,Summary.numeric_version,summary.matrix,%
Summary.factor,summary.factor,Summary.difftime,summary.default,Summary.Date,%
summary.Date,Summary.data.frame,summary.data.frame,summary.connection,substring<-,%
substr<-,subset.matrix,subset.default,subset.data.frame,strwrap,%
strtrim,strptime,strftime,storage.mode<-,storage.mode,%
stopifnot,stdout,stdin,stderr,standardGeneric,%
srcref,srcfilecopy,srcfile,sQuote,sprintf,%
split<-.default,split<-.data.frame,split<-,split.POSIXct,split.default,%
split.Date,split.data.frame,source.url,sort.POSIXlt,sort.list,%
sort.int,sort.default,solve.qr,solve.default,socketSelect,%
socketConnection,slice.index,sink.number,simpleWarning,simpleMessage,%
simpleError,simpleCondition,signalCondition,shQuote,showConnections,%
setTimeLimit,setSessionTimeLimit,setNamespaceInfo,setHook,setCConverterStatus,%
set.seed,serialize,seq.POSIXt,seq.int,seq.default,%
seq.Date,seq_len,seq_along,seek.connection,seek,%
scan.url,scale.default,saveNamespaceImage,save.image,sample.int,%
rowSums,rowsum.default,rowsum.data.frame,rownames<-,rowMeans,%
row.names<-.default,row.names<-.data.frame,row.names<-,row.names.default,row.names.data.frame,%
row.names,round.POSIXt,round.Date,RNGversion,rev.default,%
retracemem,restartFormals,restartDescription,replicate,rep.POSIXlt,%
rep.POSIXct,rep.numeric_version,rep.int,rep.factor,rep.Date,%
removeTaskCallback,removeCConverter,registerS3methods,registerS3method,reg.finalizer,%
Reduce,readLines,readChar,readBin,read.table.url,%
read.dcf,rcond,rbind.data.frame,rawToChar,rawToBits,%
rawShift,rawConnectionValue,rawConnection,raw,rapply,%
range.default,R.version.string,R.Version,R.version,R.home,%
R_system_version,quarters.POSIXt,quarters.Date,quarters,qr.X,%
qr.solve,qr.resid,qr.R,qr.qy,qr.qty,%
qr.Q,qr.fitted,qr.default,qr.coef,pushBackLength,%
pushBack,psigamma,prop.table,proc.time,printNoClass,%
print.warnings,print.table,print.summary.table,print.srcref,print.srcfile,%
print.simple.list,print.rle,print.restart,print.proc_time,print.POSIXlt,%
print.POSIXct,print.packageInfo,print.octmode,print.numeric_version,print.noquote,%
print.NativeRoutineList,print.listof,print.libraryIQR,print.hexmode,print.function,%
print.factor,print.DLLRegisteredRoutines,print.DLLInfoList,print.DLLInfo,print.difftime,%
print.default,print.Date,print.data.frame,print.connection,print.condition,%
print.by,print.AsIs,prettyNum,Position,pos.to.env,%
pmin.int,pmax.int,pipe,pi,path.expand,%
parseNamespaceFile,parse.dcf,parent.frame,parent.env<-,parent.env,%
packBits,packageStartupMessage,packageHasNamespace,packageEvent,package.description,%
package_version,Ops.POSIXt,Ops.ordered,Ops.numeric_version,Ops.factor,%
Ops.difftime,Ops.Date,Ops.data.frame,open.srcfilecopy,open.srcfile,%
open.connection,open,on.exit,oldClass<-,oldClass,%
nzchar,numeric_version,ngettext,new.env,Negate,%
namespaceImportMethods,namespaceImportFrom,namespaceImportClasses,namespaceImport,namespaceExport,%
names<-,mostattributes<-,months.POSIXt,months.Date,months,%
month.name,month.abb,mode<-,mget,message,%
merge.default,merge.data.frame,memory.profile,mem.limits,mean.POSIXlt,%
mean.POSIXct,mean.difftime,mean.default,mean.Date,mean.data.frame,%
max.col,Math.POSIXt,Math.factor,Math.difftime,Math.Date,%
Math.data.frame,match.fun,match.call,match.arg,mat.or.vec,%
margin.table,mapply,Map,manglePackageName,makeActiveBinding,%
make.unique,make.names,lower.tri,logb,lockEnvironment,%
lockBinding,loadURL,loadNamespace,loadingNamespaceInfo,loadedNamespaces,%
list.files,library.dynam.unload,library.dynam,lfactorial,levels<-.factor,%
levels<-,levels.default,LETTERS,letters,length<-.factor,%
length<-,lazyLoadDBfetch,lazyLoad,labels.default,La.svd,%
La.eigen,La.chol2inv,La.chol,l10n_info,kappa.tri,%
kappa.qr,kappa.lm,kappa.default,julian.POSIXt,julian.Date,%
julian,isTRUE,isSymmetric.matrix,isSymmetric,isSeekable,%
isS4,isRestart,isOpen,ISOdatetime,ISOdate,%
isNamespace,isIncomplete,isdebugged,isBaseNamespace,is.vector,%
is.unsorted,is.table,is.symbol,is.single,is.recursive,%
is.real,is.raw,is.R,is.qr,is.primitive,%
is.pairlist,is.package_version,is.ordered,is.object,is.numeric.POSIXt,%
is.numeric.Date,is.numeric_version,is.numeric,is.null,is.nan,%
is.name,is.na<-.factor,is.na<-.default,is.na<-,is.na.POSIXlt,%
is.na.data.frame,is.na,is.matrix,is.logical,is.loaded,%
is.list,is.language,is.integer,is.infinite,is.function,%
is.finite,is.factor,is.expression,is.environment,is.element,%
is.double,is.data.frame,is.complex,is.character,is.call,%
is.atomic,is.array,invokeRestartInteractively,invokeRestart,inverse.rle,%
intToUtf8,intToBits,importIntoEnv,identity,identical,%
icuSetCollate,iconvlist,iconv,gzfile,gzcon,%
grepl,gregexpr,gettextf,gettext,getTaskCallbackNames,%
getSrcLines,getRversion,getNumCConverters,getNativeSymbolInfo,getNamespaceVersion,%
getNamespaceUsers,getNamespaceName,getNamespaceInfo,getNamespaceImports,getNamespaceExports,%
getNamespace,getLoadedDLLs,getHook,getExportedValue,getDLLRegisteredRoutines.DLLInfo,%
getDLLRegisteredRoutines.character,getDLLRegisteredRoutines,getConnection,getCConverterStatus,getCConverterDescriptions,%
getCallingDLLe,getCallingDLL,getAllConnections,gc.time,formatDL,%
format.pval,format.POSIXlt,format.POSIXct,format.octmode,format.info,%
format.hexmode,format.factor,format.difftime,format.default,format.Date,%
format.data.frame,format.char,format.AsIs,formals<-,force,%
flush.connection,flush,findRestart,findPackageEnv,findInterval,%
Find,Filter,file.symlink,file.show,file.rename,%
file.remove,file.path,file.info,file.exists,file.create,%
file.copy,file.choose,file.append,file.access,fifo,%
factorial,F,expm1,expand.grid,eval.parent,%
environmentName,environmentIsLocked,environment<-,env.profile,Encoding<-,%
Encoding,encodeString,emptyenv,eapply,dyn.unload,%
dyn.load,duplicated.POSIXlt,duplicated.numeric_version,duplicated.matrix,duplicated.default,%
duplicated.data.frame,duplicated.array,dQuote,do.call,dir.create,%
dimnames<-.data.frame,dimnames<-,dimnames.data.frame,dim<-,dim.data.frame,%
difftime,diff.POSIXt,diff.default,diff.Date,diag<-,%
determinant.matrix,determinant,det,delayedAssign,default.stringsAsFactors,%
debugonce,data.matrix,data.frame,data.class,cut.POSIXt,%
cut.default,cut.Date,Cstack_info,contributors,conditionMessage.condition,%
conditionMessage,conditionCall.condition,conditionCall,computeRestarts,comment<-,%
colSums,colnames<-,colMeans,codes<-,codes.ordered,%
codes.factor,closeAllConnections,close.srcfile,close.connection,class<-,%
chol.default,check_tzones,chartr,charToRaw,char.expand,%
cbind.data.frame,casefold,capabilities,callCC,c.POSIXlt,%
c.POSIXct,c.numeric_version,c.noquote,c.Date,bzfile,%
by.default,by.data.frame,browserText,browserSetDebug,browserCondition,%
bquote,body<-,bindtextdomain,bindingIsLocked,bindingIsActive,%
baseenv,attributes<-,attr<-,attr.all.equal,attachNamespace,%
assign,asS4,asNamespace,as.vector.factor,as.vector,%
as.table.default,as.table,as.symbol,as.single.default,as.single,%
as.real,as.raw,as.qr,as.POSIXlt.POSIXct,as.POSIXlt.numeric,%
as.POSIXlt.factor,as.POSIXlt.default,as.POSIXlt.dates,as.POSIXlt.Date,as.POSIXlt.date,%
as.POSIXlt.character,as.POSIXlt,as.POSIXct.POSIXlt,as.POSIXct.numeric,as.POSIXct.default,%
as.POSIXct.dates,as.POSIXct.Date,as.POSIXct.date,as.POSIXct,as.pairlist,%
as.package_version,as.ordered,as.octmode,as.numeric_version,as.numeric,%
as.null.default,as.null,as.name,as.matrix.POSIXlt,as.matrix.noquote,%
as.matrix.default,as.matrix.data.frame,as.matrix,as.logical,as.list.numeric_version,%
as.list.function,as.list.factor,as.list.environment,as.list.default,as.list.data.frame,%
as.list,as.integer,as.hexmode,as.function.default,as.function,%
as.factor,as.expression.default,as.expression,as.environment,as.double.POSIXlt,%
as.double.difftime,as.double,as.difftime,as.Date.POSIXlt,as.Date.POSIXct,%
as.Date.numeric,as.Date.factor,as.Date.default,as.Date.dates,as.Date.date,%
as.Date.character,as.Date,as.data.frame.vector,as.data.frame.ts,as.data.frame.table,%
as.data.frame.raw,as.data.frame.POSIXlt,as.data.frame.POSIXct,as.data.frame.ordered,as.data.frame.numeric_version,%
as.data.frame.numeric,as.data.frame.model.matrix,as.data.frame.matrix,as.data.frame.logical,as.data.frame.list,%
as.data.frame.integer,as.data.frame.factor,as.data.frame.difftime,as.data.frame.default,as.data.frame.Date,%
as.data.frame.data.frame,as.data.frame.complex,as.data.frame.character,as.data.frame.AsIs,as.data.frame.array,%
as.data.frame,as.complex,as.character.srcref,as.character.POSIXt,as.character.octmode,%
as.character.numeric_version,as.character.hexmode,as.character.factor,as.character.error,as.character.default,%
as.character.Date,as.character.condition,as.character,as.call,as.array.default,%
as.array,anyDuplicated.matrix,anyDuplicated.default,anyDuplicated.data.frame,anyDuplicated.array,%
anyDuplicated,all.vars,all.names,all.equal.raw,all.equal.POSIXct,%
all.equal.numeric,all.equal.list,all.equal.language,all.equal.formula,all.equal.factor,%
all.equal.default,all.equal.character,all.equal,agrep,addTaskCallback,%
addNA%
},%
keywordstyle={[10]{\bf\color{RRecomdcolor}}}%
%
}%
\lstdefinestyle{Rstyle}{style=RstyleO2}

%------------------------------------------------------------------------------%
%
%% -------------------------------------------------------------------------------
\lstdefinestyle{TeXstyle}{fancyvrb=true,escapechar=`,language=[LaTeX]TeX,%
                        basicstyle={\color{black}\small},%
                        keywordstyle={\bf\color{black}},%
                        commentstyle={\color{Rcommentcolor}\ttfamily\itshape},%
                        literate={<-}{<-}2{<<-}{<<-}2}
%
% -------------------------------------------------------------------------------
\begin{document}
% -------------------------------------------------------------------------------
\title{How to work with SweaveListingUtils}
%,version \pkgExversion}
\author{\small Peter Ruckdeschel\thanks{Fraunhofer ITWM, Kaiserslautern}
\smallskip\\
\small Fraunhofer ITWM\\[-.5ex]
\small Fraunhofer Platz 1\\[-.5ex]
\small 67663 Kaiserslautern\\[-.5ex]
\small Germany\\
\small e-Mail: \href{mailto:Peter.Ruckdeschel@itwm.fraunhofer.de}%
{\small \tt {Peter.Ruckdeschel@itwm.fraunhofer.de}}\medskip\\
\parbox[t]{5cm}{
\footnotesize\sffamily
 Version control information:
\begin{tabbing}
\footnotesize\sffamily
 Last changes revision: \= \kill
 Head URL: \> \parbox[t]{6cm}{\url{\svnkw{HeadURL}}}\\[1.2ex]
 Last changed date: \> \svndate\\
 Last changes revision: \> \svnrev\\
 Version: \> \svnFullRevision*{\svnrev}\\
 Last changed by: \> \svnFullAuthor*{\svnauthor}\\
\end{tabbing}
}}
\maketitle
% -------------------------------------------------------------------------------
\begin{abstract}
% -------------------------------------------------------------------------------
In this vignette, we give short examples how to use package
\pkg{SweaveListingUtils}.
% -------------------------------------------------------------------------------
\end{abstract}
%
% -------------------------------------------------------------------------------
\section[What package SweaveListingUtils is for]{What package {\tt SweaveListingUtils} is for}
% -------------------------------------------------------------------------------
Package \pkg{SweaveListingUtils}
provides utilities for combining {\tt Sweave}, confer \citet{Lei:02a,Lei:02b,Lei:03},
with functionality of \TeX-package-\file{listings}, confer \citet{H:M:07}.
In particular, we define {\sf R} / {\tt Rd} as \TeX-package-\file{listings}
"language" and functionality to include {\sf R} / {\tt Rd} source file
(sniplets) copied from an url, by default from the {\tt svn} server at
\href{http://r-forge.r-project.org/}{R-forge}, confer \citet{RForge} in
its most recent version, thereby avoiding inconsistencies between vignette and
documented source code.
In this respect it supports (and to some extent enhances) {\tt Sweave}.
%
% -------------------------------------------------------------------------------
\section{Preparations: Preamble}\label{PrepSec}
% -------------------------------------------------------------------------------
You should include into the preamble of your \file{.Rnw} file something like
%
%------------
% use \lstinputlisting{....}
% to avoid that Sweave interprets the chunks in preamble.tex
%------------
\lstinputlisting[style=TeXstyle]{preamble.tex}
Actually, after \code{Sweave}-ing the \file{.Rnw} file
to a corresponding \file{.tex} file,  will expand to a rather long form
(depending on which packages you have attached), but you should not worry
about your document getting very long, as the inserted \TeX commands
(or more precisely {\tt listings}-package commands) only declare
the list(s) of registered keywords (for later markup).\\
In our case this should expand to something like\newline
%
\begin{footnotesize}
%
%------------
% use \lstinputlisting{....}
% to avoid that Sweave interprets the chunks in preambleExp.tex
%------------
\lstinputlisting[style=TeXstyle,basicstyle=\footnotesize]{preambleExp.tex}
%
\end{footnotesize}
Finally to clean up things (in particular unmask the base functions
\code{require} and \code{library} again) at the end of your document
you should append something like
%------------
% use \lstinputlisting{....}
% to avoid that Sweave interprets the chunks in cleanup.tex
%------------
\lstinputlisting[style=TeXstyle]{cleanup.tex}
%

%
\paragraph{Remark:} As suggested by \href{mailto:ellis@finance.ch}{Andrew Ellis},
ETH Z{\"u}rich, \code{SweaveListingPreparations} from version {\tt 0.3}
has two more options: First, by setting argument \code{withOwnFileSection}
(default \code{FALSE}), one can have one's own definition of \LaTeX\
environments for \texttt{Sinput}, \texttt{Soutput}, \texttt{Scode}, be it in
an extra file or in a section in one's \file{.Rnw} file. Second,
and this is Andrew's suggestion, by
means of argument \code{withVerbatim} (default \code{FALSE}), you may
from now on use \LaTeX\
environments for \texttt{Sinput}, \texttt{Soutput}, \texttt{Scode} using
{\tt listings}-command \lstinline[style=TeXstyle,basicstyle=\tt]|\lstnewenvironment|
instead of the original {\tt fancyvrb} definitions provided in the original
\file{Sweave.sty} file by Fritz Leisch. This way we also solve the
escaping problem (as noted by Frank E. Harrell): the escaping mechanisms
provided by {\tt listings}  command {\tt lstset} (as e.g. {\tt escapechar},
{\tt escapeinline}) as described in detail in \cite[section~4.14]{H:M:07}
are now available; in particular one can place \LaTeX\ references
\lstinline[style=TeXstyle,basicstyle=\tt]|\ref{...}|,
\lstinline[style=TeXstyle,basicstyle=\tt]|\label{...}| within comments.\newline
Just to show some little example:
%------------
% use \lstinputlisting{....}
% to avoid that Sweave interprets the chunks in exam1.tex
%------------
\lstinputlisting[style=TeXstyle]{exam00.tex}
becomes
\lstdefinestyle{Rinstyle}{style=RinstyleO,frame=trBL,backgroundcolor=\color{gray90},%
        numbers=left,numberstyle=\tiny,stepnumber=1,numbersep=7pt}
\lstdefinestyle{Routstyle}{style=RoutstyleO,frame=trBL,frameround=fttt,%
        backgroundcolor=\color{gray95},numbers=left,numberstyle=\tiny,%
        stepnumber=3,numbersep=5pt}
\begin{quotation}
\begin{Schunk}
\begin{Sinput}
> x <- rnorm(3) # define random numbers -> insert a label <here> (*\label{comment}*)
> print(round(x,2))
\end{Sinput}
\begin{Soutput}
[1]  0.15  0.32 -0.08
\end{Soutput}
\begin{Sinput}
> f <- function(x) sin(x) ## here is a ref: section(*~\ref{PrepSec}*)
> a <- 2; b <- 3
> # compute (*$\int_a^b f(x)\,dx$*)
> integrate(f, lower=a, upper=b)$value
\end{Sinput}
\begin{Soutput}
[1] 0.5738457
\end{Soutput}
\end{Schunk}
Note that line~\ref{comment} contains a comment.
\end{quotation}
\lstdefinestyle{Rinstyle}{style=RinstyleO,frame=none,backgroundcolor=\color{white},%
        numbers=none}
\lstdefinestyle{Routstyle}{style=RoutstyleO,frame=none,backgroundcolor=\color{white},%
        numbers=none}
Note that for easier reference later on, we produce copies of the original
{\tt listings} styles {\tt Rinstyle}, {\tt Routstyle}, {\tt Rcodestyle}, named
                      {\tt RinstyleO}, {\tt RoutstyleO}, {\tt RcodestyleO},
                      respectively.
If you really want to have background colors, frames and the like, however,
you might wish to specify the corresponding options in the preparational
chunk to have something like.
%------------
% use \lstinputlisting{....}
% to avoid that Sweave interprets the chunks in preamble2.tex
%------------
\lstinputlisting[style=TeXstyle]{preamble2.tex}

% -------------------------------------------------------------------------------
\section{Listings markup}
% -------------------------------------------------------------------------------
\subsection{Example of code coloring}
% -------------------------------------------------------------------------------
Any keyword of some new {\sf R} package ``loaded in'' by \code{require}
or \code{library} which is on the \code{search} list item of this package
afterwords when used in \lstinline[style=TeXstyle,basicstyle=\tt]`\lstinline{ .... }` or
\lstinline[style=TeXstyle,basicstyle=\tt]`\begin{lstlisting} .... \end{lstlisting}`
or in some Sweave chunk is typeset in style \code{keywordstyle}.
More specifically, with argument \code{keywordstyles} of functions
\code{setToBeDefinedPkgs} or \code{lstsetLanguage}
all packages may obtain their own style; in the preamble, for instance,
package \pkg{SweaveListingUtils} is colored blue, and package
\pkg{distr} (to be attached just now) will be colored red.
Also, comments are set in a different style (by default using color
\code{Rcommentcolor}). Of course, instead of colors, you may use any other markup,
like different font shapes, fonts, font sizes or whatever comes into your
mind. For this purpose, commands \code{setToBeDefinedPkgs} and
\code{changeKeywordstyles} are helpful.

Note that in order to define these new keywords correctly,
they must not be included into a
\lstinline[style=TeXstyle,basicstyle=\tt]`\begin{Schunk} .... \end{Schunk}`
environment, so  we use
%------------
% use \lstinputlisting{....}
% to avoid that Sweave interprets the chunks in reqdistr.tex
%------------
\lstinputlisting[style=TeXstyle]{reqdistr.tex}

\lstdefinestyle{RstyleO3}{style=RstyleO2,%
% --------------------------
% Registration of package distr
% --------------------------
morekeywords={[11]width<-,width,Weibull,UnivarMixingDistribution,UnivarLebDecDistribution,%
UnivarDistrList,Unif,Truncate,Td,support,%
SummandsDistr,standardMethods,size<-,size,simplifyr,%
simplifyD,showobj,shape2<-,shape2,shape1<-,%
shape1,shape<-,shape,setgaps,sdlog<-,%
sdlog,sd<-,scale<-,RtoDPQ.LC,RtoDPQ.d,%
RtoDPQ,Reals,rate<-,rate,r.discrete,%
r.ac,r,q.r,q.discrete,q.ac,%
prob<-,prob,Pois,pivot<-,pivot,%
param,p.l,p.discrete,p.ac,p,%
NumbOfSummandsDistr,Norm,ncp<-,ncp,Nbinom,%
Naturals,name<-,name,n<-,n,%
mixDistr<-,mixDistr,mixCoeff<-,mixCoeff,Minimum,%
Min<-,Min,meanlog<-,meanlog,mean<-,%
Maximum,Max<-,Max,makeAbscontDistribution,m<-,%
m,Logis,location<-,location,Lnorm,%
liesInSupport,liesIn,Length<-,Length,LatticeDistribution,%
Lattice,lattice,lambda<-,lambda,k<-,%
k,isOldVersion,img,Hyper,Huberize,%
getUp,getLow,getLabel,getdistrOption,Geom,%
gaps<-,gaps,Gammad,flat.mix,flat.LCD,%
Fd,Exp,EuclideanSpace,distroptions,distrMASK,%
DistrList,distrARITH,discreteWeight<-,discreteWeight,discretePart<-,%
discretePart,DiscreteDistribution,Dirac,dimension<-,dimension,%
df2<-,df2,df1<-,df1,df<-,%
DExp,devNew,decomposePM,d.discrete,d.ac,%
d,convpow,conv2NewVersion,CompoundDistribution,Chisq,%
Cauchy,Binom,Beta,Arcsine,acWeight<-,%
acWeight,acPart<-,acPart,AbscontDistribution%
},%
keywordstyle={[11]\bf\color{red}},%
%
% --------------------------
% Registration of package sfsmisc
% --------------------------
morekeywords={[12]xy.unique.x,xy.grid,wrapFormula,vcat,uniqueL,%
unif,u.sys,u.log,u.get0,u.Datumvonheute,%
u.datumdecode,u.date,u.boxplot.x,u.assign0,tkdensity,%
tapplySimpl,TA.plot,Sys.sizes,Sys.ps.cmd,Sys.ps,%
strcodes,str_data,signi,sHalton,seqXtend,%
rrange,roundfixS,rot2,rnls,QUnif,%
quadrant,ps.latex,ps.end,ps.do,prt.DEBUG,%
printTable2,pretty10exp,posdefify,polyn.eval,pmin.sa,%
pmax.sa,plotStep,plotDS,pl.ds,pdf.latex,%
pdf.end,pdf.do,paste.vec,p.ts,p.tachoPlot,%
p.scales,p.res.2x,p.res.2fact,p.profileTraces,p.pllines,%
p.m,p.hboxp,p.dnorm,p.dgamma,p.dchisq,%
p.datum,p.arrows,nr.sign.chg,nearcor,n.plot,%
n.code,mult.fig,mpl,mat2tex,margin2table,%
lseq,list2mat,linesHyperb.lm,last,KSd,%
iterate.lin.recursion,inv.seq,integrate.xy,ichar,hist.bxp,%
hatMat,f.robftest,errbar,empty.dimnames,ellipsePoints,%
ecdf.ksCI,eaxis,Duplicated,digitsBase,diagX,%
diagDA,dDA,D2ss,D1tr,D1ss,%
D1D2,cum.Vert.funkt,compresid2way,colcenter,col01scale,%
code2n,chars8bit,ccat,C.Wochentagkurz,C.Wochentag,%
C.weekday,C.Monatsname,boxplot.matrix,bl.string,axTexpr,%
AsciiToInt,as.integer.basedInt,as.intBase%
},%
keywordstyle={[12]{\bf\color{Rbcolor}}}%
%
}%
\lstdefinestyle{Rstyle}{style=RstyleO3}
The next example takes up package \pkg{distr}, confer \citet{R:K:S:C:04},
to illustrate particular markup for a particular package.

%
Example (note the different colorings):
%------------
% use \lstinputlisting{....}
% to avoid that Sweave interprets the chunks in exam1.tex
%------------
\lstinputlisting[style=TeXstyle]{exam1.tex}
%
which gives
\begin{Schunk}
\begin{Sinput}
> require(distr)
> N <- Norm(mean = 2, sd = 1.3)
> P <- Pois(lambda = 1.2)
> Z <- 2*N + 3 + P
> Z
\end{Sinput}
\begin{Soutput}
Distribution Object of Class: AbscontDistribution
\end{Soutput}
\begin{Sinput}
> p(Z)(0.4)
\end{Sinput}
\begin{Soutput}
[1] 0.002415387
\end{Soutput}
\begin{Sinput}
> q(Z)(0.3)
\end{Sinput}
\begin{Soutput}
[1] 6.705068
\end{Soutput}
\end{Schunk}

{\bf Remark: }{\tt .Rd} keywords will be taken from file \file{Rdlistings.sty}
in the \file{TeX} subfolder of this package,
which is according to \citet{Murd:08}.
% -------------------------------------------------------------------------------
\subsection{Changing the markup}
% -------------------------------------------------------------------------------
%
Triggered by an e-mail by David Carslaw, this subsection lists some possibilities
howto change the (default) markup of code.

\paragraph{Changing the global settings: }
%
The default markup for {\sf R} code is set in a global option
\code{Rset} to be inspected by
\begin{Schunk}
\begin{Sinput}
> getSweaveListingOption("Rset")
\end{Sinput}
\begin{Soutput}
$fancyvrb
[1] "true"

$escapechar
[1] "`"

$extendedchars
[1] "false"

$language
[1] "R"

$basicstyle
[1] "{\\color{Rcolor}\\small}"

$keywordstyle
[1] "{\\bf\\color{Rcolor}}"

$commentstyle
[1] "{\\color{Rcommentcolor}\\ttfamily\\itshape}"

$literate
[1] "{<-}{{$\\leftarrow$}}2{<<-}{{$\\twoheadleftarrow$}}2{~}{{$\\sim$}}1{<=}{{$\\leq$}}2{>=}{{$\\geq$}}2{^}{{$\\scriptstyle\\wedge$}}1"

$alsoother
[1] "{$}"

$alsoletter
[1] "{.<-}"

$otherkeywords
[1] "{!,!=,~,$,*,\\&,\\%/\\%,\\%*\\%,\\%\\%,<-,<<-,/}"

$escapeinside
[1] "{(*}{*)}"
\end{Soutput}
\end{Schunk}

Similarly, default markup for {\tt Rd} code is set in a global option
\code{Rset} to be inspected by
\begin{Schunk}
\begin{Sinput}
> getSweaveListingOption("Rdset")
\end{Sinput}
\begin{Soutput}
$fancyvrb
[1] "true"

$language
[1] "Rd"

$keywordstyle
[1] "{\\bf}"

$basicstyle
[1] "{\\color{black}\\footnotesize}"

$commentstyle
[1] "{\\ttfamily\\itshape}"

$alsolanguage
[1] "R"
\end{Soutput}
\end{Schunk}

The inspection / modification mechanism for these global options
is the same as for the {\sf R} global options, i.e., instead of
the functions \code{options}, \code{getOption}, we have functions
\code{SweaveListingoptions}, \code{getSweaveListingOption}; see also
\code{?getSweaveListingOption}.\medskip\\

\noindent Some comments are due:

The items of this list are just the tagged
{\tt name = value} list items to be passed as arguments to
(\TeX-){\tt listings} command {\tt lstset}, and you may include any
{\tt name = value} pair allowed for \lstset{}. For details confer the
documentation of the {\tt listings} package, \citet{H:M:07}.

As usual in {\sf R}, backslashes have to be escaped,
giving the double appearance in the examples listed above.

For cooperation of  {\tt listings}
with {\tt Sweave}, it is necessary, however, to use the tagged pair
{\tt "fancyvrb" = "true"}.

The colors used in the default setting
are also set as global ({\tt SweaveListing}-)options ---
i.e.; \code{Rcolor}, \code{Rcommentcolor}, \code{Rdcolor}.

Item {\tt "literate"} will be discussed in the next subsection.

Using the escape character defined as item {\tt "escapechar"},
you may force \TeX to typeset (parts of) your comments in \TeX style,
which is handy for mathematical formula.

In an e-mail, Frank Harrell suggested to use {\sf R} color names
to assign markup colors as \code{Rcolor}.
As in \TeX we are just using the command \lstinline[language=tex]{\color}
which expects a comma-separated list of the three rgb coordinates (scaled to be
in $[0,1]$), a good way to do this is, as Frank suggested,
to use \code{col2rgb(...)/255} to transform them to \TeX-digestible format.

\paragraph{Changing the markup settings without changing defaults at startup: }
%
Alternatively, you may change global markup without modifying
({\tt SweaveListing}-)option  \code{"Rset"}. To this end you may build
up your own (local) \code{"Rset"}-list,
say \code{Rset0}. This is most easily done by first copying the global default
list and then by modifying some items by simple {\sf R} list operations.
This might give the following alternative preparatory chunk to
be inserted at the beginning of your {\tt .Rnw} file.
%------------
% use \lstinputlisting{....}
% to avoid that Sweave interprets the chunks in altPrep.tex
%------------
\lstinputlisting[style=TeXstyle]{altPrep.tex}

\paragraph{Changing the markup locally: }
If you want to change the markup style within some {\tt .Rnw} file, use
something like:

%------------
% use \lstinputlisting{....}
% to avoid that Sweave interprets the chunks in locChange.tex
%------------
\lstinputlisting[style=TeXstyle]{locChange.tex}

This will add/replace item \code{"basicstyle"} to/in the existing items.
For {\tt Rd}-style you may use a respective call to \code{lstsetRd()},
and if you only want to modify the {\tt Sinput}, {\tt Soutput}, or {\tt Scode}
environments, you may use respective calls to \code{lstsetRin}, \code{lstsetRout},
\code{lstsetRcode}, respectively.

%
% -------------------------------------------------------------------------------
\subsection{Using literate programming}
% -------------------------------------------------------------------------------
%
This is ---to some degree--- a matter of taste: {\sf R} has two assignment
operators, which when typed look like {\color{Rcolor}\tt <-} and
{\color{Rcolor}\verb|<<-|}; now literally these are interpreted as one token;
the same goes for comparison operators like {\color{Rcolor} \tt <=}.
One idea of literate programming is to replace these tokens by special
symbols like $\leftarrow$, $\twoheadleftarrow$, $\le$ for printing to enhance
readability --- think of easy confusions arising between {\color{Rcolor}\tt <-}
and {\color{Rcolor}\tt < -}.

\TeX-package {\tt listings}, confer \citet{H:M:07}, to this end has the directive
{\tt literate}, and in
our default setting for {\sf R} markup, we use it at least for the replacement
of the assignments.

Note that the {\tt .Rnw} file still contains valid {\sf R} code
in the chunks; \code{stangle} will work just fine --- the chunks are just output
by \TeX\ in a somewhat transformed way.

A considerable part of {\sf R} users would rather prefer to see the code output
``as you type it''; if you tend to think you like this, you are free of course
to change the default markup as described in the previous section.
%
%
%
% -------------------------------------------------------------------------------
\section{Including Code Sniplets from R Forge}
% -------------------------------------------------------------------------------
When documenting code, which is not necessarily of the same package, and be it
{\sf R} code or {\tt .Rd}-code, we provide helper functions to integrate code
sniplets from an url (by default, we use the svn server at R-forge in its
most recent version). This can be useful to stay consistent with the current
version of the code without having to update vignettes all the time.
To this end, besides referencing by line numbers,
\code{lstinputSourceFromRForge} also offers
referencing by matching regular expressions.


For instance, to refer to some code of file \file{R/AllClasses.R}
in package \pkg{distr}, we would use:
%
%------------
% use \lstinputlisting{....}
% to avoid that Sweave interprets the chunks in AllClass.tex
%------------
\lstinputlisting[style=TeXstyle]{AllClass.tex}

% -------------------------------------------------------------------------------
which returns

